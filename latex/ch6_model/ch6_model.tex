\chapter{Numerical model}

\section{Introduction}
Dans ce chapitre on s'attache à la résolution de l'équilibre statique par relaxation dynamique.
Peut-être qu'il faudra faire un chapitre "Marsupilami" plus détaillé qui regroupe les différents éléments formulés et les connexions / liaisons
On formule l'élément en torsion à partir des équations de Kirchhoff et non du modèle énergétique (un terme est manquant).
On effectue une validation numérique. Penser à tracer la contribution de chaque terme pour vérifier les approximations effectuées.

\cite{Dill1992}


% #######################################################
\section{Discret curve-angle representation}
% #######################################################

\subsection{Discrete centerline}
On discrétise la centerline de la façon suivante :
\begin{subequations}
	\begin{gather}
		s_i = \sum\limits_0^i |\vect{e}_{i-1}|  \quad , \quad i=0..n \\
		\vect{x}_i \quad , \quad i=0..n \\
		\vect{e}_i = \vect{x}_{i+1}-\vect{x}_i \quad , \quad i=0..n-1
	\end{gather}
\end{subequations}

\subsection{Discrete Bishop frame}

\subsubsection{Discrete parallel transport at edges}

\subsubsection{Discrete parallel at vertices}

On définit ici le transport parrallele discret entre edges et entre vertices.
A chaque vertex, on associe un repère de bishop et un repère matériel
\begin{subequations}
	\begin{gather}
		\theta_i \quad , \quad i=0..n \\
		\{\vect{u}_i, \vect{v}_i\} \quad , \quad i=0..n \\
		\{\vect{d}^1_{i}, \vect{d}^2_{i}\} \quad , \quad i=0..n
	\end{gather}
\end{subequations}

\subsection{Discrete material frame}


\section{Interpolation rules}

On fait des hypothèses sur la forme des efforts pour pouvoir effectuer une interpolation entre les noeuds.
Thus, we define edge quantities and vertex quantities.

\subsection{Geometric and material properties}
We suppose that the sections are defined at verticies and that section properties remain uniform over $]s_{i-1/2}, s_{i+1/2}[$. Thus, the material and geometric properties ($P$) of the beam ($E_i, G_i, S_i, I^1_{i}, I^2_{i}, J_i, m_i$) are supposed to be piecewise constant functions of $s$ on [0,L] :
\begin{equation}
	P(s) = P_i \quad , \quad s \in [s_{i-1/2},s_{i+1/2}]
\end{equation}
Note that this functions may be discontinous at edges midspan ($s_{i+1/2}$).

\subsection{Axial force}
$\vect{N}$ is supposed to be piecewise constant between verticies on $[0,L]$ :
\begin{equation}
	\vect{N}(s) = N_i \vect{t}_i , \quad s \in ]s_i,s_{i+1}[
\end{equation}
Note that this functions may be discontinous at edges midspan ($s_{i+1/2}$).

\subsection{Torsion moment}
$\vect{Q}$ is supposed to be piecewise constant between verticies on $[0,L]$ :
\begin{equation}
	\vect{Q}(s) = Q_i \vect{t}_i , \quad s \in ]s_i,s_{i+1}[
\end{equation}
Note that this function may be discontinous at edges midspan ($s_{i+1/2}$).

\subsection{Bending moment}
$\vect{M}$ is supposed to be continuous and piecewise linear on $[0,L]$. This assumption is quite reasonable because the bending moment is effectively continous for a beam subject to punctual forces and moments. Thus, $\vect{M}$ is interpolated from the moment computed at vertices :
\begin{equation}
	\vect{M}(s) = \vect{M}_{i+1/2} + (s-\tfrac{l_i}{2}) \vect{M}'_{i+1/2} \quad , \quad s \in [s_i,s_{i+1}]
\end{equation}
With
\begin{subequations}
	\begin{gather}
	\vect{M}_{i+1/2} = \frac{\vect{M}_{i} + \vect{M}_{i+1}}{2} \\
	\vect{M}'_{i+1/2} = \frac{\vect{M}_{i+1} - \vect{M}_{i}}{l_i}
	\end{gather}
\end{subequations}

\subsection{Curvature}
Il y a une sorte de dualité entre le moment et la courbure.
Parfois c'est la mesure de la courbure qui donne accès au moment de flexion.
Parfois la courbure c'est la connaissance du moment de flexion qui donne la courbure.
Si le moment est continu et linéaire par morceaux, la courbure elle est seulement linéaire par morceau.
En effet, il peut y a voir un saut de courbure entre deux éléments de $EI$ distinct.

On interpole la courbure à partir des courbures discrètes aux noeuds et des $EI$ (également définis aux noeuds) de part et d'autre des noeuds.
On revient à la continuité du moment. Puis on déduit la courbure du moment.

Donc au repos, on considère que $\mat{B}\kappa\vect{b}$ est continu et linéaire par morceaux, qui donne l'interpolation de la courbure suivante :
\begin{equation}
	\mat{B}\kappa\vect{b}(s) =
	\frac{\mat{B}_{i}\kappa\vect{b}_{i} + \mat{B}_{i+1}\kappa\vect{b}_{i+1}}{2}
	+ (s-\tfrac{l_i}{2}) \frac{\mat{B}_{i+1}\kappa\vect{b}_{i+1} - \mat{B}_{i}\kappa\vect{b}_{i}}{l_i}
	\quad , \quad s \in [s_i,s_{i+1}]
\end{equation}

On défnit alors les courbures binormales à gauche ($\kappa\vect{b}_{i+1/2}^-$) et à droite ($\kappa\vect{b}_{i+1/2}^+$) du point $\vect{x}_{i+1/2}$ tel que :
\begin{subequations}
	\begin{gather}
		\kappa\vect{b}_{i+1/2}^- = \frac{\kappa\vect{b}_{i} + \mat{A}_i \kappa\vect{b}_{i+1}}{2} \\
		\kappa\vect{b}_{i+1/2}^+ = \frac{{\mat{A}_{i}}^{-1}\kappa\vect{b}_{i} + \kappa\vect{b}_{i+1}}{2}
	\end{gather}
\end{subequations}
Où l'on a posé $\mat{A}_i = {\mat{B}_{i}}^{-1}\mat{B}_{i+1}$, la matrice qui représente le saut des propriétés matérielles de flexion entre les noeuds $i$ et $i+1$. Comme attendu, $\mat{A}_i$ vaut l'identité lorsque $\mat{B}_i = \mat{B}_{i+1}$.

De même, on défnit la dérivée de la courbure binormale à gauche (${\kappa\vect{b}'}^{-}_{i+1/2}$) et à droite (${\kappa\vect{b}'}^{+}_{i+1/2}$) du point $\vect{x}_{i+1/2}$ tel que :
\begin{subequations}
	\begin{gather}
		{\kappa\vect{b}'}^{-}_{i+1/2} = \frac{\mat{A}_i\kappa\vect{b}_{i+1} - \kappa\vect{b}_{i}}{l_i} \\
		{\kappa\vect{b}'}^{+}_{i+1/2} = \frac{\kappa\vect{b}_{i+1} - {\mat{A}_i}^{-1}\kappa\vect{b}_{i}}{l_i}
	\end{gather}
\end{subequations}
Ainsi, on peut donner les lois d'interpolation des courbures en tout point de la centerline, dans n'importe quelle configuration (repos ou déformée) :
\begin{subequations}
	\begin{gather}
		\kappa\vect{b}(s) = \kappa\vect{b}_{i+1/2}^- + (s-\tfrac{l_i}{2}) {\kappa\vect{b}'}^{-}_{i+1/2}
		\quad s \in [s_i,s_{i+1/2}[ \\
		\kappa\vect{b}(s) = \kappa\vect{b}_{i+1/2}^+ + (s-\tfrac{l_i}{2}) {\kappa\vect{b}'}^{+}_{i+1/2}
		\quad s \in ]s_{i+1/2}, s_{i+1}]
	\end{gather}
\end{subequations}
Dans le cas où les propriétés de la poutre sont conservées ($\mat{A}_i = \mat{I}$), on écrira plus simplement:
\begin{equation}
	\kappa\vect{b}(s) = \kappa\vect{b}_{i+1/2} + (s-\tfrac{l_i}{2}) \kappa\vect{b}'_{i+1/2} \quad , \quad s \in [s_i,s_{i+1}]
\end{equation}
With
\begin{subequations}
	\begin{gather}
		\kappa\vect{b}_{i+1/2} = \kappa\vect{b}_{i+1/2}^- = \kappa\vect{b}_{i+1/2}^+ =
		\frac{\kappa\vect{b}_{i} + \kappa\vect{b}_{i+1}}{2} \\
		\kappa\vect{b}'_{i+1/2} = {\kappa\vect{b}'}_{i+1/2}^- = {\kappa\vect{b}'}_{i+1/2}^+ =
		\frac{\kappa\vect{b}_{i+1} - \kappa\vect{b}_{i}}{l_i}
	\end{gather}
\end{subequations}

\note{En pratique, si la correction apportée n'est pas pertinente, on négligera le saut de courbure dans les calculs discrets.}

\subsection{Discretization}

\subsubsection{Constitutive Equations}
\begin{subequations}
	\begin{gather}
	\tau = \theta' \\
	Q = ES\epsilon \\
	Q = GJ(\tau -{\overbar{\tau}}) \\
	M1 = EI_1(\kappa_1-{\overbar{\kappa_1}}) \\
	M1 = EI_2(\kappa_2-{\overbar{\kappa_2}}) \\
	\vect{M} = M_1\vect{d}_1 + M_2\vect{d}_2 \\
	\vect{Q} = Q\vect{d}_3
	\end{gather}
\end{subequations}

Notes :
\begin{subequations}
	\begin{gather}
	\vect{d}_3 \cdot (\vect{M} \times \vect{d}_3) = \kappa_1 M_1 + \kappa_2 M_1 \\
	\vect{d}_3 \cdot (\kappa\vect{b} \times \vect{M}) = \kappa_1 M_2 - \kappa_2 M_1 \\
	(\vect{\omega} - \overbar{\vect{\omega}})^T \mat{B} \mat{J}\vect{\omega}
	 = - \kappa\vect{b} \cdot (\vect{d}_3 \times \mat{M})
	 = \kappa_1 \mat{M}_2 -  \kappa_2 \mat{M}_1
	\end{gather}
\end{subequations}

\subsubsection{Axial Force}
\begin{equation}
	\vect{N} = (ES \epsilon) \vect{d}_3
\end{equation}

\subsubsection{Shear Force}
\begin{equation}
	\vect{T}= \vect{d}_3 \times \vect{M}'
		  + Q\kappa\vect{b} - \tau \vect{M}
\end{equation}
\subsubsection{Rotational Moment}
\begin{equation}
	\vect{\Gamma}(s) = (Q' + \kappa_1 M_2 - \kappa_2 M_1)\vect{d}_3
	= Q' \vect{d}_3 + \kappa\vect{b} \times \vect{M}
\end{equation}
\subsubsection{Quasistatic hypothesis}
\begin{equation}
	Q' + \kappa_1 M_2 - \kappa_2 M_1 \simeq 0
\end{equation}
\subsubsection{Inextensibility hypothesis}
\begin{subequations}
	\begin{gather}
	\|\vect{x}'\| = \|(\vect{x} + \vect{\epsilon})'\| = 1 \\
	\|\vect{e}_i\| = \|\overbar{\vect{e}_i}\|
	\end{gather}
\end{subequations}

En fait il faut faire quelques hypothèses sur la nature des efforts pour pourvoir les interpoler convenablement le long de la courbe. On va faire qqch qui ressemble à la super-clothoide de Bertails, et qui semble une hypothèse naturelle pour une poutre continue sur plusieurs appuis soumise à des forces et moments ponctuels :
\begin{itemize}
\item le moment est continu et linéaire par morceaux. Il est évalué ponctuellement aux noeuds et interpolé linéairement entre les noeuds.
\item la courbure est donc continue par morceaux et linéaire par morceaux. Elle est obtenue à partir du moment et de $\vect{B}$
\item N et Q sont constants par morceaux
\end{itemize}
\begin{subequations}
	\begin{gather}
		\vect{N}(s) = \vect{N}_{i}  \quad , \quad s \in ]0,|\vect{e}_i|[ \\
		\vect{M}(s) = \vect{M}_{i} + s \vect{M}'_{i} \quad s \in [0,|\vect{e}_i|]
		\quad , \quad \vect{M}'_{i}= \frac{\vect{M}_{i+1} - \vect{M}_{i}}{|\vect{e}_i|} \\
		\vect{Q}(s) = \vect{Q}_{i}  \quad , \quad s \in ]0,|\vect{e}_i|[
	\end{gather}
\end{subequations}

Faire un tableau vertex / edge quantities :
\begin{itemize}
\item les propriétés mécaniques sont définies aux noeuds
\item les repères matériels sont définis aux noeuds
\item les courbures sont définies aux noeuds
\item les moments sont définis aux noeuds et sont interpolés linéairement entre les noeuds
\item l'effort normal et le moment de torsion sont supposés uniformes sur les segments (ils sont donc définis aux segments)
\item $\alpha$ et $\beta$ sont des propriétés équivalentess définies sur les segments (égales à celles de noeuds s'il n'y a pas de saut de propriété).
\end{itemize}

\subsection{Force}

\subsubsection{Axial Force}

Axial Force exercée par la poutre sur le point courant $\vect{x}_i$ :
\begin{equation}
	\begin{aligned}
	\vect{F}_i^{\parallel}
	&= \left[\vect{N}\right]_{i-1/2}^{i+1/2}
	= \vect{N}_{i+1} - \vect{N}_{i}
	\end{aligned}
\end{equation}

Axial Force exercée par la poutre sur le premier point $\vect{x}_0$ :
\begin{equation}
	\begin{aligned}
	\vect{F}_i^{\parallel}
	&= \left[\vect{N}\right]_{0}
	= \vect{N}_{0}
	\end{aligned}
\end{equation}

Axial Force exercée par la poutre sur le dernier point $\vect{x}_n$ :
\begin{equation}
	\begin{aligned}
	\vect{F}_i^{\parallel}
	&= -\left[\vect{N}\right]_{0}
	= -\vect{N}_{n}
	\end{aligned}
\end{equation}



\subsubsection{Shear Force}

Shear Force exercée par la poutre sur le point courant $\vect{x}_i$ :
\begin{equation}
	\begin{aligned}
	\vect{F}_i^{\perp}
	&= \left[\vect{d}_3 \times \mat{M}' + Q\kappa\vect{b} \right]_{i-1/2}^{i+1/2} \\
	&= 	\frac{\vect{e}_i}{|\vect{e}_i|} \times \frac{\mat{M}_{i+1} - \mat{M}_{i}}{|\vect{e}_i|}
		- \frac{\vect{e}_{i-1}}{|\vect{e}_{i-1}|} \times \frac{\mat{M}_{i} - \mat{M}_{i-1}}{|\vect{e}_{i-1}|}
		+ Q_i\kappa\vect{b}^{-}_{i+1/2}
		- Q_{i-1}\kappa\vect{b}^{+}_{i-1/2}\\
	&= (\vect{F}^1_{i} + \vect{F}^2_{i} + \vect{H}^{-}_{i}) - (\vect{F}^1_{i-1} + \vect{F}^2_{i-1} +  \vect{H}^{+}_{i-1})
	\end{aligned}
\end{equation}

\note{Ici, on a un problème de définition de $\kappa\vect{b}$ en milieu de segment. En effet, bien que le moment soit continu, la courbure ne l'est pas nécessairement. Lorsqu'il y a un saut de $EI$ il y a nécessairement un saut de $\kappa\vect{b}$. On pourrait plutôt interpoler le moment à mi-travée et remonter à la courbure - soit à gauche, soit à droite - en fonction des propriétés géométriques locales :
\begin{subequations}
	\begin{gather}
	\vect{M}_{i+1/2} = \frac{\vect{M}_{i} + \vect{M}_{i+1}}{2} \\
	\kappa\vect{b}^{-}_{i+1/2} = \mat{B}_i^{-1}\vect{M}_{i+1/2} + \overbar{\kappa\vect{b}}^{-}_{i+1/2}\\
	\kappa\vect{b}^{+}_{i+1/2} = \mat{B}_{i+1}^{-1}\vect{M}_{i+1/2} + \overbar{\kappa\vect{b}}^{+}_{i+1/2}\\
	\vect{H}^-_{i} = Q_i\kappa\vect{b}^{-}_{i+1/2}
	= Q_i (\mat{B}_i^{-1}\vect{M}_{i+1/2} + \overbar{\kappa\vect{b}}^{-}_{i+1/2})\\
	\vect{H}^+_{i} = Q_i\kappa\vect{b}^{+}_{i+1/2}
	= Q_i (\mat{B}_{i+1}^{-1}\vect{M}_{i+1/2} + \overbar{\kappa\vect{b}}^{-}_{i+1/2})
	\end{gather}
\end{subequations}

L'autre approche consiste à ignorer la discontinuité et à simplement prendre la moyenne :
\begin{subequations}
	\begin{gather}
	\kappa\vect{b}_{i+1/2} = \kappa\vect{b}^{-}_{i+1/2} = \kappa\vect{b}^{+}_{i+1/2}
	= \frac{\kappa\vect{b}_{i} + \kappa\vect{b}_{i+1}}{2} \\
	\vect{H}^-_{i} = Q_i\kappa\vect{b}^{-}_{i+1/2} \\
	\vect{H}^+_{i} = Q_i\kappa\vect{b}^{+}_{i+1/2} \\
	\vect{H}^-_{i} = \vect{H}^+_{i} = \frac{\kappa\vect{b}_{i} + \kappa\vect{b}_{i+1}}{2}
	\end{gather}
\end{subequations}
Cette idée reste intéressante et élégante. Elle n'a de sens que pour une poutre à propriétés variables (sinon on a la continuité de la courbure également).
}

Shear Force  exercée par la poutre sur le premier point $\vect{x}_0$ :
\begin{equation}
	\begin{aligned}
	\vect{F}_0^{\perp}
	&= \left[\vect{d}_3 \times \mat{M}'+ Q\kappa\vect{b} \right]_{0} \\
	&= \frac{\vect{e}_0}{|\vect{e}_0|} \times \frac{\mat{M}_{1} - \mat{M}_{0}}{|\vect{e}_0|}
	+ Q_0\frac{\kappa\vect{b}_{0}}{2} \\
	& = \vect{F}^1_{0} + \vect{F}^2_{0}  +  \vect{H}^-_{0}  \\
	\end{aligned}
\end{equation}

Shear Force  exercée par la poutre sur le dernier point $\vect{x}_n$ :
\begin{equation}
	\begin{aligned}
	\vect{F}_n^{\perp}
	&= - \left[ \vect{d}_3 \times \mat{M}'+ Q\kappa\vect{b}\right]_{n} \\
	&= - \frac{\vect{e}_{n-1}}{|\vect{e}_{n-1}|} \times \frac{\mat{M}_{n} - \mat{M}_{n-1}}{|\vect{e}_{n-1}|}
	- Q_n\frac{\kappa\vect{b}_{n}}{2} \\
	& = -(\vect{F}^1_{n} + \vect{F}^2_{n}  +  \vect{H}^+_{n}) \\
	\end{aligned}
\end{equation}


\subsubsection{Moment of Torsion}

\begin{equation}
	\begin{aligned}
	\Gamma_i
		&= \int_{i-1/2}^{i-1/2} m(s)
		= Q_{i} - Q_{i-1}
		+ \int_{i-1/2}^{i+1/2} \kappa_{1} M_{2} - \kappa_{2} M_{1}\\
	\end{aligned}
\end{equation}

Note que le terme dans l'intégrale est nul pour une section isotrope. On retrouve les résultats bien connus sur les poutres axisymétriques à courbure au repos nulle.
Ici, 2 possibilitées :
\begin{itemize}
\item soit on considère que le moment varie lentement sur l'intervalle $]i-1/2,i+1/2[$ et on considère donc le terme $\kappa\vect{b}_i \times \vect{M}_i$ constant dans l'intégrale, ce qui donne en découpant l'intervalle en $[i-1/2,i]$ et  $[i,i+1/2]$ :
\begin{equation}
	\begin{aligned}
	\Gamma_i
		&= Q_{i} - Q_{i-1}
		+ \frac{\vect{e}_{i-1}}{|\vect{e}_{i-1}|} \cdot (\kappa\vect{b}_i \times \vect{M}_i ) \frac{|\vect{e}_{i-1}|}{2}
		+ \frac{\vect{e}_{i}}{|\vect{e}_{i}|} \cdot (\kappa\vect{b}_i \times \vect{M}_i ) \frac{|\vect{e}_{i}|}{2}\\
		&= Q_{i} - Q_{i-1}
		+ \frac{\vect{e}_{i-1} + \vect{e}_{i}}{2} \cdot (\kappa\vect{b}_i \times \vect{M}_i )		\end{aligned}
\end{equation}
\item soit on revient à l'hypothèse de continuité et de linéarité par morceaux sur le moment:
\begin{subequations}
	\begin{gather}
	\vect{M}(s) = \vect{M}_{i} + s \vect{M}'_{i} \quad \forall s \in [0,|\vect{e}_i|]
		\quad , \quad \vect{M}'_{i}= \frac{\vect{M}_{i+1} - \vect{M}_{i}}{|\vect{e}_i|} \\
	\kappa\vect{b}(s) = \mat{B}_i^{-1}\vect{M}(s) + \overbar{\kappa\vect{b}}(s) \\
	\end{gather}
\end{subequations}
d'où :
\begin{equation}
	\begin{aligned}
		&=\int_{i-1/2}^{i} \kappa_{1} M_{2} - \kappa_{2} M_{1} \\
		&= \int_{i-1/2}^{i} (\tfrac{1}{[EI]_{1,i}} - \tfrac{1}{[EI]_{2,i}})M_1M_2
	\end{aligned}
\end{equation}
\end{itemize}





With
\begin{subequations}
	\begin{gather}
	[EI]_i \\
	[GJ]_i \\
	\kappa\vect{b}_i = \frac{2 \vect{e}_{i-1} \times \vect{e}_{i}}{|\vect{e}_{i-1}| |\vect{e}_i| |\vect{e}_{i-1} + \vect{e}_{i}|} \\
	\tau_i		= \frac{\theta_{i+1} - \theta_i}{|\vect{e}_i|} \\
	\vect{N}_{i} = N_i\vect{d}_3  \quad where \quad N_i = \alpha_i \epsilon_i \quad \alpha_i = \frac{2 [ES]_i [ES]_{i+1}}{[ES]_i + [ES]_{i+1}}\\
	\vect{M}_{i} 	= [EI]_{1,i}(\kappa_{1,i} - \overbar{\kappa}_{1,i})\vect{d}_{1,i} + [EI]_{2,i}(\kappa_{2,i} - \overbar{\kappa}_{2,i})\vect{d}_{2,i} \\
	\vect{Q}_{i} = Q_i\vect{d}_3 \quad where \quad Q_i = \beta_i (\tau_i - \overbar{\tau_i}) \quad \beta_i = \frac{2 [GJ]_i [GJ]_{i+1}}{[GJ]_i + [GJ]_{i+1}}\\
	\vect{F}^{1}_{i} 	= - \frac{\vect{e}_i \times \mat{M}_i}{|\vect{e}_i|^2} \\
	\vect{F}^{2}_{i} 	= + \frac{\vect{e}_i \times \mat{M}_{i+1}}{|\vect{e}_i|^2} \\
	\vect{H}^-_{i} 	= Q_i\kappa\vect{b}^{-}_{i+1/2} \\
	\vect{H}^+_{i} 	= Q_i\kappa\vect{b}^{+}_{i+1/2} \\
	\end{gather}
\end{subequations}
\note{Ici il y a une ambiguité sur $M_i$ en fonction d'un éventuel changement de propriété méca $EI_1$ ou $EI_2$. En fait il faudrait préciser un moment à droite et un moment à gauche (ce qui correspond à la réalité). Le moment étant la courbure au noeud i pondérée par $2EI$ à droite ou à gauche

Toutes les discrétisations ne se valent pas ....

En fait, ça n'a pas vraiment de sens de définir $EI$ sur le segment. Le moment est continu. Donc il est uniquement défini par la donnée de la courbure en un noeud et du $EI$ associé à ce noeuds. Introduire un moment à gauche et un moment à droite est problématique.
Ce qui ce passe, pour une poutre isostatique simple qui change de $EI$ à mi-travée, c'est une discontinuité de courbure $y''$ alors que $y$ et $y'$ restent continues.
Donc il semble plus pertinent de définir $EI$ aux noeuds car notre modèle ne peut pas représenter des discontinuitées de courbure (ou alors il faut connecter des poutres entre-elles.

Pour la torsion, on peut s'en sortir également en définissant $\beta$ aux noeuds et en supposant $GJ$ constant entre $[i-1/2,i+1/2]$. Il y a donc (éventuellement) un saut de $\beta$ au milieu de chaque $\vect{e}_i$.

On ne peut pas faire autrement que supposer la torsion uniforme entre 2 noeuds, c'est à dire une variation linéaire (par morceaux) de $\theta$. Et il y a donc une discontinuité potentielle aux noeuds.

On écrit la continuité du champs de torsion entre les noeuds $1$ et $2$ malgré le saut de $\beta$ :

A mi travée
\begin{equation}
	\begin{aligned}
	Q_{12} = Q_{mid} =  Q_{1}^+ = Q_{2}^-
	&= [GJ]_1\frac{\theta_{mid} - \theta_{1}}{|\vect{e}|/2}
	= [GJ]_2 \frac{\theta_{2} - \theta_{mid}}{|\vect{e}|/2}
	\end{aligned}
\end{equation}
D'où :
\begin{equation}
	\theta_{mid} = \frac{[GJ]_1}{[GJ]_1 + [GJ]_2}\theta_1 + \frac{[GJ]_2}{[GJ]_1 + [GJ]_2}\theta_2
\end{equation}
On en déduit :
\begin{equation}
	Q_{12} = Q_{mid} =  Q_{1}^+ = Q_{2}^-
	= \frac{2 [GJ]_1 [GJ]_2}{[GJ]_1 + [GJ]_2} (\frac{\theta_{2} - \theta_{1}}{|\vect{e}|})
\end{equation}
Donc il faut plutôt définit $Q_i$ la torsion uniforme sur le segment $\vect{e}_i$ comme :
\begin{equation}
	Q_i = \beta_i (\tau_i - \overbar{\tau_i}) \quad where \quad \beta_i = \frac{2 [GJ]_i [GJ]_{i+1}}{[GJ]_i + [GJ]_{i+1}}
\end{equation}

On retrouve bien le cas d'une poutre de propriété constante lorsque $[GJ]_i= [GJ]_{i+1} = GJ$ alors $\beta_i = GJ$.


De manière identique, on résonne pour l'effort axial entre deux noeuds 1 et 2 auxquels sont associés des raideurs axiales $[ES]_1$ et $[ES_2]$. On cherche la raideur équivalente connaissant uniquement l'allongement de l'ensemble du segment :
\begin{subequations}
	\begin{gather}
	N_1 = [ES]_1 \cdot \epsilon_1 \\
	N_2 = [ES]_2 \cdot \epsilon_2 \\
	N =  [ES]_2 \cdot \frac{\epsilon_1 + \epsilon_2}{2}
	\end{gather}
\end{subequations}

Avec :
\begin{subequations}
	\begin{gather}
	\epsilon_1 = \frac{l_1 - l_0/2}{l_0/2} \\
	\epsilon_2 = \frac{l_2 - l_0/2}{l_0/2} \\
	\epsilon = \frac{l}{l_0} = \frac{\epsilon_1}{2} + \frac{\epsilon_2}{2}  \\
	\end{gather}
\end{subequations}

Thus,
 \begin{equation}
	N_i = N_1 = N_2 \quad \Rightarrow \quad  \alpha_i = \frac{2 [ES]_i [ES]_{i+1}}{[ES]_i + [ES]_{i+1}}
\end{equation}


 }


\bibliographystyle{alpha}
\bibliography{../bibliography}
