%\appendix
\chapter{Calculus of variations}

% --------------------------------------------------------------------------------------------
\section{Introduction}
% --------------------------------------------------------------------------------------------
In this appendix we drawback essential mathematical concepts for the calculus of variations.
Recall how the notion of energy, gradients are extended to function spaces.

\cite{Abraham2002}

% --------------------------------------------------------------------------------------------
\section{Spaces}
% --------------------------------------------------------------------------------------------

\subsection{Normed space}
% --------------------------------------------------------------------------------------------
A \emph{normed space} $V(\mathbb{K})$ is a vector space $V$ over the scalar field $\mathbb{K}$ with a norm $\|.\|$.

A \emph{norm} is a map $\| . \| : V \times V \longmapsto \mathbb{K}$ which satisfies :
\begin{align}
	&\forall x \in V, 							&& \|x\| = 0_\mathbb{K} \Rightarrow x = 0_V&&&\\
	&\forall x \in V, \forall \lambda \in \mathbb{K}, 	&& \|\lambda x\| = |\lambda| \,\|x\|&&&\\
	&\forall (x,y) \in V^2, 						&& \|x + y\| \leqslant \|x\| + \|y\|&&&
\end{align}

\subsection{Inner product space}
% --------------------------------------------------------------------------------------------
A \emph{inner product space} or \emph{pre-hilbert space} $E(\mathbb{K})$ is a vector space $E$ over the scalar field $\mathbb{K}$ with an inner product.

An \emph{inner product} is a map $\langle \,; \rangle : E \times E \longmapsto \mathbb{K}$ which is bilinear, symmetric, positive-definite :
\begin{align}
	&\forall (x,y,z) \in E^3, \forall (\lambda,\mu) \in \mathbb{K}^2, 	&& \langle \lambda x\ + \mu y \,; z\rangle  = \lambda \langle  x \,; z\rangle + \mu \langle  y\,; z\rangle&&&\\
	&												&& \langle x \,; \lambda y + \mu z \rangle  = \lambda \langle  x\,; y\rangle + \mu \langle  x \,; z\rangle&&&\notag\\
	&\forall (x,y) \in E^2, 									&& \langle x\,; y\rangle = \langle y\,; x\rangle &&&\\
	&\forall x \in E, 										&& \langle x\,; x\rangle \geqslant 0_\mathbb{K} &&&\\
	&\forall x \in E, 										&& \langle x\,; x\rangle = 0_\mathbb{K} \Rightarrow x = 0_E &&&
\end{align}
Moreover, an inner product naturally induces a norm on $E$ defined by :
\begin{align}
	\forall x \in E, \quad \|x\| = \sqrt{\langle x\,; x\rangle}
\end{align}
Thus, an inner product vector space is also naturally a normed vector space.

\subsection{Euclidean space}
% --------------------------------------------------------------------------------------------
An \emph{Euclidean space} $\mathcal{E}(\mathbb{R})$ is a finite-dimensional real vector space with an inner product.
Thus, distances and angles between vectors could be defined and measured regarding to the norm associated with the chosen inner product.

An Euclidean space is nothing but a finite-dimensional real pre-hilbert space.

\subsection{Banach space}
% --------------------------------------------------------------------------------------------

A \emph{Banach space} $\mathcal{B}(\mathbb{K})$ is a complete normed vector space, which means that it is a normed vector space in which every Cauchy sequence of $\mathcal{B}$ converges in $\mathcal{B}$ for the given norm.

Thus, a Banach space is a vector space with a metric that allows the computation of vector length and distance between vectors and is complete in the sense that a Cauchy sequence of vectors always converges to a well defined limit in that space.

\subsection{Hilbert space}
% --------------------------------------------------------------------------------------------

A \emph{Hilbert space} is an inner product vector space $\mathcal{H}(\mathbb{K})$ such that the natural norm induced by the inner product turns $\mathcal{H}$ into a complete metric space (i.e.\ every Cauchy sequence of $\mathcal{H}$ converges in $\mathcal{H}$).

The Hilbert space concept is a generalization of the Euclidean space concept.
In physics it's common to encounter Hilbert spaces as infinite-dimensional function spaces.

Hilbert spaces are Banach spaces, but the converse does not hold generally.

For example, $\mathcal{L}^2([a,b])$ is an infinite-dimensional Hilbert space with the canonical inner product $\langle f\,; g\rangle=\int_a^b fg$.

Note that $\mathcal{L}^2$ is the only Hilbert space among the $\mathcal{L}^p$ spaces.
% --------------------------------------------------------------------------------------------
\section{Derivative}
% --------------------------------------------------------------------------------------------

The well known notion of function derivative in $\mathbb{R}^\mathbb{R}$ can be extended to maps between Banach spaces.
This is useful in physics when formulating problems as variational problems, usually in terms of energy minimization. Indeed, energy is generally defined over a functional vector space and not simply over the real line.

In this case, the research of minimal values of a potential energy rests on the calculus of variations of the energy function compared to variations to other functions defining the problem (geometry, materials, boundary conditions, ...).

Mathematical concepts extended well-known notions of derivative, jacobian and hessian in Euclidean spaces (typically $\mathbb{R}^2$ or $\mathbb{R}^3$) for Banach functional spaces.

\subsection{Fréchet derivative - strong}
% --------------------------------------------------------------------------------------------
\subsubsection{Differentiability}
Let $\mathcal{B}_V$ and $\mathcal{B}_W$ be two Banach spaces and $U \subset \mathcal{B}_V$ an open subset of $\mathcal{B}_V$.
Let $f : u \longmapsto f(u)$ be a function of  $U^{\mathcal{B}_W}$.
$f$ is said to be \emph{Fréchet differentiable} at $u\in U$ if there exists a continious linear operator $Df_{u} \in 
\mathcal{L}(\mathcal{B}_V,{\mathcal{B}_W})$ such that :
\begin{align}
\lim_{\|h\| \to 0} \frac{\|f(u+h) - f(u)- Df_{u}(h)\|_{\mathcal{B}_W}}{\|h\|_{\mathcal{B}_V}} = 0
\end{align}
Or, equivalently :
\begin{align}
	f(u+h) = f(u) + Df_{u}(h) + o(\|h\|) \quad , \quad 
	\lim_{\|h\| \to 0} \frac{\|o(\|h\|)\|_{\mathcal{B}_W}}{\|h\|} = 0
\end{align}

Note that $df = Df_{u}(h)$ is called the differential of $f$ at point $u$ and represents the change in the function $f$ for a perturbation $h$ from $u$.

In the literature, it is common to found the following notation : $df = Df_u(h) = Df(u)h$, for the differential of $f$, which means nothing but $Df(u)$ is linear regarding $h$. This notation can be ambiguous as far as the linearity of $Df(u)$ in $h$ is denoted as a product which is not explicitly defined.

\subsubsection{Derivability}

If $f$ is Fréchet differentiable at $u \in U$, the continous linear operator $Df_{u} \in 
\mathcal{L}(\mathcal{B}_V,{\mathcal{B}_W})$ is called the Fréchet derivative of $f$ at $u$ and is also denoted :
\begin{align}
	f'(u) = Df(u) = Df_{u}
\end{align}
$f$ is said to be $\mathcal{C}^1$ in the sens of Fréchet if $f$ is Fréchet differentiable for all $u \in U$ and the function $Df : u \longmapsto f'(u)$ of $U^{\mathcal{L}(\mathcal{B}_V,{\mathcal{B}_W})}$ is continuous.
 
\subsubsection{Higer derivatives}
 
Because the differential of $f$ is a linear map from $\mathcal{B}_V$ to $\mathcal{L}(\mathcal{B}_V,{\mathcal{B}_W})$ it is possible to look for the differentiability of $Df$. If it exists, it is denoted $D^2f$ and maps $\mathcal{B}_V$ to $\mathcal{L}(\mathcal{B}_V,\mathcal{L}(\mathcal{B}_V,{\mathcal{B}_W}))$.
 
\subsection{Gâteaux derivative - weak}
% --------------------------------------------------------------------------------------------
\subsubsection{Differentiability}
Let $\mathcal{B}_V$ and $\mathcal{B}_W$ be two Banach spaces and $U \subset \mathcal{B}_V$ an open subset of $\mathcal{B}_V$.
Let $f : u \longmapsto f(u)$ be a function of  $U^{\mathcal{B}_W}$.
$f$ is said to be \emph{Gâteaux differentiable} at $u\in U$ if there exists a continious linear operator $Df_{u} \in 
\mathcal{L}(\mathcal{B}_V,{\mathcal{B}_W})$ such that :
\begin{align}
\forall h \in \mathcal{U}, \quad\lim_{\lambda \to 0} \frac{f(u+\lambda h) - f(u)}{\lambda} = \frac{d}{d\lambda}f(u+\lambda h)\Bigr|_{\lambda = 0} = Df_u(h)
\end{align}
Or, equivalently :
\begin{align}
	\forall h \in \mathcal{U}, \quad f(u+\lambda h) = f(u) + \lambda D_{f,u}(h) + o(|\lambda|)
	\quad , \quad \lim_{\lambda \to 0} \frac{\|o(|\lambda|)\|_{\mathcal{B}_W}}{\lambda} = 0
\end{align}

Note that $df = Df_{u}(h)$ is called the differential of $f$ at point $u$.

\subsubsection{Derivability}
If $f$ is Gâteaux differentiable at $u \in U$, the continous linear operator $Df_{u} \in 
\mathcal{L}(\mathcal{B}_V,{\mathcal{B}_W})$ is called the Gâteaux derivative of $f$ at $u$ and is also denoted :
\begin{align}
	f'(u) = Df(u) = Df_{u}
\end{align}
$f$ is said to be $\mathcal{C}^1$ in the sens of Gâteaux if $f$ is Gâteaux differentiable for all $u \in U$ and the function $Df : u \longmapsto f'(u)$ of $U^{\mathcal{L}(\mathcal{B}_V,{\mathcal{B}_W})}$ is continuous.
 
The Gâteaux derivative is a weaker form of derivative than the Fréchet derivative : if $f$ is Fréchet differentiable, then it is also Gâteaux differentiable, and its Fréchet and Gâteaux derivatives agree, but the converse does not hold generally.

\subsection{Useful properties}

Let $\mathcal{B}_V$, $\mathcal{B}_W$ and $\mathcal{B}_Z$ be three Banach spaces. 
Let $f,g : \mathcal{B}_V \longmapsto \mathcal{B}_W$ and $h : \mathcal{B}_W \longmapsto \mathcal{B}_Z$ be three Gâteaux differentiable functions. Then, the following useful properties holds :
\begin{align}
	&D(f+g)(u) = Df(u) + Dg(u)\\
	&D(h\circ f)(u) = Dh(f(u)) \circ Df(u)
\end{align}

\subsection{Partial derivative}

\section{Gradient}

The \emph{gradient vector} denotes the differential or total derivative of a differentiable function $F$ in the special case it is a map from $\mathbb{R}^n$ to $\mathbb{R}$. Thus, it satisfies the following relationships :
\begin{align}
	&\nabla F(x) = F'(x) = DF_x\\
	&DF_x(h) = \langle \nabla F(x)\,; h\rangle\\
	&F(x + h) = F(x) + \langle \nabla F(x)\,; h\rangle + o(\|h\|)
\end{align}
If the matrix notation is adopted, the previous relations leads to :
\begin{align}
	{F}({X}+{H}) = F(X) +  \nabla F(u)^T H + o(\|H\|)
	\quad , \quad \nabla F(x) = \left[\begin{array}{c}\frac{\partial F}{\partial x_1} \\ \vdots \\ \frac{\partial F}{\partial x_n}\end{array}\right] \in \mathbb(R)^n
\end{align}

\section{Jacobian}
Let $f$ be a differentiable function from $\mathbb{R}^n$ to $\mathbb{R}^p$. The differential of such a fonction is a linear application from $\mathbb{R}^n$ to $\mathbb{R}^p$ which could be represented with the following matrix called the \emph{jacobian matrix} :
\begin{align}
	&Df_{x} = \boldsymbol{J}_f(x) = \frac{df}{dx}(x) = 
	\left[\begin{array}{ccc}
		\frac{\partial f}{\partial x_1} \,,& \cdots \;, & \frac{\partial f}{\partial x_n}
	\end{array}\right] =
	\left[\begin{array}{ccc}
		\frac{\partial f_1}{\partial x_1} & \cdots & \frac{\partial f_1}{\partial x_n} \\
		\vdots & \ddots & \vdots \\
		\frac{\partial f_p}{\partial x_1} & \cdots & \frac{\partial f_p}{\partial x_n}
	\end{array}\right]\in\mathcal{M}_{p,n}(\mathbb{R})
\end{align}
Thus, with the matrix notation, the Taylor expansion takes the following form :
\begin{align}
	&F(X+H) = F(X) +  \boldsymbol{J}_f^XH + o(\|H\|)
\end{align}
In the cas $p=1$, the jacobian matrix of the functional $F$ is nothing but the gradient transpose itself :
\begin{align}
	DF_{x} = \boldsymbol{J}_F(x) = \frac{dF}{dx} = 
	\left[\begin{array}{ccc}
		\frac{\partial F}{\partial x_1} \,,& \cdots \;, &\frac{\partial F}{\partial x_n}
	\end{array}\right] = \nabla F^T
\end{align}

\section{Hessian}
Let $F$ be a differentiable function from $\mathbb{R}^n$ to $\mathbb{R}$. The second order differential of such a fonction is a linear application from $\mathbb{R}^n$ to $\mathbb{R}^n$ which could be represented with the following matrix called the \emph{hessian matrix} :
\begin{align}
	&D^2F_{x} = \boldsymbol{H}_F(x) = \frac{d^2F}{dx}(x) = 
	\left[\begin{array}{cccc}
		\frac{\partial F^2_1}{\partial x_1^2} & \frac{\partial F^2_1}{\partial x_1\partial x_2} &\cdots & \frac{\partial F^2_1}{\partial x_1\partial x_n} \\
		\frac{\partial F^2_1}{\partial x_2\partial x_1} & \frac{\partial F^2_1}{\partial x_2^2} &\cdots & \frac{\partial F^2_1}{\partial x_2\partial x_n} \\
		\vdots & &\ddots & \vdots \\
		\frac{\partial F^2_p}{\partial x_n\partial x_1} & \frac{\partial F^2_p}{\partial x_n\partial x_2}&\cdots & \frac{\partial F^2_p}{\partial x_n^2}
	\end{array}\right]\in\mathcal{M}_{n,n}(\mathbb{R})
\end{align}
Thus, with the matrix notation, the Taylor expansion takes the following form :
\begin{align}
	F(X+H) = F(X) +  \boldsymbol{J}_F^XH + \tfrac{1}{2}H^T\boldsymbol{H}_F^XH + o(\|H\|)
\end{align}


\section{Functional}
% --------------------------------------------------------------------------------------------
A \emph{functional} is a map from a vector space $E(\mathbb{K})$ into its underlying scalar field $\mathbb{K}$. Here $\mathcal{E}_p[\boldsymbol{x},\theta]$ is a functional depending over $\boldsymbol{x}$ and $\theta$.




\bibliographystyle{alpha}
\bibliography{../bibliography}