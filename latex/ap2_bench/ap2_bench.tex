%\appendix
\chapter{Bench for HPC}

\section{Introduction}
% ====================

In this section aims at providing basic but reliable guidlines to produce fast and mannagable code
for our algorithms

\cite{Abraham2002}

\section{Languages}
% =================
- Csharp
- Julia
- C++
- Intel MKL
- OpenBLAS

\section{From syntax to processor}
% ================================

A short story about how a code is translated to get machin instructions

\section{Benchmark}
% ==================


\begin{figure}
  \centering
  \begin{tikzpicture}
    \pgfplotstableread[col sep=comma]{ap2_bench/plot/inputdata.txt}\mydata;
    \begin{axis}[
      xbar, xmin=0, enlarge x limits=0.05,
      xmajorgrids,
      width=15cm,
      height=20cm,
      enlarge y limits=0.06,
      % xlabel={compilation time / seconds},
      area style,
      ytick=data,
      yticklabels from table={\mydata}{cat} % Get tables from second column of data table
      ,bar width=0.1
      ,legend cell align=left,
      ,legend style={area legend, draw=white, legend columns=1}
      % ,xtick=\empty, ytick=\empty,
      ,separate axis lines
      ,x axis line style=white
      ,y axis line style=white
      ,every x tick/.style={color=white, very thin}
      ,every y tick/.style={color=white, very thin}
    ]
    \addplot [style={white,fill=Tblue,mark=none}]
              table [x expr=\thisrowno{5}, y expr=-\coordindex] {\mydata};

    \addplot [style={white,fill=Tred,mark=none}]
              table [x expr=\thisrowno{4}, y expr=-\coordindex] {\mydata};

    \addplot [style={white,fill=Tgray,mark=none}]
              table [x expr=\thisrowno{3}, y expr=-\coordindex] {\mydata};

    \addplot [style={white,fill=Tgray,mark=none}]
              table [x expr=\thisrowno{2}, y expr=-\coordindex] {\mydata};

    \addplot [style={white,fill=Tgray,mark=none}]
              table [x expr=\thisrowno{1}, y expr=-\coordindex] {\mydata};

    \legend{Intel MKL, OpenBLAS, loop NBC, naive loop, native}
    \end{axis}
  \end{tikzpicture}
\caption{Each operator is evaluated on a vector of Float64 of size $n=10^6$ for about 10s.
Results are given relatively to MKL performance (MKL = 1). }
\end{figure}

\bibliographystyle{alpha}
\bibliography{../bibliography}
