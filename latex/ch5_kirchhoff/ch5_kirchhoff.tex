\chapter{Elastic rod : a novel element from kirchhoff equations}
\lipsum[1]

\section{Introduction}
Dans ce chapitre, après un bref rappel sur le cadre mathématique d'étude des courbes paramétrique de l'espace, on présente les notions de courbures et de torsion géométrique associées au repère de fraient. On montre ensuite le cas plus général d'un repère mobile quelconque attaché à une courbe gamma. On définit enfin la particularité d'un repère mobile adapté à un courbe, et on présente, en sus du repère de Frenet, une approche différente pour accrocher des repères le long d'une courbe (Bishop / RMF / Zéro-twisting frame)

% --------------------------------------------------------------------------------------------------------------------------------------------
% SMOOTH SPACE CURVE
% --------------------------------------------------------------------------------------------------------------------------------------------


\bibliographystyle{alpha}
\bibliography{../bibliography}