\chapter{Elastic rod : variational approch}

% #######################################################
\section{Introduction}
% #######################################################

\lipsum[1]

Basile \cite{Bergou2010}

Basile \cite{Bergou2008}
Je
Basile \cite{Audoly2000}

Sina \cite{Nabei2014}

% #######################################################
\section{Kinematic}
% #######################################################

Bishop

 $\boldsymbol{x}$ only depends on arc-length along the curve : $\boldsymbol{x}(t)$

Implicit dependence of $\theta$ in $\boldsymbol{x}$ : $\theta(t) \equiv \theta[x](t)$

Implicit dependence of $\boldsymbol{\omega}$ in $\boldsymbol{x}$ and $\theta$ : $\boldsymbol{\omega}(t) \equiv \boldsymbol{\omega}[\boldsymbol{x},\theta](t)$

We denote $\tau$ the twist along the curve : $\tau(t) = \frac{\partial \theta}{\partial t} = \theta'(t)$

We will make the following mathematical assumptions :
\begin{align}
	&\fonction{\boldsymbol{x}}{[0,L]}{\mathbb{R}^3}{t}{\boldsymbol{x}(t)}
	&&\boldsymbol{x} \in \mathcal{C}^{\infty}([0,L]^{\mathbb{R}^3})
	\\ \notag\\
	&\fonction{\theta[\boldsymbol{x}]}{[0,L]}{\mathbb{R}}{t}{\theta[\boldsymbol{x}](t)}
	&&\theta[\boldsymbol{x}] \in \mathcal{C}^{\infty}([0,L]^{\mathbb{R}})
	\\ \notag\\
	&\fonction{\boldsymbol{\omega}[\boldsymbol{x},\theta]}{[0,L]}{\mathbb{R}^2}{t}{\boldsymbol{\omega}[\boldsymbol{x},\theta](t)}
	&&\boldsymbol{\omega}[\boldsymbol{x},\theta] \in \mathcal{C}^{\infty}([0,L]^{\mathbb{R}^2})
\end{align}

The dependence in the arc-length is denoted by the parameter $t \in [0,L]$ by the use of parenthesis $(.)$.
The dependencies in functions are denoted by the use of brackets $[.]$.

When computings energy gradients, it comes to differentiate energies regarding there dependencies in functions $\boldsymbol{x}$ and $\theta$. Thus, we may recall that :
\begin{align}
	&\fonction{\theta}{\mathcal{C}^{\infty}([0,L]^{\mathbb{R}^3})}{\mathcal{C}^{\infty}([0,L]^\mathbb{R})}{\boldsymbol{x}}{\theta[\boldsymbol{x}]}
	&&\theta \in \mathcal{C}^{\infty}
	\\ \notag\\
	&\fonction{\boldsymbol{\omega}}{\mathcal{C}^{\infty}([0,L]^{\mathbb{R}^3}) \times \mathcal{C}^{\infty}([0,L]^{\mathbb{R}})}{\mathcal{C}^{\infty}([0,L]^{\mathbb{R}^2})}{(\boldsymbol{x},\theta)}{\boldsymbol{\omega}[\boldsymbol{x},\theta]}
	&&\boldsymbol{\omega} \in \mathcal{C}^{\infty}
\end{align}





% #######################################################
\section{Energy}
% #######################################################

\begin{align}
\mathcal{E}_p[\boldsymbol{x},\theta] & = \mathcal{E}_{stretch}[\boldsymbol{x}] + \mathcal{E}_{bend}[\boldsymbol{x},\theta] + \mathcal{E}_{twist}[\theta]
\end{align}
\begin{align}
\mathcal{E}_{stretch}[\boldsymbol{x}] & = \tfrac{1}{2} \int_{0}^{L} K_s(\tfrac{\|\boldsymbol{x}'\|}{\|\bar{\boldsymbol{x}}'\|}-1)^2dt \\
\mathcal{E}_{bend}[\boldsymbol{x},\theta] & = \tfrac{1}{2} \int_{0}^{L} (\boldsymbol\omega -\boldsymbol{\bar{\omega}})^T B \, (\boldsymbol\omega -\boldsymbol{\bar{\omega}})dt \\
\mathcal{E}_{twist}[\theta] & = \tfrac{1}{2} \int_{0}^{L} \beta(\theta' -{\bar{\theta'}})^2dt
\end{align}

Energies could be seen as functional, i.e.\ functions that map vectors from a functional vector space to there underlying scalar field $\mathbb{R}$.

\begin{align}
	&\fonction{\mathcal{E}_{stretch}}{\mathcal{C}^{\infty}([0,L]^{\mathbb{R}^3})}{\mathbb{R}}{\boldsymbol{x}}{\mathcal{E}_{stretch}[\boldsymbol{x}]}
	\\ \notag\\
	&\fonction{\mathcal{E}_{bend}}{\mathcal{C}^{\infty}([0,L]^{\mathbb{R}^3}) \times \mathcal{C}^\infty([0,L]^{\mathbb{R}^3},[0,L]^{\mathbb{R}})}{\mathbb{R}}{(\boldsymbol{x},\theta)}{\mathcal{E}_{bend}[\boldsymbol{x},\theta]}
	\\ \notag\\
	&\fonction{\mathcal{E}_{twist}}{\mathcal{C}^\infty([0,L]^{\mathbb{R}^3},[0,L]^{\mathbb{R}})}{\mathbb{R}}{\theta}{\mathcal{E}_{twist}[\theta]}
\end{align}

% #######################################################
\section{Inextensibility}
% #######################################################





% #######################################################
\section{Gradients}
% #######################################################


% --------------------------------------------------------------------------------------------
\subsection{Momentum}
% --------------------------------------------------------------------------------------------

Calcul du moment :
\begin{align}
	\mathcal{M} & = -\frac{d\mathcal{E}_p}{d\theta} =  - \frac{\partial \mathcal{E}_p}{\partial \theta} = - \frac{\partial \mathcal{E}_{bend}}{\partial \theta} - \frac{\partial \mathcal{E}_{twist}}{\partial \theta} \\[0.5em]
\end{align}


\subsubsection{Calcul de $\frac{\partial \mathcal{E}_{twist}}{\partial \theta}$}
% --------------------------------------------------------------------------------------------

\begin{align}
	\mathcal{E}_{twist}[\theta + \lambda h_{\theta}] & = \tfrac{1}{2} \int_{0}^{L} \beta\big((\theta + \lambda h_{\theta})' -{\bar{\theta'}}\big)^2 \,dt \\ 
%   	& = \tfrac{1}{2} \int_{0}^{L} \beta( (\theta'-{\bar{\theta'}})^2 + 2 \lambda (\theta'-{\bar{\theta'}}) h_{\theta}' + \lambda^2 h_{\theta}'^2) \,dt \notag\\
   	& = \mathcal{E}_{twist}[\theta] + \lambda\int_{0}^{L} \beta(\theta'-{\bar{\theta'}}) h_{\theta}' dt +   \lambda^2 \int_{0}^{L} \frac{\beta h_{\theta}'^2}{2} \,dt \notag\\
	& = \mathcal{E}_{twist}[\theta] + \lambda\left[\beta(\theta'-{\bar{\theta'}}) h_{\theta}\right]_0^L - \lambda\int_{0}^{L} \big(\beta(\theta'-{\bar{\theta'}})\big)' h_{\theta}  \,dt + o(\lambda) \notag\\
	& = \mathcal{E}_{twist}[\theta] + \lambda\int_{0}^{L} \big[\beta(\theta'-{\bar{\theta'}})(\delta_L-\delta_0) - \big(\beta(\theta'-{\bar{\theta'}})\big)'\big] h_{\theta}  \,dt + o(\lambda) \notag\\
	& = \mathcal{E}_{twist}[\theta] + \lambda(D_{\mathcal{E}_{twist}} \cdot h_{\theta}) + o(\lambda) \notag
\end{align}

With :
\begin{align}
\frac{\partial \mathcal{E}_{twist}}{\partial \theta} \equiv D_{\mathcal{E}_t} = - \big(\beta(\theta'-{\bar{\theta'}})\big)' + \beta(\theta'-{\bar{\theta'}})(\delta_L-\delta_0) \quad , \quad \frac{\partial \mathcal{E}_{twist}}{\partial \theta} : [0;L] \longrightarrow 	\mathbb{R}
\end{align}



\subsubsection{Calcul de $\frac{\partial \boldsymbol{\omega}}{\partial \theta}$}
% --------------------------------------------------------------------------------------------

On montre par des considérations géométriques que :
\begin{align}
	\boldsymbol{d_1}[\boldsymbol{x}, \theta + \lambda h_{\theta}] & = \boldsymbol{d_1}[\boldsymbol{x}, \theta] + \sin(\lambda h_{\theta})\boldsymbol{d_2}[\boldsymbol{x}, \theta] - (1-\cos(\lambda h_{\theta})\boldsymbol{d_1}[\boldsymbol{x}, \theta] \\
   	& = \boldsymbol{d_1}[\boldsymbol{x}, \theta] + \lambda h_{\theta}\boldsymbol{d_2}[\boldsymbol{x}, \theta] + o(\lambda) \notag
\end{align}
\begin{align}
	\boldsymbol{d_2}[\boldsymbol{x}, \theta + \lambda h_{\theta}] & = \boldsymbol{d_2}[\boldsymbol{x}, \theta] - \sin(\lambda h_{\theta})\boldsymbol{d_1}[\boldsymbol{x}, \theta] - (1-\cos(\lambda h_{\theta})\boldsymbol{d_2}[\boldsymbol{x}, \theta] \\
   	& = \boldsymbol{d_2}[\boldsymbol{x}, \theta] - \lambda h_{\theta}\boldsymbol{d_1}[\boldsymbol{x}, \theta] + o(\lambda) \notag
\end{align}

On en déduit pour le vecteur courbure matérielle :
\begin{align}
	\boldsymbol{\omega}[\boldsymbol{x}, \theta + \lambda h_{\theta}] & = 
	\left[\begin{array}{c}
	\boldsymbol{x}'' \cdot  \boldsymbol{d_1}[\boldsymbol{x}, \theta + \lambda h_{\theta}]\\
	\boldsymbol{x}'' \cdot \boldsymbol{d_2}[\boldsymbol{x}, \theta + \lambda h_{\theta}]\\
	\end{array}\right] \\
	& = \boldsymbol{\omega}[\boldsymbol{x}, \theta] + \lambda 
	\left[\begin{array}{c}
	\boldsymbol{x}'' \cdot  \boldsymbol{d_2}[\boldsymbol{x}, \theta]\\
	- \boldsymbol{x}'' \cdot \boldsymbol{d_1}[\boldsymbol{x}, \theta]\\
	\end{array}\right] \cdot h_{\theta}
	 + o(\lambda)\notag\\
	 & = \boldsymbol{\omega}[\boldsymbol{x}, \theta] + \lambda(D_{\boldsymbol{\omega}} \cdot h_{\theta}) + o(\lambda)\notag
\end{align}

With :
\begin{align}
	&\frac{\partial \boldsymbol{\omega}}{\partial \theta} \equiv D_{\boldsymbol{\omega}} = - R_{\pi/2}\:\boldsymbol{\omega}[\boldsymbol{x}, \theta]
	\quad , \quad R_{\pi/2} = \left[\begin{array}{cc}0 & -1 \\1 & 0\end{array}\right]
	\quad , \quad \frac{\partial \boldsymbol{\omega}}{\partial \theta}  : [0;L] \longrightarrow \mathbb{R}^2
\end{align}


\subsubsection{Calcul de $\frac{\partial \mathcal{E}_{bend}}{\partial \theta}$}
% --------------------------------------------------------------------------------------------

\begin{align}
	\mathcal{E}_{bend}[\boldsymbol{x},\theta + \lambda h_{\theta}]
		& = \tfrac{1}{2} \int_{0}^{L} \big(\boldsymbol{\omega}[\boldsymbol{x}, \theta + \lambda h_{\theta}] - \bar{\boldsymbol{\omega}}[\boldsymbol{x}, \theta]\big)^T B \,  \big(\boldsymbol{\omega}[\boldsymbol{x}, \theta + \lambda h_{\theta}] - \bar{\boldsymbol{\omega}}[\boldsymbol{x}, \theta]\big)dt \notag\\
%	& = \tfrac{1}{2} \int_{0}^{L} \big((\boldsymbol{\omega} - \bar{\boldsymbol{\omega}}) - \lambda R_{\pi/2}\:\boldsymbol{\omega} \cdot h_{\theta} + o(\lambda)\big)^T B \,  \big((\boldsymbol{\omega} - \bar{\boldsymbol{\omega}}) - \lambda R_{\pi/2}\:\boldsymbol{\omega} \cdot h_{\theta} + o(\lambda)\big)dt \notag\\
		& = \mathcal{E}_{bend}[\boldsymbol{x},\theta] - \lambda \int_{0}^{L} \big((\boldsymbol{\omega} - \bar{\boldsymbol{\omega}})^T B R_{\pi/2}\:\boldsymbol{\omega}\big)\cdot h_{\theta} dt + o(\lambda) \notag\\
		& =  \mathcal{E}_{bend}[\boldsymbol{x},\theta] + \lambda(D_{\mathcal{E}_{bend}} \cdot h_{\theta}) + o(\lambda)
\end{align}

With :
\begin{align}
	\frac{\partial \mathcal{E}_{bend}}{\partial \theta} \equiv D_{\mathcal{E}_{bend}} = -(\boldsymbol{\omega} - \bar{\boldsymbol{\omega}})^T B R_{\pi/2}\:\boldsymbol{\omega}
	\quad , \quad \frac{\partial \mathcal{E}_{bend}}{\partial \theta} : [0;L] \longrightarrow \mathbb{R}
\end{align}


\subsubsection{Calcul de $\mathcal{M}$}
% --------------------------------------------------------------------------------------------

Finally:
\begin{align}
\mathcal{M} 	& = -\frac{d\mathcal{E}_p}{d\theta} = -(\boldsymbol{\omega} - \bar{\boldsymbol{\omega}})^T B R_{\pi/2}\:\boldsymbol{\omega} - \big(\beta(\theta'-{\bar{\theta'}})\big)' + \beta(\theta'-{\bar{\theta'}})(\delta_L-\delta_0)
\end{align}

ATTENTION : écrire ici que M est porté par $d_3$

% --------------------------------------------------------------------------------------------
\subsection{Forces}
% --------------------------------------------------------------------------------------------

Calcul des efforts:
\begin{align}
\boldsymbol{\mathcal{F}} & = -\frac{d\mathcal{E}_p}{d\boldsymbol{x}} =  - \frac{\partial \mathcal{E}_p}{\partial \boldsymbol{x}} - \int_{0}^{L} \tfrac{\partial \mathcal{E}_p}{\partial \theta} \tfrac{\partial \theta}{\partial \boldsymbol{x}}
\end{align}


\subsubsection{Calcul de $\frac{\partial \theta}{\partial \boldsymbol{x}}$ par le writh}
% --------------------------------------------------------------------------------------------

\cite{Fuller1978}, \cite{deVries2005},  \cite{Vauquelin2000}, \cite{Berger2009}

On montre ici qu'en un point d'abscisse s, une variation de la centerline de $\lambda \boldsymbol{h_x}$ entraine une variation de $\theta$ intégrée sur la courbe $\Gamma$ de $(\frac{\partial \theta[\boldsymbol{x}](s)}{\partial\boldsymbol{x}} \cdot \lambda \boldsymbol{h_x})$.

$H : t \mapsto \left\{\begin{array}{c}0  , \quad t<0 \\1  , \quad t\geqslant0\end{array}\right.$ est la fonction de Heaviside.

$\delta_{t_0} : t \mapsto \delta(t-t_0)$ est la distribution de dirac centrée en $t_0$.

ATTENTION : ici il y a qqch à expliquer entre $\theta$ et $\psi$ (il y a une question de signe à détailler).

\begin{align}
	\Delta\psi_{\lambda\boldsymbol{h_x}}[\boldsymbol{x}](s) = \psi[\boldsymbol{x} + \lambda\boldsymbol{h_x}](s) - \psi[\boldsymbol{x}](s)
\end{align}

\begin{align}
	\Delta\psi_{\lambda\boldsymbol{h_x}}[\boldsymbol{x}](s) & = \int_{0}^{s} \frac{\boldsymbol{x'} \times (\boldsymbol{x} + \lambda\boldsymbol{h_x})'}{1+ \boldsymbol{x'} \cdot  (\boldsymbol{x} + \lambda\boldsymbol{h_x})'} \cdot \big(\boldsymbol{x''} + (\boldsymbol{x} + \lambda\boldsymbol{h_x})''\big) dt \\
	& = \int_{0}^{s} \frac{\lambda \boldsymbol{x'} \times \boldsymbol{h_x}'}{2(1+ \frac{\lambda \boldsymbol{x'} \cdot \boldsymbol{h_x}'}{2})} \cdot (2\boldsymbol{x''} + \lambda\boldsymbol{h_x}'') dt \notag\\
	& = \int_{0}^{s} \frac{\lambda}{2}(\boldsymbol{x'} \times \boldsymbol{h_x}')\big(1- \frac{\lambda \boldsymbol{x'} \cdot \boldsymbol{h_x}'}{2} + o(\lambda)\big)(2\boldsymbol{x''} + \lambda\boldsymbol{h_x}'') dt \notag\\
	& = \int_{0}^{s} \big(-\boldsymbol{k_b} \cdot \lambda \boldsymbol{h_x'} + \frac{\lambda^2}{2}(\boldsymbol{x'} \times \boldsymbol{h_x}')\cdot\boldsymbol{h_x''}\big)\big(1- \frac{\lambda \boldsymbol{x'} \cdot \boldsymbol{h_x}'}{2} + o(\lambda)\big) dt \notag\\
	& = - \int_{0}^{s} -\boldsymbol{k_b} \cdot \lambda \boldsymbol{h_x'}dt +o(\lambda) \notag\\
	& = \big[-\boldsymbol{k_b} \cdot \lambda \boldsymbol{h_x}\big]_{0}^{s} + \int_{0}^{s} \boldsymbol{k_b'} \cdot \lambda \boldsymbol{h_x}dt +o(\lambda) \notag\\
	& = \int_{0}^{L} \big((1-H)\boldsymbol{k_b'} - (\delta_s - \delta_0)\boldsymbol{k_b}\big)\cdot\lambda \boldsymbol{h_x}dt +o(\lambda) \notag\\
	& = \lambda(\boldsymbol{D_{\psi}}(s)\cdot\boldsymbol{h_x}) + o(\lambda)\notag
\end{align}

With :
\begin{align}
	\frac{\partial \theta}{\partial \boldsymbol{x}}(s) \equiv -\boldsymbol{D_{\psi}}(s) = (\delta_s - \delta_0)\boldsymbol{k_b} - (1-H)\boldsymbol{k_b'}
	\quad , \quad \frac{\partial \theta}{\partial \boldsymbol{x}}(s) : [0;L] \longrightarrow \mathbb{R}^3
\end{align}



\subsubsection{Calcul de $\frac{\partial \boldsymbol{\omega}}{\partial \boldsymbol{x}}$}
% --------------------------------------------------------------------------------------------

Quand $\boldsymbol{x}$ varie de $\lambda\boldsymbol{h_x}$, le repère materiel est tourné de sorte que, en décomposant sur le repère de base :
\begin{align}
	\boldsymbol{d_1}[\boldsymbol{x}+\lambda\boldsymbol{h_x}, \theta] & = \boldsymbol{d_1}[\boldsymbol{x}, \theta] + \Delta\psi_{\lambda\boldsymbol{h_x}}[\boldsymbol{x}]\boldsymbol{d_2}[\boldsymbol{x}, \theta] + \Delta\phi_{\lambda\boldsymbol{h_x}}[\boldsymbol{x}]\boldsymbol{d_3}[\boldsymbol{x}, \theta] + o(\lambda)\\
	\boldsymbol{d_2}[\boldsymbol{x}+\lambda\boldsymbol{h_x}, \theta] & = \boldsymbol{d_2}[\boldsymbol{x}, \theta] - \Delta\psi_{\lambda\boldsymbol{h_x}}[\boldsymbol{x}]\boldsymbol{d_1}[\boldsymbol{x}, \theta] - \Delta\phi_{\lambda\boldsymbol{h_x}}[\boldsymbol{x}]\boldsymbol{d_3}[\boldsymbol{x}, \theta] + o(\lambda)
\end{align}
On montre aisément que la contribution en $\Delta\psi$ autour de $\boldsymbol{d_1}$ et $\boldsymbol{d_2}$ est d'un ordre supérieur à celle autour de $\boldsymbol{d_3} = \boldsymbol{x}'$ (ce qui se comprend bien dans le cas d'une variation en hélice).

Par propriété d'orthogonalité des vecteurs du repère mobile (cf § sur le curve framing) on montre que :
\begin{align}
	\boldsymbol{d_1}\cdot\boldsymbol{d_3} = 0 &\Rightarrow \boldsymbol{d_1}[\boldsymbol{x}+\lambda\boldsymbol{h_x}, \theta] \cdot (\boldsymbol{x}+\lambda\boldsymbol{h_x})' = 0\\
	&\Rightarrow \Delta\phi_{\lambda\boldsymbol{h_x}}[\boldsymbol{x}] = \boldsymbol{d_1}[\boldsymbol{x}+\lambda\boldsymbol{h_x}, \theta] \cdot \boldsymbol{x}' = \lambda(-\boldsymbol{d_1}[\boldsymbol{x}, \theta] \cdot \boldsymbol{h_x}') + o(\lambda) \notag
\end{align}
\begin{align}
	\boldsymbol{d_2}\cdot\boldsymbol{d_3} = 0 &\Rightarrow \boldsymbol{d_1}[\boldsymbol{x}+\lambda\boldsymbol{h_x}, \theta] \cdot (\boldsymbol{x}+\lambda\boldsymbol{h_x})' = 0\\
	&\Rightarrow \Delta\phi_{\lambda\boldsymbol{h_x}}[\boldsymbol{x}] = \boldsymbol{d_2}[\boldsymbol{x}+\lambda\boldsymbol{h_x}, \theta] \cdot \boldsymbol{x}' = \lambda(-\boldsymbol{d_2}[\boldsymbol{x}, \theta] \cdot \boldsymbol{h_x}') + o(\lambda) \notag
\end{align}

On en déduit le calcul de la variation de $\boldsymbol{\omega}$ pour une vairation de la centerline de $\lambda \boldsymbol{h_x}$ qui vaut :
\begin{align}
	\boldsymbol{\omega}[\boldsymbol{x}+\lambda \boldsymbol{h_x},\theta]
	& =
	\left[\begin{array}{c}
	(\boldsymbol{x}+\lambda \boldsymbol{h_x})'' \cdot  \boldsymbol{d_1}[\boldsymbol{x}+\lambda \boldsymbol{h_x},\theta]\\
	(\boldsymbol{x}+\lambda \boldsymbol{h_x})'' \cdot \boldsymbol{d_2}[\boldsymbol{x}+\lambda \boldsymbol{h_x},\theta]\\
	\end{array}\right] \notag\\
	& = 
	\left[\begin{array}{c}
	(\boldsymbol{x}+\lambda \boldsymbol{h_x})'' \cdot ( 	\boldsymbol{d_1}[\boldsymbol{x}] 
												+ (-\lambda\frac{\partial \theta}{\partial \boldsymbol{x}}(s) \cdot \boldsymbol{h_x} + o(\lambda))\boldsymbol{d_2}[\boldsymbol{x}]
												- (\lambda \boldsymbol{d_1[\boldsymbol{x}]}\cdot\boldsymbol{h_x} + o(\lambda)) \cdot \boldsymbol{x}''
											)\\
	(\boldsymbol{x}+\lambda \boldsymbol{h_x})'' \cdot ( 	\boldsymbol{d_2}[\boldsymbol{x}] 
												- (-\lambda\frac{\partial \theta}{\partial \boldsymbol{x}}(s) \cdot \boldsymbol{h_x} + o(\lambda))\boldsymbol{d_1}[\boldsymbol{x}]
												- (\lambda \boldsymbol{d_2[\boldsymbol{x}]}\cdot\boldsymbol{h_x} + o(\lambda)) \cdot \boldsymbol{x}''
											)\\
	\end{array}\right] \notag\\
	  \begin{split}
      		& = \boldsymbol{\omega}[\boldsymbol{x},\theta] +
        			\lambda
        			\left[\begin{array}{c}
        			\boldsymbol{d_1}[\boldsymbol{x},\theta]^T \\
        			\boldsymbol{d_2}[\boldsymbol{x},\theta]^T
        			\end{array}\right] \cdot \boldsymbol{h_x}'' 
        			 -
        			\big(\lambda\frac{\partial \theta}{\partial \boldsymbol{x}}(s) \cdot \boldsymbol{h_x}\big)
        			\left[\begin{array}{c}
        			\boldsymbol{d_2}[\boldsymbol{x},\theta]\cdot \boldsymbol{x}''\\
        			-\boldsymbol{d_1}[\boldsymbol{x},\theta]\cdot \boldsymbol{x}''\\
        			\end{array}\right] \\
        		& \hspace{5cm}+ \lambda(\boldsymbol{x}'' \cdot \boldsymbol{x}')
        		\left[\begin{array}{c}
        		\boldsymbol{d_1}[\boldsymbol{x},\theta]\cdot \boldsymbol{h_x}'\\
        		\boldsymbol{d_2}[\boldsymbol{x},\theta]\cdot \boldsymbol{h_x}''\\
        		\end{array}\right]
        		 + o(\lambda)
 	 \end{split} \notag
\end{align}


\begin{align}
	\boldsymbol{\omega}[\boldsymbol{x}+\lambda \boldsymbol{h_x},\theta] & = 
	\left[\begin{array}{c}
		(\boldsymbol{x}+\lambda \boldsymbol{h_x})'' \cdot  \boldsymbol{d_1}[\boldsymbol{x}+\lambda \boldsymbol{h_x},\theta]\\
		(\boldsymbol{x}+\lambda \boldsymbol{h_x})'' \cdot \boldsymbol{d_2}[\boldsymbol{x}+\lambda \boldsymbol{h_x},\theta]\\
	\end{array}\right] && POUET &&& POUET\notag\\
& = 
\left[\begin{array}{c}
	(\boldsymbol{x}+\lambda \boldsymbol{h_x})'' \cdot ( 	\boldsymbol{d_1}[\boldsymbol{x}] 
												+ (-\lambda\frac{\partial \theta}{\partial \boldsymbol{x}}(s) \cdot \boldsymbol{h_x} + o(\lambda))\boldsymbol{d_2}[\boldsymbol{x}]
											)\\
	(\boldsymbol{x}+\lambda \boldsymbol{h_x})'' \cdot ( 	\boldsymbol{d_2}[\boldsymbol{x}] 
												- (-\lambda\frac{\partial \theta}{\partial \boldsymbol{x}}(s) \cdot \boldsymbol{h_x} + o(\lambda))\boldsymbol{d_1}[\boldsymbol{x}]
											)\\
	\end{array}\right] && POUET &&& POUET\notag\\
& \quad- 
\left[\begin{array}{c}
												- (\lambda \boldsymbol{d_1[\boldsymbol{x}]}\cdot\boldsymbol{h_x} + o(\lambda)) \cdot \boldsymbol{x}''
											)\\
												- (\lambda \boldsymbol{d_2[\boldsymbol{x}]}\cdot\boldsymbol{h_x} + o(\lambda)) \cdot \boldsymbol{x}''
											)\\
	\end{array}\right] && POUET &&& POUET\notag\\
\end{align}



\begin{align}
  g + h & = i \\
  \begin{split}
      a & = b + c - d \\
        & \quad + e - f
  \end{split} \\
\end{align}

\begin{equation}\label{e:barwq}\begin{split}
H_c&=\frac{1}{2n} \sum^n_{l=0}(-1)^{l}(n-{l})^{p-2}
\sum_{l _1+\dots+ l _p=l}\prod^p_{i=1} \binom{n_i}{l _i}\\ 
&\quad\cdot[(n-l )-(n_i-l _i)]^{n_i-l _i}\cdot
\Bigl[(n-l )^2-\sum^p_{j=1}(n_i-l _i)^2\Bigr].
\end{split}\end{equation}

	\begin{align}
        		\boldsymbol{\omega}[\boldsymbol{x},\theta] +
        		\lambda
        		\left[\begin{array}{c}
        		\boldsymbol{d_1}[\boldsymbol{x}]^T \\
        		\boldsymbol{d_2}[\boldsymbol{x}]^T
        		\end{array}\right] \cdot \boldsymbol{h_x}'' 
        		& -
        		\big(\lambda\frac{\partial \theta}{\partial \boldsymbol{x}}(s) \cdot \boldsymbol{h_x}\big)
        		\left[\begin{array}{c}
        		\boldsymbol{d_2}[\boldsymbol{x},\theta]\cdot \boldsymbol{x}''\\
        		-\boldsymbol{d_1}[\boldsymbol{x},\theta]\cdot \boldsymbol{x}''\\
        		\end{array}\right] \\
        		& - 
        		\lambda(\boldsymbol{x}'' \cdot \boldsymbol{x}')
        		\left[\begin{array}{c}
        		\boldsymbol{d_1}[\boldsymbol{x},\theta]\cdot \boldsymbol{h_x}'\\
        		\boldsymbol{d_2}[\boldsymbol{x},\theta]\cdot \boldsymbol{h_x}''\\
        		\end{array}\right] \\
        		& + o(\lambda)
	\end{align}

% #######################################################
\section{Discretization}
% #######################################################




% #######################################################
\section{Connection}
% #######################################################


\bibliographystyle{alpha}
\bibliography{../bibliography}