%%%%%%%%%%%%%%%%%%%%%%%%%%%%%%%%%%%%%%%%%%%%%%
%
%		Thesis Settings
%		Custom settings
%
%		2011
%
%%%%%%%%%%%%%%%%%%%%%%%%%%%%%%%%%%%%%%%%%%%%%%

%
%   Use this file for your own custom packages, command-definitions, etc...
%

% LDP Packages
%--------------------------------------
\usepackage{amsmath}
\usepackage{amsthm}
\usepackage{amssymb}
\usepackage{stmaryrd}
\usepackage{lipsum}
\usepackage{MnSymbol}


\usepackage{natbib}
% use [sectionbib] option so bibliography appears like a not numbered section at the end of each chapter
\usepackage[sectionbib]{chapterbib}


% LDP Theorem
%--------------------------------------
\newtheoremstyle{mystyleA} % style name
    {}              % space above, empty = `usual value'
    {20pt}          % space below
    {}              % body font
    {}              % indent
    {\bfseries}     % head font
    {.}             % punctuation after head
    {5pt}           % space after head (\newline)
    {}              % Thm head spec

\theoremstyle{mystyleA}
\newtheorem*{mydef}{Definition}
\newtheorem*{myrk}{Remark}

\newtheoremstyle{mystyleB} % style name
    {}              % space above, empty = `usual value'
    {20pt}          % space below
    {\itshape}      % body font
    {}              % indent
    {\itshape}      % head font
    {.}             % punctuation after head
    {5pt}           % space after head (\newline)
    {}              % Thm head spec
\theoremstyle{mystyleB}
\newtheorem*{myproof}{Preuve}

% LDP Custom functions
% ---------------
\newcommand{\fonction}[5]{\begin{array}{lrcl}
#1: & #2 & \longrightarrow & #3 \\
    & #4 & \longmapsto & #5 \end{array}}
% \fonction{f}{E}{F}{x}{f(x)}

% the following lines are for creating a simplified TO-DO box. However since boites is not per default installed with all latex-distributions, we have removed this example again
% if you want to use it and do not have "boites" installed, you can get it from here: http://www.ctan.org/tex-archive/macros/latex/contrib/boites
%
%\usepackage{boites,boites_exemples}
%\newcommand{\todolist}[1]{\begin{boiteepaisseavecuntitre}{TO DO in this chapter} #1 \end{boiteepaisseavecuntitre}}  % creates a little box
% %\newcommand{\todolist}[1]{}  % to be used when to do is not to be printed
