%%%%%%%%%%%%%%%%%%%%%%%%%%%%%%%%%%%%%%%%%%%%%%
%
%		Thesis Settings
%		Custom settings
%
%		2011
%
%%%%%%%%%%%%%%%%%%%%%%%%%%%%%%%%%%%%%%%%%%%%%%

%
%   Use this file for your own custom packages, command-definitions, etc...
%

% LDP Packages
%--------------------------------------
\usepackage{amsmath}
\usepackage{amsthm}
\usepackage{amssymb}
\usepackage{stmaryrd}
\usepackage{lipsum}
\usepackage{MnSymbol}
\usepackage{minted}
\usepackage{tikz}
\usepackage{pgf,pgfplots}
\usepackage{xcolor}

\usepackage{natbib}
% use [sectionbib] option so bibliography appears like a not numbered section at the end of each chapter
\usepackage[sectionbib]{chapterbib}


% LDP Theorem
%--------------------------------------
\newtheoremstyle{mystyleA} % style name
    {}              % space above, empty = `usual value'
    {20pt}          % space below
    {}              % body font
    {}              % indent
    {\bfseries}     % head font
    {.}             % punctuation after head
    {5pt}           % space after head (\newline)
    {}              % Thm head spec

\theoremstyle{mystyleA}
\newtheorem*{mydef}{Definition}
\newtheorem*{myrk}{Remark}

\newtheoremstyle{mystyleB} % style name
    {}              % space above, empty = `usual value'
    {20pt}          % space below
    {\itshape}      % body font
    {}              % indent"
    {\itshape}      % head font
    {.}             % punctuation after head
    {5pt}           % space after head (\newline)
    {}              % Thm head spec
\theoremstyle{mystyleB}
\newtheorem*{myproof}{Preuve}

% LDP PfgPlots definition
%--------------------------------------
\pgfplotsset{width=10cm, compat=1.10}
%\pgfplotsset{every axis plot post/.append  style={line width = 2pt}}
\pgfplotsset{grid style = {
%    dash pattern = on 0.05mm off 1mm,
    line cap = round,
    Tgray,
    line width = 0.25pt}
}
\usepgfplotslibrary{fillbetween}

% 3pt-circle-curvature where curvatureA(phi, alpha)
\def\curvatureA(#1,#2){2*sin(#1)/(1+(#2)^2 + 2*(#2)*cos(#1))^0.5}
%  Bi-tangent-circle-curvature where curvatureA(phi, alpha)
\def\curvatureB(#1,#2){4*tan(#1/2)/(1+(#2))}
% curvature A / curvature B
\def\curvatureRap(#1,#2){(1+#2)/(1+(#2)^2 + 2*(#2)*cos(#1))^0.5*cos(#1/2)^2}


\definecolor{Tblue}{RGB}{104, 135, 255}
\definecolor{Tred}{RGB}{255, 110, 103}
\definecolor{Tgreen}{RGB}{147, 250, 103}
\definecolor{Tgray}{RGB}{184, 184, 184}
\definecolor{Tdarkgray}{RGB}{56, 56, 56}

% LDP Custom functions
%--------------------------------------
    
% typo
\newcommand{\guil}[1]{\og#1\fg{}}
\newcommand{\note}[1]{\color{blue}#1\color{black}}
 
% fonction inline (SHORT)
\newcommand{\fonction}[3]{#1 : #2 \longmapsto #3}

% fonction 2 lines (LONG)
\newcommand{\fonctionL}[5]{\begin{array}{lrcl}
#1: & #2 & \longrightarrow & #3 \\
    & #4 & \longmapsto & #5 \end{array}}

% differential
\newcommand{\diff}[2]{\boldsymbol{D}#1(#2)}
\newcommand{\diffN}[3]{\boldsymbol{D}^#1#2(#3)}
\newcommand{\diffof}[3]{\boldsymbol{D}#1(#2)\cdot#3}
\newcommand{\diffNof}[4]{\boldsymbol{D}^#1#2(#3)\cdot#4}

% partial differential
\newcommand{\pdiff}[3]{\boldsymbol{D}_#1#2(#3)}
\newcommand{\pdiffN}[4]{\boldsymbol{D}^#1_#2#3(#4)}
\newcommand{\pdiffof}[4]{\boldsymbol{D}_#1#2(#3)\cdot#4}
\newcommand{\pdiffNof}[5]{\boldsymbol{D}^#1_#2#3(#4)\cdot#5}

% vector and matrix
\newcommand{\vect}{\boldsymbol}
\newcommand{\mat}[1]{\boldsymbol{\mathit{#1}}} 

\newcommand{\scalar}[2]{\langle #1\,; #2\rangle}

\newcommand{\grad}[1]{grad\;#1}
\newcommand{\norm}[1]{\left\|#1\right\|    }

\newcommand{\overbar}[1]{\mkern 1.5mu\overline{\mkern-1.5mu#1\mkern-1.5mu}\mkern 1.5mu}

% \fonction{f}{E}{F}{x}{f(x)}

% the following lines are for creating a simplified TO-DO box. However since boites is not per default installed with all latex-distributions, we have removed this example again
% if you want to use it and do not have "boites" installed, you can get it from here: http://www.ctan.org/tex-archive/macros/latex/contrib/boites
%
%\usepackage{boites,boites_exemples}
%\newcommand{\todolist}[1]{\begin{boiteepaisseavecuntitre}{TO DO in this chapter} #1 \end{boiteepaisseavecuntitre}}  % creates a little box
% %\newcommand{\todolist}[1]{}  % to be used when to do is not to be printed
