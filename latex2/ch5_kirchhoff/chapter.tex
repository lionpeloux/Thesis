\chapter{Elastic rod : equilibrium approach}

\section{Introduction}
Ici on explique que l'approche par les équations d'équilibre est beaucoup plus directe que l'approche énergétique.

\subsection{Goals and contribution}
Dans ce chapitre, après un bref rappel sur le cadre mathématique d'étude des courbes paramétrique de l'espace, on présente les notions de courbures et de torsion géométrique associées au repère de fraient. On montre ensuite le cas plus général d'un repère mobile quelconque attaché à une courbe gamma. On définit enfin la particularité d'un repère mobile adapté à un courbe, et on présente, en sus du repère de Frenet, une approche différente pour accrocher des repères le long d'une courbe (Bishop / RMF / Zéro-twisting frame)

Ici il faudrait préciser la terminologie des auteurs / équations / hypothèses :
Euler-Bernoulli, Navier-Bernoulli, Kirchhoff, Love, Clebesh, Cosserat, Vlassov

\subsection{Related work}
On peu s'instruire dans la publi de Dill \cite{Dill1992}.
Regarder en particulier le premier chapitre de l'HDR de Neukirch \cite{Neukirch2009}.
Regarder également la chronologie des modèles proposée dans la thèse de Theetten \cite{Theetten2007}.
Pourquoi pas proposer une frise chronologique + un tableau de synthèse des hyptohèses.

\cite{Dill1992}
\citet{Neukirch2009}
\cite{Adriaenssens1999}
\cite{Hoogenboom2006}
\cite{Lang2009}
\cite{Spillmann2008}
\cite{Antman2005}

\cite{Neukirch2009} : p69 - \cite{Dill1992} : p16

Dans les tentatives dans notre domaine, citer :

Kirchhoff : \cite{Kirchhoff1850, Kirchhoff1876} \\
Clebsch : \cite{Clebsch1883} \\
Love : \cite{Love1892} \\
Timoshenko : \cite{Timoshenko1921, Timoshenko1922, Timoshenko1951}

\blockcquote[p.~607]{Langer1996}{Note that $\gamma$ having unit speed corresponds to the rod being inextensible; this is not always assumed in the theory, nor is the material frame necessarily assumed to be orthonormal as it is here}

\blockcquote[p.~607]{Langer1996}{Natural frames and the curve angle representation of rod}


Départ :
\cite{Day1965} : already includes a rotational DOF !!
\cite{Wakefield1980}
\cite{Barnes1999} : revue intéressante de la DR.

3 pts classique :
\cite{Adriaenssens1999}
\cite{Douthe2006}

2 x 3pts :
\cite{Barnes2013}

6 Dofs :
\cite{DAmico2014}

4Dofs :
\cite{DuPeloux2015}
\cite{DAmico2016}

Dans le champ de l'animation  avec élément finis
\cite{Duan2013}
\cite{Meier2014}


\subsection{Overview}
Résumé du chapitre

\blockcquote[p.~xvii]{Benvenuto1991b}{The battle between weight and rigidity constitutes, in itself, the single aesthetic theme of art in architecture : and to bring out this conflict in the most varied and clearest way is its office.}

\blockcquote[p.~xvii]{Villaggio1997}{The theory of elastic structures is, by definition, the collection of all reasonable models, proposed during almost three centuries, concerned with simplifying the solutions of problems involving elastic bodies. The equations describing the motion and equilibrium of a three-dimensional elastic body were formulated in full generality during the first half of the nineteenth century, but their solutions are known only in a few cases.}

\blockcquote[p.~68]{Audoly2010}{In a deformed state, the center line has no particular reason to remain straight and, in general, $\vect{d}_1$ and $\vect{d}_2$ will twist along the center line. However, in the case of small strain that we consider, the triad $(\vect{d}_1,\vect{d}_2,\vect{d}_3)$ remains approximately orthonormal, provided it has been chosen orthonormal in the reference configuration. This is known as the Euler-Bernoulli or Navier-Bernoulli kinematical hypothesis,or sometimes the assumption of unshearable rods.}

\section{Dynamical equations for Kirchhoff rods}

Attention, en faisant l'hypothèse du repère mobile attaché à la courbe pour représenter les sections, ont fait une hypothèse sérieuse.

A thorough order-of-magnitude analysis is exposed in \cite{Dill1992, Coleman1993}

A larger scope full development is given in chapter 8 from \cite[pp.~270-274]{Antman2005}

\footnote{\blockcquote[p.~1]{Coleman1993}{We discuss here the dynamical equations of a theory of elastic rods that is due to Kirchhoff and Clebsch. This properly invariant theory is applicable to motions in which the strains relative to an undistorted configuration remain small, although rotations may be large. It is constructed to be a first-order theory, i.e., a theory that is complete to within an error of order two in an appropriate dimensionless measure of thickness, curvature, twist, and extension.}}
\footnote{\blockcquote[p.~1]{Coleman1993}{In a first-order theory of thin rods, one can treat the rod as inextensible [\dots]}}

\blockcquote[p.~270]{Antman2005}{[we formulate] a general dynamical theory of rods that can undergo large deformations in space by suffering flexure, torsion, extension, and shear. We call the resulting geometrically exact theory the \emph{special Cosserat theory of rods}.}

\blockcquote[p.~238]{Antman2005}{The classical elastic rod theory of Kirchhoff (1859), called the kinetic analogue, is is a special case of our rod theory [...]}

\blockcquote[p.~341]{Antman2005}{If the reference configuration is not straight, then the uncoupling between the extension and the flexure and shear is lost.}

\cite[pp.~283]{Antman2005} defines the stretch (strain rate) as :
\begin{equation}
	v_3 = \vect{x}' \cdot (\vect{d}_1 \times \vect{d}_2) =  \vect{x}' \cdot \vect{d}_3 = 1+\epsilon
\end{equation}

\subsection{Assumptions}

\begin{itemize}
	\item axial extension is small
	\item sections remain planar in the deformed configuration
	\item sections remlain perpendicular to the centerline in the deformed configuration
\end{itemize}

ces équations sont valables à l'ordre 2 en $\alpha$ où :
\begin{equation}
	\alpha = \max_{s \in [0,L]} \{ \abs{\kappa(s)} h, \abs{\overbar{\kappa}(s)} h, h/L\}
\end{equation}

On fait par ailleurs l'hypothèse inextensible pour la dérivation du repère materiel.

dynamical equations

Writing the balance of linear and angular momentum of a beam slice of infinitesimal yields to the dynamic Kirchhoff equations for a slender beam. An extensive proof of this development is available in \cite{Dill1992}.

\footnote{\blockcquote[p.~5]{Dill1992}{The principal normal, binormal, and torsion of the axis, viewed as an element of a space curve, have no special significance in the theory of rods. Use of those special directions as base vectors does not simplify the theory and can mislead the reader into attributing significance to them when none exists. In particular, the curvature of the rod should not be confused with the curvature of the space curve which the axis forms.}}
\footnote{\blockcquote[p.~18]{Dill1992}{Kirchhoff's theory can only apply to that class of problems for three dimensional bodies such that the loads on the sides are relatively small and slowly varying. The dominate mode of deformation must be a global bending and twisting with small axial extension. If there are substantial local variations in curvatures or substantial transverse shears, his theory of bending of rods will not provide a satisfactory first approximation.}}
\footnote{\blockcquote[p.~15]{Dill1992}{There are no constitutive relations for $F_1$ or $F_2$. They are determined by the balance of momentum as in the elementary linear theory of bending of rods.}}

\begin{figure}[t]
	\centering
	\includegraphics[]{kirchhoff_law.pdf}
	\caption{Internal forces ($\vect{F}$) and moments ($\vect{M}$) acting on an infinitesimal beam slice of length $ds$. The beam is also subject to distributed external forces ($\vect{f}$) and moments ($\vect{m}$).} By convention, internal forces and moments are forces and moments applied by the right part to the left part of the beam.
	\label{fig:5_0}
\end{figure}

\subsection{Balance of linear momentum}
On fait un bilan sur une tranche d'épaisseur $ds$, de centre de gravité $G$ positionné en $\vect{x}_G$ :
\begin{equation}
	\vect{F}(s+ds)-\vect{F}(s) + \vect{f}(s)ds = \left(\frac{\partial \vect{F}}{\partial s}(s)+\vect{f}(s)\right)ds = (\rho Sds)\ddot{\vect{x}}_G
\end{equation}
Which leads to the first equation of Kirchhoff law :
\begin{equation}
	\frac{\partial \vect{F}}{\partial s}+\vect{f} = \rho S \ddot{\vect{x}}_G
\end{equation}

\subsection{Balance of angular momentum}
On fait un bilan sur une tranche d'épaisseur $ds$, de centre de gravité $G$ positionné en $\vect{x}_G$. On applique le théorème du moment cinétique dans un référentiel inertiel :
\begin{equation}
	\begin{aligned}
		\frac{d}{dt}(dI_G) &=
		\vect{M}(s+ds)-\vect{M}(s) + \vect{m}(s)ds
		+ (\tfrac{1}{2}ds\vect{x}')\times \vect{F}(s+ds) + (-\tfrac{1}{2}ds\vect{x}')\times -\vect{F}(s)\\
		&= \left(\frac{\partial \vect{M}}{\partial s}(s)+\vect{m}(s) + \vect{x}'\times \vect{F}(s)\right)ds
	\end{aligned}
\end{equation}
L'évolution temporelle des vecteurs matériels est cette fois décrite par un vecteur de Darboux temporel -- spin vector in \cite{Coleman1993} -- noté $\vect{\Lambda}$ tel que :
Compatibility equation between the curvature vector and the spin vector ($\dot{\kappa\vect{b}} - \vect{\Lambda}^{'} = \vect{\Lambda} \times \kappa\vect{b}$).
\begin{equation}
	\dot{\vect{d}_{i}}(s) =\vect{\Lambda}(t) \times \vect{d}_i(s)	\quad,\quad
	\vect{\Lambda}(t)
	=
	\begin{bmatrix}
		\Lambda_3(t) \\
		\Lambda_1(t) \\
		\Lambda_2(t)
	\end{bmatrix}
\end{equation}
Les lois de composition / dérivation de la mécanique nous permettent décrire :
\begin{equation}
	\begin{aligned}
		\frac{d}{dt}(dI_G) &= dI_G\dot{\vect{\Lambda}} + \vect{\Lambda}\times dI_G
	\end{aligned}
\end{equation}
Qu'est ce qu'on met dans $dI_G$ ? Et bien tout simplement l'opérateur d'inertie de la section, qui s'exprime à l'aide des moments quadratiques des directions principales de la façon suivante, dans la base des directions principales d'inertie au premier ordre en $ds$ :
\begin{equation}
	dI_G =
	 \begin{bmatrix}
			dI_{G3} & 0 & 0 \\
			0 & dI_{G1} & 0 \\
			0 & 0 & dI_{G2}
	\end{bmatrix}
	\simeq \rho ds
		\begin{bmatrix}
			I_1 + I_2 & 0 & 0 \\
			0 & I_1 & 0 \\
			0 & 0 & I_2
		\end{bmatrix}
\end{equation}
Where :
\begin{subequations}
	\begin{align}
		dI_{G3} &= \int_V \rho (x_1^2+ x_2^2)\;dV
		\simeq \rho ds \int_V (x_1^2+x_2^2)\;dx_1dx_2
		\simeq \rho ds (I_1 + I_2)
		\\
		dI_{G1} &= \int_V \rho (x_2^2+ x_3^2)\;dV
		\simeq \rho ds \int_V x_2^2\;dx_1dx_2
		\simeq \rho ds I_1
		\\
		dI_{G2} &= \int_V \rho (x_1^2+ x_3^2)\;dV
		\simeq \rho ds \int_V x_1^2\;dx_1dx_2
		\simeq \rho ds I_2
	\end{align}
\end{subequations}
Et l'on peut alors écrire la seconde loi de Kirchhoff sous la forme suivante :
\begin{equation}
	\begin{aligned}
		\frac{\partial \vect{M}}{\partial s}(s)+\vect{m}(s) + \vect{x}'\times \vect{F}(s)
		= \rho
			\begin{bmatrix}
				(I_1 + I_2)\dot{\Lambda}_3 + (I_2 - I_1)\Lambda_1\Lambda_2&\\
				I_1 (\dot{\Lambda}_1 + \Lambda_2 \Lambda_3) &\\
				I_2 (\dot{\Lambda}_2 - \Lambda_3 \Lambda_1) &
			\end{bmatrix}
	\end{aligned}
\end{equation}
On peut alors conclure sur l'expression de l'equation de kirchoff : \footnote{Recall that : $\dot{\vect{d}_i} = \vect{\Lambda} \times \vect{d}_i$}\textsuperscript{,}\footnote{Remark that : $(\vect{\Lambda}\times\dot{\vect{d}_i})\times\vect{d}_i = \Lambda_i(\vect{\Lambda}\times\dot{\vect{d}_i})$}
\begin{equation}
	\frac{\partial \vect{M}}{\partial s}(s)+\vect{m}(s) + \vect{d}_3 \times \vect{F}(s) = I_1 \vect{d}_1 \times \ddot{\vect{d}_1} + I_2 \vect{d}_2 \times \ddot{\vect{d}_2}
\end{equation}

\section{Equations of motion}
\subsection{Constitutive equations}

"An order-of-magnitude analysis leading ..."

Attention, pas d'effort normal par loi constitutive en principe car on est dans un modèle inextensible.
L'effort normal est calculé par la loi d'équilibre avec les moments et/ou efforts tranchants.
Ici, on postulera tout de même une telle loi constitutive pour la résolution numérique. Ce qui nous amène à considérer une tige quasiment inextensible.

\note{
point à creuser. en gros je suis entrain de dire que dans le modèle classique à 3DOF type Douthe ou Barnes, il n'est pas nécessaire d'introduire la raideur axiale (mais alors où intervient la section ?). L'effort normal est déduit des équations d'équilibre.

En fait cela ne semble pas possible. Il faut alors revenir à l'équation constitutive qui donne l'effort normal, mais alors quid de l'hypothèse quasistatique ?

Dans le fond, l'hyptohèse d'inextensibilité c'est dire que les déformations axiales sont négligeable devant les autres modes de déformation (flexion et/ou torsion).
Mais pour caractériser l'effort normal lui même, il faut bien considérer une élongation.

Ou alors, peut-être qu'il faut comprendre que l'effort normal est déduit uniquement des conditions aux limites et/ou éventuellement des efforts extérieurs appliqués à la centerline.

Pour comprendre le traitement de l'inextensibilité, regarder \cite{Antman2005} p50.
Qu'apporte l'hypothèse d'inextensibilité. Est-elle raisonnable. Tps de calcul par rapport au cas extensible.
}

\begin{subequations}
	\begin{align}
		\vect{N} &= ES \epsilon \vect{d}_3
		\\
		\vect{M_1} &= EI_1(\kappa_1-\overbar{\kappa_1})\vect{d}_1
		\\
		\vect{M_2} &= EI_2(\kappa_2-\overbar{\kappa_2})\vect{d}_2
		\\
		\vect{Q} &= [GJ(\theta'-\overbar{\theta}') - EC_w(\theta'''-\overbar{\theta}''')]\vect{d}_3
	\end{align}
\end{subequations}

where :
\begin{equation}
	I_1 = \int_S x_2^2\;dx_1 dx_2
\end{equation}
\begin{equation}
	I_2 = \int_S x_1^2\;dx_1 dx_2
\end{equation}
\begin{equation}
	J = \int_S (x_1^2+ x_2^2 + x_1 \frac{\partial \phi}{\partial x_2} - x_2 \frac{\partial \phi}{\partial x_1})\;dx_1 dx_2
\end{equation}

with $\phi(x_1, x_2)$ is the warping function of the cross section.


\subsection{Internal forces and moments}
Efforts internes de coupure :
\begin{subequations}
	\begin{align}
		\vect{F}_{int} &= N\vect{d}_3 + F_1\vect{d}_1 + F_2\vect{d}_2
		\\
		\vect{M}_{int} &= Q\vect{d}_3 + M_1\vect{d}_1 + M_2\vect{d}_2
	\end{align}
\end{subequations}
Efforts externes appliqués linéiques :
\begin{subequations}
	\begin{align}
		\vect{f}_{ext} &= f_3\vect{d}_3 + f_1\vect{d}_1 + f_2\vect{d}_2
		\\
		\vect{m}_{ext} &= m_3\vect{d}_3 + m_1\vect{d}_1 + m_2\vect{d}_2
	\end{align}
\end{subequations}

\subsection{Rod dynamic}

First Kirchhoff law projecting on the material frame basis :
\begin{subequations}
	\begin{align}
		N' + \kappa_1 F_2 - \kappa_2 F_1 + f_3 &= \rho S \ddot{x_3}
		\label{eq:K1_3}\\
		F_1' + \kappa_2 N - \tau F_2 + f_1 &= \rho S \ddot{x_1}
		\label{eq:K1_1}\\
		F_2' - \kappa_1 N + \tau F_1 + f_2 &= \rho S \ddot{x_2}
		\label{eq:K1_2}
	\end{align}
\end{subequations}
Qu'on écrit vectoriellement : 
\begin{equation}
	\vect{F}' + \vect{\Omega} \times \vect{F} + \vect{f}_{ext} = \rho S \ddot{\vect{x}} 
	\quad with \quad
	\vect{F}' = 
	\begin{bmatrix}F'_1 & F'_2 & N'\end{bmatrix}^T
\end{equation}
This is nothing but the application of the transport theorem when differentiating a vector expressed in the material frame : 
\begin{equation}
	\frac{\partial \vect{F}}{\partial s}\Bigr|_{global} = \frac{\partial \vect{F}}{\partial s}\Bigr|_{local} + \vect{\Omega}(s) \times \vect{F}
\end{equation}

Second Kirchhoff law projecting on the material frame basis : \footnote{As explained in \cite[p. 18]{Dill1992}, if the inextensibility assumption does not hold, the right terms to consider are $-(1+\epsilon)F_2$ in \cref{eq:K2_1} and $(1+\epsilon)F_1$ in \cref{eq:K2_2}.}
\begin{subequations}
	\begin{align}
		Q' + \kappa_1 M_2 - \kappa_2 M_1 + m_3 &= (I_1 + I_2)\dot{\Lambda}_3 + (I_2 - I_1)\Lambda_1\Lambda_2
		\label{eq:K2_2}\\
		M_1' + \kappa_2 Q - \tau M_2 - F_2 + m_1 &= I_1 (\dot{\Lambda}_1 + \Lambda_2 \Lambda_3)
		\label{eq:K2_1}\\
		M_2' - \kappa_1 Q + \tau M_1 + F_1 + m_2 &= I_2 (\dot{\Lambda}_2 - \Lambda_3 \Lambda_1)
		\label{eq:K2_2}
	\end{align}
\end{subequations}
Qu'on écrit vectoriellement : 
\begin{align}
	\vect{M}^{'} + \vect{\Omega} \times \vect{M} + \vect{m}_{ext} + \vect{d}_3 \times \vect{F} = I_1 \vect{d}_1 \times \ddot{\vect{d}_1} + I_2 \vect{d}_2 \times \ddot{\vect{d}_2} \\
	\quad with \quad
	\vect{M}^{'} = 
	\begin{bmatrix}M'_1 & M'_2 & Q'\end{bmatrix}^T
\end{align}
This is nothing but the application of the transport theorem when differentiating a vector expressed in the material frame : 
\begin{equation}
	\frac{\partial \vect{M}}{\partial s}\Bigr|_{global} = \frac{\partial \vect{M}}{\partial s}\Bigr|_{local} + \vect{\Omega}(s) \times \vect{M}
\end{equation}


\clearpage
\makebox[\textwidth]{} % necessaire pour le saut de page et le flush des floats
%\newpage
\section{Geometric interpretation}

The previous section has established the dynamical equations for elastic rods. This section gives a simple and straight forward geometric interpretation of this equations as they can be 
 
\begin{figure}[h]
	\centering
	\includegraphics[]{kirchhoff_geometry.pdf}
	\caption{Osculating circles for a spiral curve at different parameters.}
	\label{fig:5}
\end{figure} 

\begin{figure}[p]
  \begin{leftfullpage}
    \captionsetup[subfloat]{captionskip=10pt}
     	\centering
     	\subfloat[][Infinitesimal deformation.]{\includegraphics{kirchhoff_geometry_kb.pdf}\label{fig:5_2_a}} \\
	\vspace{30pt}
	\subfloat[][Contributions of the internal forces.]{\includegraphics{kirchhoff_balance_T_kb.pdf}\label{fig:5_2_b}}
	\subfloat[][Contributions of the internal moments.]{\includegraphics{kirchhoff_balance_M_kb.pdf}\label{fig:5_2_c}}
	\vspace{30pt}
	\caption{Influence of the curvature ($\kappa$) in the deflection of internal forces and moments along the centerline.}     
	\label{fig:5_2}
 \end{leftfullpage}
\end{figure}
\begin{figure}[p]
	\begin{fullpage}
	\subsubsection{Contributions to the balance of forces}
	\vspace{10pt}
	$\vect{N}(s+ds)$ is deflected from $\vect{d}_3(s)$ by the rotation of angle $\kappa ds$ around $\vect{\kappa b}$ (\cref{fig:5_2_b}). Thus, its contribution to the balance of forces onto $\vect{d}_3(s)$ is : 
	\begin{equation*}
		N(s+ds) \cos(\kappa ds) - N(s) = N'(s) ds + o(ds)
	\end{equation*}	
	\vspace{10pt}
	
	\subsubsection{Contributions to the balance of moments}
	\vspace{10pt}
	$\vect{Q}(s+ds)$ is deflected from $\vect{d}_3(s)$ by the rotation of angle $\kappa ds$ around $\vect{\kappa b}$ (\cref{fig:5_2_c}). Thus, its contribution to the balance of moments onto $\vect{d}_3(s)$ is : 
	\begin{equation*}
		Q(s+ds) \cos(\kappa ds) - Q(s) = Q'(s) ds + o(ds)
	\end{equation*}	
	  \end{fullpage}
\end{figure}

% ============
 
\begin{figure}[p]
  \begin{leftfullpage}
    \captionsetup[subfloat]{captionskip=10pt}
     	\centering
     	\subfloat[][Infinitesimal deformation.]{\includegraphics{kirchhoff_geometry_d1.pdf}\label{fig:5_3_a}} \\
	\vspace{30pt}
	\subfloat[][Contributions of the internal forces.]{\includegraphics{kirchhoff_balance_T_d1.pdf}\label{fig:5_3_b}}
	\subfloat[][Contributions of the internal moments.]{\includegraphics{kirchhoff_balance_M_d1.pdf}\label{fig:5_3_c}}
	\vspace{30pt}
	\caption{Influence of the first material curvature ($\kappa_1$) in the deflection of internal forces and moments along the centerline.}     
	\label{fig:5_3}
 \end{leftfullpage}
\end{figure}
\begin{figure}[p]
	\begin{fullpage}
	\subsubsection{Contributions to the balance of forces}
	\vspace{10pt}
	
	$\vect{T}_2(s+ds)$ is deflected from $\vect{d}_2(s)$ by the combined rotations of angle $\tau ds$ around $\vect{d}_3$ and $\kappa_2 ds$ around $\vect{d}_2$ (\cref{fig:5_3_b}). Thus, its contribution to the balance of forces onto $\vect{d}_1(s)$ is : 
	\begin{equation*}
		-T_2(s+ds) \sin(\tau ds) \cos(\kappa_2 ds) = -\tau T_2(s) ds + o(ds)
	\end{equation*}	
	
	$\vect{T}_2(s+ds)$ is deflected from $\vect{d}_2(s)$ by the combined rotations of angle $\tau ds$ around $\vect{d}_3$ and $\kappa_1 ds$ around $\vect{d}_1$ (\cref{fig:5_3_b}). Thus, its contribution to the balance of forces onto $\vect{d}_2(s)$ is : 
	\begin{equation*}
		-T_2(s) + T_2(s+ds) \cos(\tau ds) \cos(\kappa_1 ds) = T'_2 (s) ds + o(ds)
	\end{equation*}	
	
	$\vect{T}_2(s+ds)$ is deflected from $\vect{d}_2(s)$ by the combined rotations of angle $\tau ds$ around $\vect{d}_3$ and $\kappa_1 ds$ around $\vect{d}_1$ (\cref{fig:5_3_b}). Thus, its contribution to the balance of forces onto $\vect{d}_3(s)$ is : 
	\begin{equation*}
		T_2(s+ds) \cos(\tau ds) \sin(\kappa_1 ds) = \kappa_1 T_2(s) ds + o(ds)
	\end{equation*}
		
	$\vect{N}(s+ds)$ is deflected from $\vect{d}_3(s)$ by the combined rotations of angle $\kappa_2 ds$ around $\vect{d}_2$ and $\kappa_1 ds$ around $\vect{d}_1$ (\cref{fig:5_3_b}). Thus, its contribution to the balance of forces onto $\vect{d}_2(s)$ is : 
	\begin{equation*}
		-N(s+ds) \cos(\kappa_2 ds) \sin(\kappa_1 ds) = -\kappa_1 N(s) ds + o(ds)
	\end{equation*}	
	\vspace{10pt}

	\subsubsection{Contributions to the balance of moments}
	\vspace{10pt}
	
	$\vect{T}_2(s+ds)$ is deflected from the plane normal to $\vect{d}_1(s)$ by a rotation of angle $\tau ds$ around $\vect{d}_3$ (\cref{fig:5_3_b}). It produces a moment around $\vect{d}_1$ with the lever arm $b =  \cos(\kappa_2 ds) ds$. Thus, its contribution to the balance of moments onto $\vect{d}_1(s)$ is : 
	\begin{equation*}
		-T_2(s+ds) \cos(\tau ds) (\cos(\kappa_2 ds) ds) = -T_2(s) ds + o(ds)
	\end{equation*}
	
	$\vect{M}_2(s+ds)$ is deflected from $\vect{d}_2(s)$ by the combined rotations of angle $\tau ds$ around $\vect{d}_3$ and $\kappa_2 ds$ around $\vect{d}_2$ (\cref{fig:5_3_c}). Thus, its contribution to the balance of moments onto $\vect{d}_1(s)$ is : 
	\begin{equation*}
		-M_2(s+ds) \sin(\tau ds) \cos(\kappa_2 ds) = -\tau M_2 (s) ds + o(ds)
	\end{equation*}	
	
	$\vect{M}_2(s+ds)$ is deflected from $\vect{d}_2(s)$ by the combined rotations of angle $\tau ds$ around $\vect{d}_3$ and $\kappa_1 ds$ around $\vect{d}_1$ (\cref{fig:5_3_c}). Thus, its contribution to the balance of moments onto $\vect{d}_2(s)$ is : 
	\begin{equation*}
		-M_2(s) + M_2(s+ds) \cos(\tau ds) \cos(\kappa_1 ds) = M'_2 (s) ds + o(ds)
	\end{equation*}
	
	$\vect{M}_2(s+ds)$ is deflected from $\vect{d}_2(s)$ by the combined rotations of angle $\tau ds$ around $\vect{d}_3$ and $\kappa_1 ds$ around $\vect{d}_1$ (\cref{fig:5_3_c}). Thus, its contribution to the balance of moments onto $\vect{d}_3(s)$ is : 
	\begin{equation*}
		M_2(s+ds) \cos(\tau ds) \sin(\kappa_1 ds) = \kappa_1 M_2 (s) ds + o(ds)
	\end{equation*}	
	
	$\vect{Q}(s+ds)$ is deflected from $\vect{d}_3(s)$ by the combined rotations of angle $\kappa_2 ds$ around $\vect{d}_2$ and $\kappa_1 ds$ around $\vect{d}_1$ (\cref{fig:5_3_c}). Thus, its contribution to the balance of moments onto $\vect{d}_2(s)$ is : 
	\begin{equation*}
		-Q(s+ds) \cos(\kappa_2 ds) \sin(\kappa_1 ds) = -\kappa_1 Q(s) ds + o(ds)
	\end{equation*}	
	  \end{fullpage}
\end{figure}

% ============
 
\begin{figure}[p]
  \begin{leftfullpage}
    \captionsetup[subfloat]{captionskip=10pt}
     	\centering
     	\subfloat[][Infinitesimal deformation.]{\includegraphics{kirchhoff_geometry_d2.pdf}\label{fig:5_4_a}} \\
	\vspace{30pt}
	\subfloat[][Contributions of the internal forces.]{\includegraphics{kirchhoff_balance_T_d2.pdf}\label{fig:5_4_b}}
	\subfloat[][Contributions of the internal moments.]{\includegraphics{kirchhoff_balance_M_d2.pdf}\label{fig:5_4_c}}
	\vspace{30pt}
	\caption{Influence of the second material curvature ($\kappa_2$) in the deflection of internal forces and moments along the centerline.}     
	\label{fig:5_4}
 \end{leftfullpage}
\end{figure}
\begin{figure}[p]
	\begin{fullpage}
	\subsubsection{Contributions to the balance of forces}
	\vspace{10pt}
	
	$\vect{T}_1(s+ds)$ is deflected from $\vect{d}_1(s)$ by the combined rotations of angle $\tau ds$ around $\vect{d}_3$ and $\kappa_2 ds$ around $\vect{d}_2$ (\cref{fig:5_4_b}). Thus, its contribution to the balance of forces onto $\vect{d}_1(s)$ is : 
	\begin{equation*}
		-T_1(s) + T_1(s+ds) \cos(\tau ds) \cos(\kappa_2 ds) = T'_1 (s) ds + o(ds)
	\end{equation*}
	
	$\vect{T}_1(s+ds)$ is deflected from $\vect{d}_1(s)$ by the combined rotations of angle $\tau ds$ around $\vect{d}_3$ and $\kappa_1 ds$ around $\vect{d}_1$ (\cref{fig:5_4_b}). Thus, its contribution to the balance of forces onto $\vect{d}_2(s)$ is : 
	\begin{equation*}
		T_1(s+ds) \sin(\tau ds) \cos(\kappa_1 ds) = \tau T_1 (s) ds + o(ds)
	\end{equation*}	
	
	$\vect{T}_1(s+ds)$ is deflected from $\vect{d}_1(s)$ by the combined rotations of angle $\tau ds$ around $\vect{d}_3$ and $\kappa_2 ds$ around $\vect{d}_2$ (\cref{fig:5_4_b}). Thus, its contribution to the balance of forces onto $\vect{d}_3(s)$ is : 
	\begin{equation*}
		-T_1(s+ds) \cos(\tau ds) \sin(\kappa_2 ds) = - \kappa_2 T_1(s) ds + o(ds)
	\end{equation*}	
	
	$\vect{N}(s+ds)$ is deflected from $\vect{d}_3(s)$ by the combined rotations of angle $\kappa_1 ds$ around $\vect{d}_1$ and $\kappa_2 ds$ around $\vect{d}_2$ (\cref{fig:5_4_b}). Thus, its contribution to the balance of forces onto $\vect{d}_1(s)$ is : 
	\begin{equation*}
		N(s+ds) \cos(\kappa_1 ds) \sin(\kappa_2 ds) = \kappa_2 N(s) ds + o(ds)
	\end{equation*}
	\vspace{10pt}

	\subsubsection{Contributions to the balance of moments}
	\vspace{10pt}
	
		$\vect{T}_1(s+ds)$ is deflected from the plane normal to $\vect{d}_2(s)$ by the angle $\tau ds$ around $\vect{d}_3$ along $ds$ (\cref{fig:5_4_b}). It produces a moment around $\vect{d}_2$ with the lever arm $b =  \cos(\kappa_1 ds) ds$. Thus, its contribution to the balance of moments onto $\vect{d}_2(s)$ is : 
	\begin{equation*}
		T_1(s+ds) \cos(\tau ds) (\cos(\kappa_1 ds) ds) = T_1(s) ds + o(ds)
	\end{equation*}
	
	$\vect{M}_1(s+ds)$ is deflected from $\vect{d}_1(s)$ by the combined rotations of angle $\tau ds$ around $\vect{d}_3$ and $\kappa_2 ds$ around $\vect{d}_2$ (\cref{fig:5_4_c}). Thus, its contribution to the balance of moments onto $\vect{d}_1(s)$ is : 
	\begin{equation*}
		-M_1(s) + M_1(s+ds) \cos(\tau ds) \cos(\kappa_2 ds) = M'_1 (s) ds + o(ds)
	\end{equation*}	
	
	$\vect{M}_1(s+ds)$ is deflected from $\vect{d}_1(s)$ by the combined rotations of angle $\tau ds$ around $\vect{d}_3$ and $\kappa_2 ds$ around $\vect{d}_2$ (\cref{fig:5_4_c}). Thus, its contribution to the balance of moments onto $\vect{d}_2(s)$ is : 
	\begin{equation*}
		M_1(s+ds) \sin(\tau ds) \cos(\kappa_2 ds) = \tau M_1 (s) ds + o(ds)
	\end{equation*}	
	
	$\vect{M}_1(s+ds)$ is deflected from $\vect{d}_1(s)$ by the combined rotations of angle $\tau ds$ around $\vect{d}_3$ and $\kappa_2 ds$ around $\vect{d}_2$ (\cref{fig:5_4_c}). Thus, its contribution to the balance of moments onto $\vect{d}_3(s)$ is : 
	\begin{equation*}
		-M_1(s+ds) \cos(\tau ds) \sin(\kappa_2 ds) = -\kappa_2 M_1 (s) ds + o(ds)
	\end{equation*}	
	
	$\vect{Q}(s+ds)$ is deflected from $\vect{d}_3(s)$ by the combined rotations of angle $\kappa_1 ds$ around $\vect{d}_1$ and $\kappa_2 ds$ around $\vect{d}_2$ (\cref{fig:5_4_c}). Thus, its contribution to the balance of moments onto $\vect{d}_1(s)$ is : 
	\begin{equation*}
		Q(s+ds) \cos(\kappa_1 ds) \sin(\kappa_2 ds) = \kappa_2 Q(s) ds + o(ds)
	\end{equation*}	
	  \end{fullpage}
\end{figure}

\clearpage
\section{Numerical resolution}

\subsection{Main hypothesis}

On néglige les forces d'inertie liées à la rotation de l'élément  (devant quoi ?? traitement quasi-statique par rapport à la rotation). Cette hypothèse est faite explicitement chez Florence Bertail :

\blockcquote[]{Casati2013}{neglecting inertial momentum due to the vanishing cross-section lead to the following dynamic equations for a Kirchhoff rod}
\blockcquote[p. 17]{Dill1992}{It follows that $\omega_1$ and $\omega_2$ can be neglected in the kinetic energy [\ldots]. However, $\omega_3$, which provides the angular momentum about the axis of the rod, must be retained, This assumption of Kirchhoff is consistent with the technical theory of beams where rotary inertia is known to provide corrections to the natural frequencies of vibration of $O(\alpha^2)$ if the length measure is the half-wave length.}


Cette hypothèse est faite mais passée sous silence chez Douthe, Adriaenssen, D'Amico lorsqu'ils déduisent l'effort tranchant du moment de flexion.

Principe :

- les équations constitutives permettent le calcul de $M_1$, $M_2$, $Q$ à partir de la géométrie $\{\vect{x},\theta\}$.

- La seconde loi de kirchhoff projetée sur les axes matériels 1 et 2 de la section me donnent accès aux efforts tranchants $T_1$ et $T_2$.

- La seconde loi de kirchhoff projetée sur les axes matériel 3 (tangente à la centerline) de la section me donnent l"hypothèse quasi-statique de Audoly.

%\begin{figure}[t]
%	\centering
%	\includegraphics[]{discrete_model.pdf}
%	\caption{Biarc model for a discrete beam. The centerline is divided into curved segments (grey solid hatch). Each segment is defined as a 3-noded element with uniform material and section properties. It has two end vertices (white) called \emph{handle} as they are used to interact with the model, for instance to apply loads or restrains. It has one mid vertex (grey) called \emph{ghost} as it is used only to enrich the segment kinematics and is not accessible to the end user.}
%	\label{fig:discrete_model}
%\end{figure} 


\begin{figure}[p]
\begin{fullpage}
	\captionsetup[subfloat]{captionskip=20pt}
     	\centering
     	\subfloat[][Centerline of the discrete biarc model.]{\includegraphics{discrete_model.pdf}\label{fig:discrete_model}} \\
	\vspace{30pt}
	\subfloat[][Number of segments, edges and vertices whether the centerline is closed or open.]{
		\begin{tabular}{l  l | c  c } 
		 &  & open & closed \\
		\hline
		segments & $n_s$ & $n_s$ & $n_s$ \\
		\hline
		edges & $n_e$ & $2 n_s$ & $2 n_s$ \\
		\hline
		vertices & $n$ & $2 n_s + 1$ & $2 n_s$ \\
		\hline
		ghosts & $n_g$ & $n_s$ & $n_s$ \\
		\hline
		handles & $n_h$ & $n_s+1$ & $n_s$ \\
	\end{tabular}\label{tab:count}}
	\vspace{20pt}
	\caption{Biarc model for a discrete beam. The centerline is divided into curved segments (grey solid hatch). Each segment is defined as a three-noded element with uniform material and section properties. It has two end vertices (white) called \emph{handle} as they are used to interact with the model, for instance to apply loads or restrains. It has one mid vertex (grey) called \emph{ghost} as it is used only to enrich the segment kinematics and is not accessible to the end user.}
\end{fullpage}
\end{figure}



\clearpage
\subsection{Discret beam model}
% =======================



Let's introduce the discrete biarc model to describe the configuration of a beam. It is composed of a discrete curve called \emph{centerline} and a discrete adapted frame called \emph{material frame} as its axes are chosen to be the principal axes of the beam cross section (\cref{fig:discrete_model}). The centerline itself is organized in $n_s$ consecutive adjacent segments which are three-vertices and two-edges elements with uniform material and section properties.

Beams can either be closed or open. The corresponding number of vertices, edges and segments are reported in \cref{tab:count}.
%\begin{table}[h]
%	\centering
%	\begin{tabular}{l  l | c  c } 
%		 &  & open & closed \\
%		\hline
%		segments & $n_s$ & $n_s$ & $n_s$ \\
%		\hline
%		edges & $n_e$ & $2 n_s$ & $2 n_s$ \\
%		\hline
%		vertices & $n$ & $2 n_s + 1$ & $2 n_s$ \\
%		\hline
%		ghosts & $n_g$ & $n_s$ & $n_s$ \\
%		\hline
%		handles & $n_h$ & $n_s+1$ & $n_s$ \\
%	\end{tabular}
%\caption{Number of segments, edges and vertices whether the centerline is closed or open.}
%\label{tab:count}
%\end{table}

\subsubsection{Centerline}

The discrete centerline is a polygonal space curve (\cref{fig:discrete_model}) defined as an ordered sequence of $n+1$ pairwise disjoint \emph{vertices} : $\Gamma = (\vect{x}_0,  \vect{x}_1, \ldots, \vect{x}_n) \in \mathbb{R}^{3(n+1)}$. Consecutive pairs of vertices define $n$ straight segments $(\vect{e}_0,  \vect{e}_1, \ldots, \vect{e}_{n-1})$ called \emph{edges} and pointing from one vertex to the next one : $\vect{e}_i = \vect{x}_{i+1} - \vect{x}_{i}$ : 
\begin{equation}
%\setlength{\jot}{8pt}
	\left\{
	\begin{aligned}
		\vect{e}_i 	&= \vect{x}_{i+1} - \vect{x}_{i} \\
		l_i 		&= \norm{\vect{e}_i} \\
		\vect{u}_i 	&= \vect{e}_i / l_i\\
	\end{aligned}
	\right.
\end{equation}

The length of the ith edge is denoted $l_i $ and its normalized direction vector is denoted $\vect{u}_i$. The arc length of the ith vertex is denoted $s_i$ and is given by : 
\begin{equation}
	\left\{
	\begin{aligned}
		s_0 	&= 0 				& 	&i = 0		\\
		s_i 	&= \sum_{k=0}^{i-1} l_k	&	&i \in \llbracket 1, n-1 \rrbracket	\\
		s_n 	&=  L 				&	&i = n		\\
	\end{aligned}
	\right.
\end{equation}
Thus, the centerline is parametrized by arc length and $\Gamma(s_i) = \vect{x}_i$. Additionally, we define the vertex-based mean length at vertex $\vect{x}_i$ : 
\begin{equation}
%\setlength{\jot}{8pt}
	\left\{
	\begin{aligned}
		\overbar{l_0} 	& =  \tfrac{1}{2}l_0				&		&i = 0					\\
		\overbar{l_i}	& =  \tfrac{1}{2}(l_{i-1} + l_i)		&		&i \in \llbracket 1, n-1 \rrbracket	\\
		\overbar{l_n} 	& =  \tfrac{1}{2}l_{n-1} 			&		&i = n					\\
	\end{aligned}
	\right.
\end{equation}

\subsubsection{Segments}

The discrete centerline is divided into $n_s$ curved segments. Each segment is a three-noded element -- see \cref{fig:discrete_model} where the area covered by a segment is represented as a grey solid hatch. The ith segment is composed of three vertices ($\vect{x}_{2i}, \vect{x}_{2i+1},  \vect{x}_{2i+2}$) spanning two edges ($\vect{e}_{2i}, \vect{e}_{2i+1}$). The (i-1)th segment and the ith segment share the same vertex $\vect{x}_{2i}$ at arc length $s_{2i}$.

Each segment has two end vertices called \emph{handle} ($\vect{x}_{2i}, \vect{x}_{2i+2}$) and one mid vertex called \emph{ghost} ($\vect{x}_{2i+1}$) as this one is not accessible to the end user in order to interact with the model (link, restrain, loading, ...). Ghost vertices are used only for internal purpose to give a higher richness in the kinematic description of a segment than a two-noded segment would.

We define the \emph{chord length} of the ith segment as the distance between $\vect{x}_{2i}$ and $\vect{x}_{2i+2}$ : $L_i = \norm{\vect{e}_{2i} + \vect{e}_{2i+1}}$.

\subsubsection{Material and section properties}

In addition, the model assumes that a segment has uniform section ($S$, $I_1$, $I_2$, $J$)\footnote{$S$ is the cross section area ; $I_1$, $I_2$ and $J$ are the cross section principal moments of inertia.} and material ($E$, $G$)\footnote{$E$ is the elastic modulus and $G$ is the shear  modulus for the considered material} properties over its length : $s \in ]s_{2i},s_{2i+2}[$. For the sake of simplicity, we introduce for further calculations the \emph{material stiffness matrix} ($\mat{B}_i$) attached to each segment. It has the following form in the material frame basis :
\begin{equation}
	\mat{B}_i = \begin{bmatrix} 
			EI_1		&	0		&	0		\\
			0		&	EI_{2}	&	0		\\
			0		&	0		&	GJ_{}	\\
		\end{bmatrix}_i
\end{equation}
 

\subsubsection{External loads}

Also, the model assumes that each segment can be loaded with uniform external distributed forces ($\vect{f}_{ext}$) and moments ($\vect{m}_{ext}$).





\subsubsection{External loads}

External concentrated forces ($\vect{F}_{ext}$) and moments ($\vect{M}_{ext}$) are applied to the segment end vertices ($\vect{x}_{2i}$,  $\vect{x}_{2i+2}$).

This discret model involves that axial, bending and torsion strains, section and material properties will be continuous fonctions of the arc length over each segment $]\vect{x}_{2i},  \vect{x}_{2i+2}[$. Discontinuities in strains, internal and external forces, internal and external moments will be located at handle vertices. The left and right limits of this fonctions at handle vertices will be denoted respectively by $f^-$ and $f^+$. Possibly they are continuous at handle nodes that is the left and right limits agree ($f^- = f^+$).

Lets call : $l_i = \norm{\vect{e}_{i}}$ with $i \in [0,n_e]$.
Lets call : $u_i = \frac{\vect{e}_{i}}{l_i}$ with $i \in [0,n_e]$.


Lets call : $L_i = \norm{\vect{e}_{2i} + \vect{e}_{2i+1}}$ with $i \in [0,n_g]$.

We have : $\vect{d}_{3, i+1/2} = \vect{u}_i$

Let $\mat{B}_i$ be the material stiffness matrix along the principal axes of inertia, uniform over the slice $]\vect{x}_{2i},  \vect{x}_{2i+2}[$. Thus, it has the following form in the material basis :
\begin{equation}
	\mat{B}_i = \begin{bmatrix}
			EI_1		&	0		&	0		\\
			0		&	EI_{2}	&	0		\\
			0		&	0		&	GJ_{}	\\
		\end{bmatrix}_i
\end{equation}

Thus, one will write the constitutive equations for the bending moment in matrix notation as :
\begin{equation}
	\vect{M}_i =  \mat{B}_{i}(\vect{\kappa b}_{i} - \overbar{\vect{\kappa b}}_{i})
\end{equation}
With $\vect{\kappa b} = \begin{bmatrix}\kappa_1 & \kappa_2 & \tau \end{bmatrix}^T$ expressed in the material frame.


\subsection{Discret bending moments and curvatures}
We assume that the internal bending moment and curvature are quadratic functions of the arc length over $]\vect{x}_{2i},  \vect{x}_{2i+2}[$.
While they must be continuous over this interval, they might be discontinuous at handle vertices and be subjected to jump discontinuities in direction and magnitude.

\clearpage

\def\tabularxcolumn#1{m{#1}} % vertical center in X column
% -------------------------------------------------------------------------------------
\subsubsection{Curvature at ghost vertices}
% -------------------------------------------------------------------------------------

For a given geometry of the centerline, the curvature binormal vector at ghost vertex  $\vect{x}_{2i-1}$ (resp. $\vect{x}_{2i+1}$) is computed considering the circumscribed osculating circle passing through the vertices ($\vect{x}_{2i-2}, \vect{x}_{2i-1},  \vect{x}_{2i}$) of the ($i-1$)\textit{th} segment -- resp. through the vertices ($\vect{x}_{2i}, \vect{x}_{2i+1},  \vect{x}_{2i+2}$) of the $i$-\textit{th} segment.

\begin{tabularx}{\textwidth}[t]{>{\centering\arraybackslash}m{0.48\textwidth} >{\centering\arraybackslash}X} % >{\centering\arraybackslash}
	\includegraphics[]{E1.pdf}
	& 
	$\begin{aligned}[t] % placement: default is "center", options are "top" and "bottom"
	\vect{\kappa b}_{2i-1} 	& =  \frac{2}{L_{i-1}} \vect{u}_{2i-2} \times \vect{u}_{2i-1}\\[0.5em]
	\vect{\kappa b}_{2i+1} 	& =  \frac{2}{L_{i}} \vect{u}_{2i} \times \vect{u}_{2i+1} 
	\end{aligned}$
\end{tabularx}

% -------------------------------------------------------------------------------------
\subsubsection{Unit tangent vectors at ghost vertices}
% -------------------------------------------------------------------------------------

This definition of the curvature leads to a natural definition of the unit tangent vector at ghost vertex $\vect{x}_{2i-1}$ (resp. $\vect{x}_{2i+1}$), as the unit vector tangent to the osculating circle of the ($i-1$)\textit{th} segment (resp. $i$-\textit{th} segment) at that point. 

\begin{tabularx}{\textwidth}[t]{>{\centering\arraybackslash}m{0.48\textwidth} >{\centering\arraybackslash}X} % >{\centering\arraybackslash}
	\includegraphics[]{E2.pdf}
	& 
	$\begin{aligned}[t] % placement: default is "center", options are "top" and "bottom"
	\vect{t}_{2i-1}	&=  \frac{l_{2i-1}}{L_{i-1}} \vect{u}_{2i-2}	+ 	\frac{l_{2i-2}}{L_{i-1}} \vect{u}_{2i-1} 	\\[0.5em]
	\vect{t}_{2i+1} 	&=  \frac{l_{2i+1}}{L_{i}} \vect{u}_{2i}		+ 	\frac{l_{2i}}{L_{i}} \vect{u}_{2i+1} 	
	\end{aligned}$
\end{tabularx}

% -------------------------------------------------------------------------------------
\subsubsection{Left/right unit tangent vectors at handle vertices}
% -------------------------------------------------------------------------------------

Equivalently, the definition of the osculating circles of the ($i-1$)\textit{th} and $i$-\textit{th} segments leads to a natural definition of the left ($\vect{t}_{2i}^-$) and right ($\vect{t}_{2i}^+$) unit tangent vectors at handle vertex $\vect{x}_{2i}$, for segments of uniform curvature. When both segments have the same curvature, left and right vectors agree.

\begin{tabularx}{\textwidth}[t]{>{\centering\arraybackslash}m{0.48\textwidth} >{\centering\arraybackslash}X} % >{\centering\arraybackslash}
	\includegraphics[]{E3.pdf}
	& 
	$\begin{aligned}[t] % placement: default is "center", options are "top" and "bottom"
	\vect{t}_{2i}^- 	&= 2 (\vect{t}_{2i-1} \cdot \vect{u}_{2i-1}) \vect{u}_{2i-1} - \vect{t}_{2i-1} \\[0.5em]
	\vect{t}_{2i}^+ 	&= 2 (\vect{t}_{2i+1} \cdot \vect{u}_{2i}) \vect{u}_{2i} - \vect{t}_{2i+1}
	\end{aligned}$
\end{tabularx}

% -------------------------------------------------------------------------------------
\subsubsection{Unit tangent vectors at handle vertices}
% -------------------------------------------------------------------------------------

The unit tangent vector $\vect{t}_{2i}$ -- that is the beam section normal -- at handle vertex $\vect{x}_{2i}$ is chosen to be the mean of the left and right unit tangent vectors at that vertex.\footnote{Consequently, this model assumes that the field of tangents along the centerline is continuous and is thus unable to model cases where the centerline is not at least $\mathcal{C}^1$. In such case the beam must be considered as two parts glued together.}

\begin{tabularx}{\textwidth}[t]{>{\centering\arraybackslash}m{0.48\textwidth} >{\centering\arraybackslash}X} % >{\centering\arraybackslash}
	\includegraphics[]{E4.pdf}
	& 
	$\begin{aligned}[t] % placement: default is "center", options are "top" and "bottom"
	\vect{t}_{2i} 	&= \frac{\vect{t}_{2i}^- + \vect{t}_{2i}^+ }{\norm{\vect{t}_{2i}^- + \vect{t}_{2i}^+}}
	\end{aligned}$
\end{tabularx}

This way, the determination of the tangent vectors -- or equivalently the section normals -- in the static equilibrium configuration will be done in the flow of the dynamic relaxation process, without the need of introducing any additional degrees of freedom (for instance the usual Euler angles). The position of the vertices rules the orientation of the section normals.

% -------------------------------------------------------------------------------------
\subsubsection{Left/right bending moments at handle vertices}
% -------------------------------------------------------------------------------------

Given the unit tangent vector $\vect{t}_{2i}$, one can define the left ($\vect{\kappa}_{2i}^-$) and right ($\vect{\kappa}_{2i}^+$) curvatures at handle vertex $\vect{x}_{2i}$. The left curvature is initially evaluated from the left osculating circle, defined as the circle passing through $\vect{x}_{2i-1}$ and $\vect{x}_{2i}$ and tangent to $\vect{t}_{2i}$ at $\vect{x}_{2i}$. The right curvature is initially evaluated from the right osculating circle, defined as the circle passing through $\vect{x}_{2i}$ and $\vect{x}_{2i+1}$ and tangent to $\vect{t}_{2i}$ at $\vect{x}_{2i}$.\footnote{Remark that the centerline is now approximated with a biarc in the vicinity of $\vect{x}_{2i}$. This is the reason why this model is called the \textquote{biarc model}.}${}^,$\footnote{This model offers the ability to represent discontinuities in curvature -- thus in bending moment -- at handle vertices as the left and right curvatures does not necessarily agree. This is quite different from the classical 3-dof element \cite{Barnes1999, Adriaenssens1999, Douthe2006} which assumes that the curvature -- thus the bending moment -- is $\mathcal{C}^0$ and can be evaluated at every vertices from the circumscribed osculating circle.}

\begin{tabularx}{\textwidth}[t]{>{\centering\arraybackslash}m{0.48\textwidth} >{\centering\arraybackslash}X} % >{\centering\arraybackslash}
	\includegraphics[]{E5.pdf}
	& 
	$\begin{aligned}[t] % placement: default is "center", options are "top" and "bottom"
	\vect{\kappa b}_{2i}^- &=  \frac{2}{l_{2i-1}} \vect{u}_{2i-1} \times \vect{t}_{2i}
	\\[0.5em]
	\vect{\kappa b}_{2i}^+ &=  \frac{2}{l_{2i}} \vect{t}_{2i} \times \vect{u}_{2i}
	\end{aligned}$
\end{tabularx}

However, this values need to be adjusted so that the static condition for rotational equilibrium ($\vect{M}^{ext}  + \vect{M}^+ - \vect{M}^- = 0$) is satisfied at all time. Then, this condition will be satisfied -- in particular -- at the end of the solving process. To achieve this goal, we first compute a realistic mean value ($\vect{M}_{2i}$) for the internal bending moment as :
\begin{equation}
		\vect{M}_{2i} 	=  \frac{1}{2} \mat{B}_{i-1}(\vect{\kappa b}_{2i}^- - \overbar{\vect{\kappa b}}_{2i}^-)
					+  \frac{1}{2} \mat{B}_{i}(\vect{\kappa b}_{2i}^+ - \overbar{\vect{\kappa b}}_{2i}^+)
\end{equation}
To enforce the jump discontinuity in bending moment ($\vect{M}^{ext} = \vect{M}^- - \vect{M}^+$) across the handle vertex, we define the left and right bending moments at $\vect{x}_{2i}$ as :
\begin{subequations}
	\begin{align}
		\vect{M}_{2i}^- 	&=  \vect{M}_{2i} + \frac{1}{2} \vect{M}_{2i}^{ext} 
		\\[0.5em]
		\vect{M}_{2i}^+ 	&=  \vect{M}_{2i} - \frac{1}{2} \vect{M}_{2i}^{ext} 
	\end{align}
\end{subequations}
Note that in the case where no external concentrated bending moment is applied to the handle vertex, the internal bending moment is continuous across the vertex.

% -------------------------------------------------------------------------------------
\subsubsection{Left/right curvatures at handle vertices}
% -------------------------------------------------------------------------------------

Finally, the left and right curvatures at handle vertex $\vect{x}_{2i}$ are computed back with the constitutive law :
\begin{subequations}
	\begin{align}
		\vect{\kappa b}_{2i}^-  &=  \mat{B}_{i-1}^{\;-1} \vect{M}_{2i}^- + \overbar{\vect{\kappa b}}_{2i}^- 
		\\[0.5em]
		\vect{\kappa b}_{2i}^+  &=  \mat{B}_{i}^{\;-1} \vect{M}_{2i}^+ + \overbar{\vect{\kappa b}}_{2i}^+ 
	\end{align}
\end{subequations}

% -------------------------------------------------------------------------------------
\subsubsection{Bending moment at ghost vertices}
% -------------------------------------------------------------------------------------

The internal bending moment at ghost vertices is simply given by the constitutive law as :
\begin{subequations}
	\begin{align}
		\vect{M}_{2i-1} &=  \mat{B}_{i-1}(\vect{\kappa b}_{2i-1} - \overbar{\vect{\kappa b}}_{2i-1})
		\\[0.5em]
		\vect{M}_{2i+1} &=  \mat{B}_{i}(\vect{\kappa b}_{2i+1} - \overbar{\vect{\kappa b}}_{2i+1})
	\end{align}
\end{subequations}

\subsection{Discret twisting moment}
% ==========================

We assume the twisting moment and the rate of twist to vary linearly over $]\vect{x}_{2i},  \vect{x}_{2i+2}[$.
Thus, the rate of twist at mid edge is given by :
\begin{equation}
	\tau_{i+1/2} = \frac{\Delta\theta_{i}}{l_i}
\end{equation}
And $\theta_{i+1} - \theta_{i}$ is the additional twisting angle between two frames with parallel transport.
\begin{equation}
	Q_{i+1/2} =  GJ(\tau_{i+1/2} - \overbar{\tau}_{i+1/2})
\end{equation}

\subsection{Discret axial force}

We assume the axial force and the axial strain to vary linearly over $]\vect{x}_{2i},  \vect{x}_{2i+2}[$.
Thus, the axial strain at mid edge is given by :
\begin{equation}
	\epsilon_{i+1/2} = \frac{l_{i}}{\overbar{l_i}} - 1
\end{equation}
\begin{equation}
	N_{i+1/2} =  ES\epsilon_{i+1/2}
\end{equation}

\subsection{Discret shear force}

Shear forces are computed from the second Kirchhoff law, considering that the inertial term is negligible.

\begin{equation}
	 \vect{F}_{i+1/2} = \vect{d}_{3, i+1/2} \times (\vect{M}_{i+1/2}^{'} + \vect{m}_{ext, i}) + Q_{i+1/2} \vect{\kappa b}_{i+1/2} - \tau_{i+1/2} \vect{M}_{i+1/2}
\end{equation}

\subsection{Interpolation}


\section{Conclusion}
Remind that the beam is subject to a distributed external force $\vect{f}_{ext}$ and a distributed external moment $\vect{m}_{ext}$.

We neglect rotational inertial effects on $\vect{d}_1$ et $\vect{d}_2$ in \eqref{eq:3_16_b} and \eqref{eq:3_16_c} which leads to the following shear force :
\begin{equation}
	\vect{F}^{\perp}(s) = \vect{d}_3 \times (\vect{M}' + \vect{\Omega} \times \vect{M} + \vect{m}_{ext})
\end{equation}
\begin{equation}
	\vect{F}^{\parallel}(s) = N \vect{d}_3
\end{equation}
\note{We may neglect as well the last term ($\tau \vect{M}$) and get back to the shear force obtained by the variational approach.} The total internal force acting on the beam is hence given by :
\begin{equation}
	\vect{F}(s) = \vect{N}(s) + \vect{T}(s)
\end{equation}
Sections are subject to the following rotational moment around the centerline :
\begin{equation}
	\begin{aligned}
	\vect{\Gamma}(s) = Q' + \vect{d}_3 \cdot ( \kappa\vect{b} \times \mat{M} + \vect{m}_{ext})
	\end{aligned}
\end{equation}

\clearpage



\clearpage
\bibliographystyle{alpha}
\bibliography{../library}
