% Author :  Lionel du Peloux
% Contact : lionel.dupeloux@gmail
% Year : 2017

% ===========================
% COMPILER DIRECTIVES
% ===========================
% !TEX encoding = UTF-8 Unicode
% !BIB TS-program = biber
% !BIB program = biber

\newrefsegment
\chapter*{Conclusion}\label{chp=conclu}
\addcontentsline{toc}{chapter}{\textbf{Conclusion}}
\markboth{Conclusion}{Conclusion} % necessary to have the correct header with \chapter* and book.cls

\begin{otherlanguage}{french}

Ce travail de thèse s'est intéressé aux modèles de calcul dédiés aux structures précontraintes par flexion. Il s'est inscrit dans un projet de recherche plus large sur les structures de type \emph{gridshell élastique}, développé par le laboratoire Navier. Initié au début des années 2000 par J.-F. Caron et O. Baverel, ce dernier entend revisiter le travail de l'ingénieur et architecte allemand Frei Otto sous le double aspect de la structure et des matériaux composites. J'ai rejoins ce projet en mai 2010 en qualité d'ingénieur de recherche, puis en tant que doctorant à partir d'octobre 2014. Sur ces presque 8 années de collaboration j'ai eu la chance de pouvoir non seulement développer une recherche personnelle sur cette thématique, mais également de pouvoir confronter le fruit de cette recherche à la réalité en concevant et construisant un certain nombre de gridshells en matériau composite ou en bois. Et c'est probablement ce qui caractérise le mieux la spécificité de mon travail : cette confrontation répétée entre théorie et pratique.

Construire courbe se révèle complexe à tous les niveaux et les gridshells n'échappent pas à cette règle. En effet, la définition géométrique de l'ouvrage en constitue la pierre angulaire et, à ce titre, en assure aussi bien l'identité architecturale que la faisabilité sur le plan structurel. Structure et Architecture s'en trouvent ainsi associées de manière symbiotique. Et c'est dans ce lien étroit que se noue leur complexité intrinsèque.

\subsection*{Revue}

Dans la première partie de notre travail, nous avons souhaité nous immerger en profondeur et par l'expérience dans la complexité de ces structures. Nous avons commencé notre étude (voir \cref{chp=gridshell}) par effectuer une revue critique des projets de gridshell élastique réalisés depuis les années 1960 jusqu'à nos jours. Cette brève histoire dessine à elle seule le potentiel de ces structures, notamment en terme d'expression formelle et de performance structurelle. Loin de les enfermer dans un style d'architecture particulier, elle en souligne au contraire la grande variété. Nous avons complété cette revue de projet par une revue de littérature approfondie sur l'ensemble des domaines de recherche connexes à cette thématique (géométrie, structure, matériaux, logiciel).

\subsection*{Expérimentation et maquette numérique}

Nous avons ensuite présenté la plus importante de nos réalisations, la conception et la construction de la cathédrale éphémère de Créteil, premier véritable bâtiment réalisé à ce jour sur le principe du gridshell élastique en matériau composite (voir \cref{chp=creteil}). Construit en 2013, il est toujours en service. A cette occasion, nous avons mis au point une méthode, des outils et des critères d'évaluation pour permettre à des concepteurs -- architectes et ingénieurs -- de répondre de façon maîtrisée à un projet de gridshell \cite{DuPeloux2016}. Cette méthode s'appuie sur la réalisation d'une maquette numérique interactive qui associe des fonctions de modelage 3D basées sur une représentation NURBS des surfaces, des fonctions de maillage par la méthode du compas, et des fonctions de recherche de forme grâce à un code de calcul non linéaire basé sur la méthode de la relaxation dynamique. Elle a la particularité de recentrer le processus de conception sur la définition d'une forme et redonne ainsi de la place à l'expression de l'intention architecturale, là où la complexité des techniques de recherche de forme (sur modèle physique ou numérique) l'en avait privée. Nous avons montré comment cette liberté « retrouvée » a effectivement servi l'architecture du projet pour créer un espace qui fasse sens vis-à-vis de sa destination (un lieu de culte) et qui ne soit pas le produit de contraintes purement techniques. Ce travail, publié en 2016, s'est récemment vu distingué par l'International Association for Bridge and Structural Engine (IABSE).\footnote{IABSE Awards 2017, Outstanding Paper Award, Technical Report.}

\subsection*{Pertinence des outils}

Les outils que nous avons mis au point à l'occasion de ce projet ont pallié à l'inadéquation des outils de design existants, qui sont davantage orientés vers la justification des ouvrages que vers leur conception. Ils nous ont permis d'appréhender la problématique de l'interaction forme-maillage-structure avec beaucoup plus d'agilité que si nous avions eu recours aux seuls outils disponibles dans le commerce. Ils ont rendu possible le développement de ce projet de gridshell dans des contraintes de planning et de coût sévères, à l'opposé des moyens engagés pour la multihalle de Mannheim en 1975. Cependant, cette méthode a également montré un certain nombre de limites qui ont restreint notre capacité à développer une représentation riche et fonctionnelle du projet sous la forme d'une maquette numérique.

\subsection*{Limitation des outils}

Sur le plan de la fonctionnalité de la représentation, il faut bien reconnaître que la maquette actuelle ne permet ni le niveau d'interactivité ni le niveau de réactivité qu'offrirait une simple maquette physique manipulable \emph{à la main}. Bien que cet aspect n'ai pas constitué l'enjeu principal de notre travail, nous avons porté une grande attention à cette question dans le développement de nos outils, en essayant d'optimiser l'intégration des fonctions et la rapidité du code de calcul pour fournir l'expérience utilisateur la plus fluide et intuitive possible. C'est pour ces mêmes raisons que nous avons choisi d'implémenter nos outils dans le framework \rhino{} \& \grasshopper{}.\footnote{J'ai commencé à développer ces outils sous la forme de scripts python pour \rhino{} à l'occasion du projet Solidays. J'ai progressivement migré ces outils vers \Csharp{} et développé des bibliothèques de composants \grasshopper{}. Aujourd'hui, cette maquette est concrètement contrôlée par un canevas \grasshopper{}.} Pour aller plus loin sur les questions d'interactivité on pourrait explorer le champ de la réalité virtuelle et augmentée pour s'affranchir des limitations inhérentes à l'utilisation d'une souris, d'un clavier et d'un écran pour accéder à la maquette. Pour aller plus loin sur les questions de réactivité on pourrait explorer la piste du calcul parallèle (SIMD, CPU, GPU, \telp{}) pour accélérer les codes de maillage et de recherche de forme ; on pourrait également explorer d'autres méthodes de résolution numériques potentiellement plus rapides que la relaxation dynamique ; ou bien on pourrait encore implémenter des fonctionnalités de raffinement automatique de grille pour travailler avec les modèles les plus légers possibles en terme de degrés de liberté.

Sur le plan de la richesse de la représentation, le code de calcul structurel utilisé reposait sur un élément de poutre discret à seulement trois degrés de liberté \cite{Adriaenssens2000}. De ce fait, il ne permettait pas la modélisation des phénomènes de torsion et de couplage flexion-torsion dans les éléments structuraux. Bien que ces phénomènes puissent être négligés en première approximation dans le cas de grilles constituées de poutres de section circulaire et rectilignes dans leur configuration naturelle, ces phénomènes peuvent cependant se révéler critiques pour des matériaux fortement anisotropes comme le bois et ou les composites pultrudés, qui en effet résistent mal à des sollicitations de torsion. Par ailleurs, lorsque la section des poutres employées est anisotrope -- comme c'est souvent le cas pour les gridshells en bois -- ces phénomènes influent fortement sur la forme d'équilibre de la grille et sur le niveau de contrainte observé dans la structure, les poutres pouvant se retrouver soumises à d'importantes courbures selon leur axe fort d'inertie. En outre, l'élément discret à 3 degrés de liberté ne peut représenter la notion de moment que sous la forme d'un couple d'effort. Il reste donc très limité pour modéliser les conditions cinématiques parfois complexes des connexions ou des conditions d'appui, notamment lorsqu'un transfert de moment s'opère (e.g. au niveau d'un encastrement).

\subsection*{Noveaux modèles de poutre}

Dans la seconde partie de notre travail (voir \cref{part=1}), nous avons donc cherché à dépasser les limitations du modèle de calcul employé pour le projet de la cathédrale éphémère de Créteil. L'objectif poursuivi était de renforcer la précision et la complétude des informations mécaniques retournées par la maquette aux concepteurs, sans pour autant sacrifier le niveau d'interactivité et de réactivité précédemment atteint et qui faisait justement la pertinence de cet outil.

Dans une première tentative (voir \cref{chp=energy}), à partir de travaux récents sur les tiges élastiques appliqués au champ des computer graphics \cite{Bergou2008}, et dans la continuité d'un précédent travail de thèse auquel nous avons collaboré \cite{Tayeb2015a}, nous avons, par une approche variationnelle, formulé un élément de poutre discret qui puisse rendre compte des phénomènes de torsion \cite{Lefevre2017}. La description cinématique de l'élément repose ici sur la définition d'une ligne moyenne comprise comme une courbe paramétrique de l'espace~; et d'une section droite positionnée à l'aide d'un repère mobile adapté à cette courbe, lui-même entièrement déterminé, à une constante près, par une unique variable scalaire. Ainsi, cet élément possède un nombre minimal de degrés de liberté, à savoir 4. Cependant, ce nouveau modèle ne répond pas à l'ensemble des limitations identifiées précédemment. En particulier, il ne permet pas de représenter certaines discontinuités qui apparaissent là où les actions mécaniques s'exercent de manière concentrée, comme par exemple au niveau d'un appui, d'une connexion ou bien d'une charge ponctuelle. Cette capacité est pourtant primordiale pour l'étude des détails de la structure, qui sont des points clefs du système constructif comme nous l'avons montré dans notre présentation de la cathédrale éphémère.

Dans une seconde tentative (voir \cref{chp=kirchhoff}), nous avons donc cherché à combler ces lacunes et à pouvoir rendre compte des discontinuités qui découlent des actions concentrées en sus des phénomènes de torsion. Nous avons commencé par montrer comment, à partir des équations dynamiques de Kirchhoff, nous pouvions formuler de manière relativement directe un élément de poutre à 4 degrés de liberté. Cette approche est apparue plus évidente que la première. Par ailleurs, on a montré qu'elle traitait naturellement la question des actions extérieures et s'insérait parfaitement dans le cadre conceptuel de la relaxation dynamique basé sur le principe fondamental de la dynamique. Puis nous avons développé une réflexion approfondie sur la notion de courbure discrète (voir \cref{chp=curve}) qui nous a permis d'identifier les mécanismes géométriques nécessaires à la modélisation des discontinuités de courbure (et donc de moment). En combinant ces résultats nous sommes parvenus à mettre au point un élément de poutre discret à 4 degrés de liberté et 3 noeuds (voir \cref{chp=numerical_model}), contre 2 pour les modèles précédents. Les propriétés de section et de matériau sont supposées uniformes sur la longueur de l'élément. Il rend compte du comportement axial, de flexion et de torsion de la poutre, dans le cadre de la théorie de Kirchhoff, pour des sections dont le centre de torsion est confondu avec le centre de masse. Il peut subir des actions concentrées en ses extrémités et des actions distribuées uniformes en partie courante. Les efforts internes sont donc continus sur la longueur de l'élément mais peuvent subir des sauts au niveau des ses extrémités. Nous avons également présenté la démarche à suivre pour implémenter des conditions d'appui de type libre, rigide ou élastique.

\subsection*{Développement d'un code de calcul}

Finalement, nous avons présenté succinctement \emph{Marsupilami}, le code de calcul que nous avons mis au point et qui implémente ce nouvel élément. Il se matérialise sous la forme d'une API \Csharp{} libre de toutes dépendances. Cette API a été partiellement implémentée dans une bibliothèque de composants \grasshopper{} pour servir d'interface graphique. De nombreuses pistes ont été explorées concernant l'architecture du code pour le doter de nouvelles possibilités, notamment grâce à l'usage des évènements (raffinement automatique de maillage, force suiveuse, parallélisation des calculs, interaction utilisateur, \telp{}). Le code essaie de tirer le meilleur parti des abstractions proposées par le langage \Csharp{} pour marier différents types d'éléments, de conditions d'appui et même de noeuds selon leur nombre de degrés de liberté (3, 4 ou 6). Nous avons pu valider la précision de notre nouvel élément en comparant les résultats de \emph{Marsupilami} avec ceux du logiciel \emph{Abaqus}, référence en la matière, sur un certain nombre de cas tests. Réalisés sur des poutres seuls, ce travail de validation demande a être poursuivi sur des structures complètes.\footnote{Par exemple un recalcul de la cathédrale de Créteil pourrait peut-être permettre de comprendre certaines des ruptures observées 6 mois après le montage.} Cependant, \emph{Marsupilami} n'a pour l'instant rien d'un véritable logiciel que l'on pourrait utiliser dans un contexte de production. Dans son état actuel il s'agit plus d'une \emph{preuve de concept}, qui mériterait un effort de développement conséquent pour établir une première version stable transférable à d'autres utilisateurs.

\subsection*{Perspectives}
Les modèles, les outils et les méthodes développés au cours de cette recherche ont rendu possible la conception et la réalisation d'un certain nombre de prototypes à une échelle parfois importante, comme ce fût le cas des gridshells de Solidays en 2011 et de Créteil en 2013. L'expérience acquise sur ces projets a mis en valeur la nécessité de disposer d'outils de conception agiles pour aborder l'interaction forme-maillage-structure. Elle a aussi souligné les éléments qui mériteraient d'être approfondis~: 

% \begin{itemize}
% \item 
    \textbf{\emph{Marsupilami}}. Le code actuel pourrait devenir une API \Csharp{} fort pratique pour le calcul des gridshells moyennant un effort de développement conséquent. Le travail pourrait consister à consolider et étendre l'API actuelle pour la rendre stable et facilement extensible~; ainsi qu'à développer une interface interactive pour la plateforme \rhino{} \& \grasshopper{}. Ce travail devrait garder comme objectif la capacité à générer des maquettes numériques de conception qui soient les plus agiles possible. L'API elle même pourrait potentiellement faire l'objet d'un développement collaboratif, pourquoi pas en partenariat avec d'autres laboratoires, ce qui permettrait de pourvoir aux compétences nécessaires à un tel projet. En ce sens, une licence de type \emph{Open Source} pourrait permettre une meilleur diffusion du code et donc de toucher de potentiels contributeurs.
    
    % \item 
    \textbf{Système Constructif}. La noix de connexion et le manchon constituent deux détails clefs du système constructif actuel. Ces pièces pourraient faire l'objet de nombreuses améliorations pour en augmenter la légèreté, ou bien pour les rendre plus fonctionnelles afin de faciliter l'assemblage de la grille par les opérateurs. Il en va de même pour le dispositif de contreventement qui impact grandement les coûts de construction comme nous l'avons montré. Dans ce sens nous avons pu tester en 2016, à petite échelle sur trois gridshells en bois d'environ \SI{50}{m^2}, un système à câble installé sur la grille au sol et activé une fois la déformation terminée. Cela a permis de réduire le temps de travail en hauteur de façon significative et a nécessité des développement nouveaux pour les pièces de connexion du système de triangulation.

    % \item 
    \textbf{Enveloppe}. L'enveloppe des ces structures reste un champ difficile à maîtriser du fait de la courbure géométrique. Les membranes employées jusqu'ici ne garantissent aucunes performances acoustiques ou thermiques sérieuses. On pourrait contourner ce problème en identifiant les les applications potentielles où ces critères ne sont pas rédhibitoires, comme par exemple pour certaines structures à usage temporaire ou bien pour des couvertures d'espaces industriels qui ne doivent assurer aucune autre fonction que celle de l'étanchéité. Cette question peut être abordée de manière plus globale avec celle de la structure et du contreventement. Nous avons eu l'occasion de développer une réflexion originale sur le sujet, en mettant au point un concept de structure hybride dans lequel l'enveloppe assure à la fois le clos-couvert du bâtiment et le contreventement de la résille en matériau composite \cite{Cuvilliers2017}. L'idée principale est d'utiliser le gridshell comme cintre pour couler une fine enveloppe en béton fibré par dessus. Une connexion mécanique est assurée entre la résille et le béton pour permettre à l'enveloppe de jouer le rôle de contreventement d'une part~; et minimiser l'épaisseur de béton nécéssaire d'autre part.

    \textbf{Autres}. On pourra aussi considérer plus largement les applications potentielles du présent travail, et principalement de l'élément de poutre mis au point, au domaine de l'\emph{active-bending}. En s'intéressant par exemple au problème de positionnement des gaines sur les bras robotisés, un outil autrefois réservé aux grands industriels et en passe de se démocratiser, qui peuvent venir en contact des outils ou géner le mobilité du bras et dont les mouvements sont difficiles à prévoir (à cause du couplage flexion torsion). Pour rester dans le secteur de la construction, on pourra également regarder du côté des \emph{coumpound mechanisum} appliqués aux \emph{shading device} \cite{Charpentier2017}.

% \end{itemize}





\blockcquote[Frei Otto][p.~14]{IL13}{Even if all the participants are happy today that this building was completed on time, we should not forget to temper our enthusiasm with criticism. If one of the participants had the oppotunity to solve a similar problem in future, he would wish tio change many things. Now we shall watch this building very carefully. Already we have seen much, but we shall still see more. Whether, however, this valuable experiences will further influence any other building in a positive sens remains only a hope}


\end{otherlanguage}
