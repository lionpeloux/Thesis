% Author :  Lionel du Peloux
% Contact : lionel.dupeloux@gmail
% Year : 2017

\chapter*{Conclusion}\label{chp:conclu}
\addcontentsline{toc}{chapter}{\textbf{Conclusion}}


La paternité des structures de type \emph{gridshell élastique} est couramment attribuée à l'architecte et ingénieur allemand Frei Otto, qui les a intensivement étudiées au XX\textsuperscript{ème} siècle. Fruit de son travail de recherche, il réalise en 1975, en collaboration avec l'ingénieur Edmund Happold du bureau Arup, un projet expérimental de grande ampleur~: la multi-halle de Mannheim. Cette réalisation emblématique ancrera durablement les gridshells dans le paysage des typologies structurelles candidates à l'avènement de géométries non-standard, caractérisées par l'absence d'orthogonalité. Cette capacité à \emph{former la forme} de façon efficiente prend tout son sens dans le contexte actuel où, d'une part la forme s'impose comme une composante prédominante de l'architecture moderne (F. Gehry, Z. Hadid,~\telp{}) et d'autre part l'enveloppe s'affirme comme le lieu névralgique de la performance des bâtiments, notamment environnementale.



