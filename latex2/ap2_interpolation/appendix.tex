\chapter{Parabolic interpolation}
\label{ch:interpolation}

% ----------------------------------------------
% Lagrange interpolating polynomial
% -----------------------------------------------
\section{Introduction}
In this appendix, we give the required formulas to conduct a parabolic interpolation of a scalar or vector-valued function over an interval.

We look for a polynomial interpolation of order 2 of a continuous scalar or vector-valued function $\vect{V}\colon t \mapsto \vect{V}(t)$ over the interval $[t_0,t_2]$ ; supposing that the value of the function is known for three distinct parameters $t_0 < t_1 < t_2$~:
\begin{subequations}
	\begin{alignat}{1}
	\vect{V}(t_0) = \vect{V}_0 \\
	\vect{V}(t_1) = \vect{V}_1 \\
	\vect{V}(t_2) = \vect{V}_2
	\end{alignat}
\end{subequations}
This interpolation method is employed several times in this thesis, for instance to evaluate the position of a kinetic energy peak during the dynamic relaxation process. It is also employed for evaluating the bending moment and the curvature of a discrete rod at mid-edge, knowing its values at vertices.

Note that this interpolation method is valid if the basis in which $\vect{V}$ is decomposed does not depend on the parameter $t$. Otherwise, the classical transportation term should be considered ($\vect{\omega} \times \vect{V}$).

% ----------------------------------------------
% Lagrange interpolating polynomial
% ----------------------------------------------
\section{Lagrange interpolating polynomial}
The Lagrange interpolation of order two is given by the following polynomial~:
\begin{equation}
	\vect{V}(t) = 
	\vect{V}_0 \frac{(t-t_1)(t-t_2)}{(t_0-t_1)(t_0-t_2)}
	+ \vect{V}_1 \frac{(t-t_0)(t-t_2)}{(t_1-t_0)(t_1-t_2)}
	+ \vect{V}_2 \frac{(t-t_0)(t-t_1)}{(t_2-t_0)(t_2-t_1)}
	\label{eq:lagrange}
\end{equation}
% ----------------------------------------------
% Lagrange interpolating polynomial
% ----------------------------------------------
\section{Reparametrization}
Lets introduce the distances $l_0$ and $l_1$ in the parametric space~:
\begin{subequations}
	\begin{alignat}{1}
	l_0 &= t_1 - t_0 \\
	l_1 &= t_2 - t_1
	\end{alignat}
\end{subequations}
Lets introduce the change of variable $u = t - t_1$. The polynomial in \cref{eq:lagrange} can be rewritten in the form~:
\begin{equation}
	\vect{V}(u) = 
	\vect{V}_0 \frac{u(u-l_1)}{l_0(l_0+l_1)}
	- \vect{V}_1 \frac{(u+l_0)(u-l_1)}{l_0 l_1}
	+ \vect{V}_2 \frac{u(u+l_0)}{l_1(l_0+l_1)}
	\label{eq:poly}
\end{equation}
where :
\begin{subequations}
	\begin{alignat}{1}
	u_0 &= -l_0 \\
	u_1 &= 0 \\
	u_2 &= l_1
	\end{alignat}
\end{subequations}
The derivative of this polynomial is also required to determine the extremum value of $\vect{V}$. Differentiating \cref{eq:poly} gives~:
\begin{equation}
	\vect{V}'(u) = 
	\vect{V}_0 \frac{2u - l_1}{l_0(l_0+l_1)}
	- \vect{V}_1 \frac{2u + (l_0 - l_1)}{l_0 l_1}
	+ \vect{V}_2 \frac{2u + l_0}{l_1(l_0+l_1)}
	\label{eq:dpoly}
\end{equation}
This expression can be factorized to give the more compact form :
\begin{equation}
	\vect{V}'(u) = 
	\left(\frac{\vect{V}_1 - \vect{V}_0}{l_0}\right) \frac{l_1 - 2u}{l_0+l_1}
	+
	\left(\frac{\vect{V}_2 - \vect{V}_1}{l_1}\right)  \frac{l_0 + 2u}{l_0+l_1}
	\label{eq:dpoly2}
\end{equation}
% ----------------------------------------------
% Characteristic values
% ----------------------------------------------
\section{Characteristic values}
Using \cref{eq:poly} the interpolated values of $\vect{V}$ at mid distance between $t_0$ and $t_1$ ($u = -l_0/2$), and at mid distance between $t_1$ and $t_2$ ($u =+l_1/2$) are given by~:
\begin{subequations}
	\begin{alignat}{7}
	\vect{V}_{01}
	&= 
	&&\vect{V}_0 \frac{l_0 + 2 l_1}{4(l_0+l_1)}
	&\,+\,&\vect{V}_1 \frac{l_0+2 l_1}{4 l_1}
	&\,-\,&\vect{V}_2 \frac{{l_0}^2}{4 l_1 (l_0+l_1)}
	\label{eq:V01_a}
	\\[0.5em]
	\vect{V}_{12}
	&= 
	- &&\vect{V}_0 \frac{{l_1}^2}{4 l_0 (l_0+l_1)}
	&\,+\,&\vect{V}_1 \frac{2 l_0 + l_1}{4 l_0}
	&\,+\,&\vect{V}_2 \frac{2 l_0 + l_1}{4 (l_0+l_1)}
	\label{eq:V01_b}
	\end{alignat}
\end{subequations}
Using \cref{eq:dpoly2} the interpolated values of $\vect{V}'$ at mid distance between $t_0$ and $t_1$ ($u = -l_0/2$), and at mid distance between $t_1$ and $t_2$ ($u =+l_1/2$) are given by~:
\begin{subequations}
	\begin{alignat}{5}
	\vect{V}'_{01}
	&= \frac{\vect{V}_1 - \vect{V}_0}{l_0}
	\\[0.5em]
	\vect{V}'_{12}
	&= \frac{\vect{V}_2 - \vect{V}_1}{l_1}
	\end{alignat}
\end{subequations}
Remark that this is an interesting result as at these parameters the evaluation of $\vect{V}'$ boils down to a finite difference scheme.

Using \cref{eq:dpoly2} and introducing $\alpha = \frac{l_0}{l_0+l_1}$ the interpolated values of $\vect{V}'$ at $t_0$, $t_1$ and $t_2$ are given by :
\begin{subequations}
	\begin{alignat}{5}
	\vect{V}'_0 &= (1+\alpha) \vect{V}'_{01} -  \alpha \vect{V}'_{12}
	 \\
	\vect{V}'_1 &= (1-\alpha) \vect{V}'_{01} +  \alpha \vect{V}'_{12}
	\\
	\vect{V}'_2 &= (\alpha-1) \vect{V}'_{01} +  (2 - \alpha) \vect{V}'_{12}
	\end{alignat}
\end{subequations}
Lets rewrite \cref{eq:V01_a,eq:V01_b} with the help of $\alpha$ :
\begin{subequations}
	\begin{alignat}{5}
	\vect{V}_{01} &= \frac{1}{4}
	\left(
	(2-\alpha)\vect{V}_0
	+ \frac{2 - \alpha}{1 - \alpha} \vect{V}_1
	- \frac{\alpha^2}{1 - \alpha} \vect{V}_2
	\right)
	\\
	\vect{V}_{01} &= \frac{1}{4}
	\left(
	- \frac{(1 - \alpha)^2}{\alpha} \vect{V}_0
	+ \frac{1 + \alpha}{\alpha} \vect{V}_1
	+ (1 + \alpha) \vect{V}_2
	\right)
	\end{alignat}
\end{subequations}

\section{Extremum value}
The extremum value of the parabola is obtained for $\vect{V}'(u^*) = 0$. It's a minimum if $\vect{V}'_{12} > \vect{V}'_{01}$ and it's a maximum if $\vect{V}'_{12} > \vect{V}'_{01}$~:
\begin{equation}
	u^* = \frac{l_1 \vect{V}'_{01} + l_0 \vect{V}'_{12}}{2(\vect{V}'_{01}-\vect{V}'_{12})}
	\label{eq:u}
\end{equation}
Remark that if  $\vect{V}'_{12} = \vect{V}'_{01}$ it does not make sens to compute $u^*$ as in this case the parabola degenerates into a line. The value of the function at this parameter is given by~:
\begin{equation}
	\vect{V}(u^*) = \vect{V}_1 +  \frac{(l_1 \vect{V}'_{01} + l_0 \vect{V}'_{12})^2}{4(l_0 + l_1)(\vect{V}'_{01}-\vect{V}'_{12})}
	\label{eq:Vu}
\end{equation}
The parabola in \cref{eq:poly} now writes~:
\begin{equation}
	\vect{V}(u) = - \frac{\vect{V}'_{01}-\vect{V}'_{12}}{l_0 + l_1} (u - u^*)^2 + \vect{V}(u^*)
\end{equation}
The extremum is located in $[t_0,t_2]$ if the sign of $\vect{V}'$ changes on this interval. This condition is satisfied whenever $\vect{V}'_{01} \cdot \vect{V}'_{12} < 0$. 

Finally, in the special case of a uniform discretization where $l_0 = l_1 = l$, \cref{eq:u,eq:Vu} become~:
\begin{subequations}
	\begin{alignat}{5}
		u^* &= \frac{l}{2}\left( \frac{\vect{V}_0 - \vect{V}_2}{\vect{V}_0 - 2\vect{V}_1 + \vect{V}_2} \right)
		\\[0.5em]
		\vect{V}(u^*) &= \vect{V}_1 - \frac{u^*}{4l} (\vect{V}_2 - \vect{V}_0)
	\end{alignat}
\end{subequations}

\bibliographystyle{alpha}
\bibliography{../library}
