%\begingroup
%\let\cleardoublepage\clearpage

% English abstract
\cleardoublepage
\chapter*{Abstract}
%\markboth{Abstract}{Abstract}
\addcontentsline{toc}{chapter}{Abstract (English/Français)} % adds an entry to the table of contents
% put your text here
Summary to be provided
\vskip0.5cm
Key words~: elastic gridshell, torsion, bending active, composite material
%put your text here


The last decades have seen the emergence of non-standard architectural shapes. Designers find often themselves helpless with the geometrical complexity of these objects. Furthermore, the available tools dissociate shape and structural behaviour, which adds another complication. This dissertation takes the point of view based on invariance under geometrical transformations, and studies several strategies for fabrication-aware shape modelling. Three technological constraints have been identified and correspond to three independent contributions of this thesis. The repetition of nodes is studied via transformations by parallelism. They are used to generalise
surfaces of revolution. A special parametrisation of moulding surfaces is found with this method. The resulting structure has a high node congruence. Cyclidic nets are then used to model shapes parametrised by their lines of curvature. This guarantees
meshing by planar panels and torsion-free beam layout. The contribution of this dissertation is the implementation of several improvements, like doubly-curved creases, a hole-filling strategy that allows the extension of cyclidic nets to complex topologies, and the generation of a generalisation of canal surfaces from two rail curves and one profile curves. Finally, an innovative method inspired by descriptive geometry is proposed to generate doubly-curved
shapes covered with planar facets. The method, called marionette technique, reduces the problem to a linear problem, which can be solved in real-time. A comparative study shows that this technique can be used to parametrise shape optimisation of shell structures without loss of performance compared to usual modelling technique. The handling of fabrication constraints in shape optimisation opens new possibilities for its practical application, like gridshells or plated shell structures. The relevance of those solutions is demonstrated through multiple case-studies.
Keywords : structural morphology, generative tool, design process, descriptive geometry, structural optimisation, gridshell, marionette mesh, thin shell.
5

% French abstract
%\begin{otherlanguage}{french}
%\cleardoublepage
%\chapter*{Résumé}
%%\markboth{Résumé}{Résumé}
%% put your text here
%\lipsum[1-2]
%\vskip0.5cm
%Mots clefs: 
%%put your text here
%\end{otherlanguage}

%% German abstract
%\begin{otherlanguage}{german}
%\cleardoublepage
%\chapter*{Zusammenfassung}
%%\markboth{Zusammenfassung}{Zusammenfassung}
%% put your text here
%\lipsum[1-2]
%\vskip0.5cm
%Stichwörter: 
%%put your text here
%\end{otherlanguage}

%\endgroup			
%\vfill
