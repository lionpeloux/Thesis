% !TEX encoding = UTF-8 Unicode

% English abstract
\cleardoublepage
\chapter*{Abstract}
\addcontentsline{toc}{chapter}{Abstract (English/Français)}
% adds an entry to the table of contents
% put your text here Summary to be provided
The last decades have seen the emergence of non-standard architectural shapes. Designers find often themselves helpless with the geometrical complexity of these objects. Furthermore, the available tools dissociate shape and structural behaviour, which adds another complication. This dissertation takes the point of view based on invariance under geometrical transformations, and studies several strategies for fabrication-aware shape modelling. Three technological constraints have been identified and correspond to three independent contributions of this thesis. The repetition of nodes is studied via transformations by parallelism. They are used to generalise surfaces of revolution. A special parametrisation of moulding surfaces is found with this method. The resulting structure has a high node congruence. Cyclidic nets are then used to model shapes parametrised by their lines of curvature. This guarantees meshing by planar panels and torsion-free beam layout. The contribution of this dissertation is the implementation of several improvements, like doubly-curved creases, a hole-filling strategy that allows the extension of cyclidic nets to complex topologies, and the generation of a generalisation of canal surfaces from two rail curves and one profile curves. Finally, an innovative method inspired by descriptive geometry is proposed to generate doubly-curved shapes covered with planar facets. The method, called marionette technique, reduces the problem to a linear problem, which can be solved in real-time. A comparative study shows that this technique can be used to parametrise shape optimisation of shell structures without loss of performance compared to usual modelling technique. The handling of fabrication constraints in shape optimisation opens new possibilities for its practical application, like gridshells or plated shell structures. The relevance of those solutions is demonstrated through multiple case-studies.
\vspace{12pt}\\
\textbf{Keywords}~: structural morphology, generative tool, design process, descriptive geometry, structural optimisation, gridshell, marionette mesh, thin shell.

% French abstract
\begin{otherlanguage}{french}
\cleardoublepage
\chapter*{Résumé}
%Structures découvertes pleine de potentiel pour former la forme.
%Tombées dans l'oubli puis remise au goût du jour 25 years later.
%La thèse aborde sous un angle expérimentale méthodologique et aussi théorique.
Les structures de type \emph{gridshell élastique} permettent de réaliser des enveloppes courbes par la déformation réversible d'une grille structurelle initialement plane et régulière sans rigidité de cisaillement. Cette capacité à \guil{former la forme} de façon efficiente prend tout son sens dans le contexte actuel où, d’une part la forme s'impose comme une composante prédominante de l'architecture moderne et d’autre part l'enveloppe s'affirme comme le lieu névralgique de la performance des bâtiments.
\\
\\
Fruit des recherches de l'architecte et ingénieur allemand Frei Otto dans les années 50, elles ont été rendues populaires par la construction de la Multihalle de Mannheim en 1975. Cependant, en dépit de leur fort potentiel, très peu de projets de ce type ont vu le jour depuis cette réalisation emblématique. En effet, la conception d'un tel ouvrage se heurte à deux difficultés majeures. La première difficulté est d'ordre technologique et concerne la fonctionnalisation de la structure en tant que squelette à géométrie non-standard, c'est à dire caractérisée par l'absence d'orthogonalité, pour constituer une enveloppe capable de répondre à un panel toujours plus large de critères performantiels tels que l'étanchéité, l'isolation, l'accoustique, la transparence, \telp{} La seconde difficulté est d'ordre théorique et concerne le développement d'outils de conception adaptés à cette typologie structurelle d'un nouveau genre.
%
%
%
%est celle de la fonctionalité de l'enveloppe.
%
%
%Concevoir un gridshell élastique requière 
%
%
%
%la conception de ces structures se heurtent à deux difficultés majeures
%
%Deux difficultés majeures : une structure n'est pas une enveloppe. La géométrie non-standard, caractérisée par l'absence d'orthogonalité, rend très complexe et couteuse la réalisation des fonctions d'étanchéités et d'isolation demandées.
%
%Cette capacité 
%
%Mannheim.
%
%La paternité des gridshells est couramment attribuée à l'architecte et ingénieur allemand Frei Otto, qui les a intensivement étudiés au XXème siècle. En 1975 il réalise, en collaboration avec l'ingénieur Edmund Happold du bureau Arup, un projet expérimental de grande ampleur~: la multi-halle de Mannheim. Cette réalisation emblématique ancrera durablement les gridshells dans le paysage des typologies structurelles candidates à l’avènement de géométries non-standard, caractérisées par l'absence d'orthogonalité. Cette capacité à \guil{former la forme} de façon efficiente prend tout son sens dans le contexte actuel où, d’une part la forme s'impose comme une composante pré-dominante de l'architecture moderne (Gehry, Hadid, ...), et d’autre part l'enveloppe s'affirme comme le lieu névralgique de la performance environnementale des bâtiments.
%
%Dans ce manuscrit,
\vspace{12pt}\\
\textbf{Keywords}~: structural morphology, generative tool, design process, descriptive geometry, structural optimisation,
\end{otherlanguage}
