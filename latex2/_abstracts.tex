% !TEX encoding = UTF-8 Unicode

% English abstract
\cleardoublepage
\chapter*{Abstract}
\addcontentsline{toc}{chapter}{Abstract (English/Français)}
% adds an entry to the table of contents
% put your text here Summary to be provided
%\vspace{12pt}\\
\textbf{Keywords}~: gridshell, form-finding, active-bending, free-form, torsion, elastic rod.
% French abstract
\begin{otherlanguage}{french}
\cleardoublepage
\chapter*{Résumé}
%Structures découvertes pleine de potentiel pour former la forme.
%Tombées dans l'oubli puis remise au goût du jour 25 years later.
%La thèse aborde sous un angle expérimentale méthodologique et aussi théorique.
Les structures de type \emph{gridshell élastique} permettent de réaliser des enveloppes courbes par la déformation réversible d'une grille structurelle régulière initialement plane. Cette capacité à \guil{former la forme} de façon efficiente prend tout son sens dans le contexte actuel où, d’une part la forme s'impose comme une composante prédominante de l'architecture moderne, et d’autre part l'enveloppe s'affirme comme le lieu névralgique de la performance des bâtiments.
\\
\\
Fruit des recherches de l'architecte et ingénieur allemand Frei Otto dans les années 1960, elles ont été rendues populaires par la construction de la Multihalle de Mannheim en 1975. Cependant, en dépit de leur potentiel, très peu de projets de ce type ont vu le jour suite à cette réalisation emblématique qui en a pourtant démontré la faisabilité à grande échelle. Et pour cause, les moyens engagés à l'époque ne sauraient assurer la reproductibilité de cette expérience dans un contexte plus classique de projet, notamment sur le plan économique. Par ailleurs, les techniques et les méthodes développées alors sont pour la plus part tombées en désuétude ou reposent sur des disciplines scientifiques qui ont considérablement évoluées. Le cadre réglementaire a profondément muté, apportant une certaine rigidité vis-à-vis de la pénétration des innovations. Ainsi la conception des gridshells se pose-t-elle en des termes nouveaux aux architectes et ingénieurs actuels et se heurte à l'inadéquation des outils et méthodes existant.
\\
\\
Dans cette thèse, qui marque une étape importante dans une aventure de recherche personnelle initiée en 2010, nous tentons d'embrasser la question de la conception des gridshells élastiques dans toute sa complexité, en abordant aussi bien les aspects théoriques que techniques et constructifs. Dans une première partie, nous livrons une revue approfondie de cette thématique et nous présentons de façon détaillée l'une de nos principales réalisation, la cathédrale éphémère de Créteil, construite en 2013 et toujours en service. Dans une seconde partie, nous développons un élément de poutre discret original avec un nombre minimal de degrés de liberté adapté à la modélisation de la flexion et de la torsion dans les gridshells. Enrichi d'un noeud fantôme, il permet de modéliser plus finement les phénomènes physiques au niveau des connexions et des appuis. Son implémentation numérique est finalement présentée et validée sur quelques cas types.
%Cette thèse marque une étape importante dans cette aventure de recherche initiée en 2010 et qui m'aura conduit à concevoir et à construire plusieurs gridshells de taille importante aussi bien en matériau composite qu'en bois.
%
%
%%
%% Elle embrasse la question de la conception et de la réalisation 
%%
%%\\
%\\
%
%
%
%Les techniques et les méthodes mises au point à l'époques sont tombées en désuétudes et les En effet, la conception de ces ouvrages se posent en des termes nouveaux pour les architectes et ingénieurs actuels, 
%
%
%conséquence de l'évolution du cadre réglementaire, de la mutation des techniques de calcul,.
%
%
%
%est celle de la fonctionalité de l'enveloppe.
%
%
%Concevoir un gridshell élastique requière 
%
%
%
%la conception de ces structures se heurtent à deux difficultés majeures
%
%Deux difficultés majeures : une structure n'est pas une enveloppe. La géométrie non-standard, caractérisée par l'absence d'orthogonalité, rend très complexe et couteuse la réalisation des fonctions d'étanchéités et d'isolation demandées.
%
%Cette capacité 
%
%Mannheim.
%
%La paternité des gridshells est couramment attribuée à l'architecte et ingénieur allemand Frei Otto, qui les a intensivement étudiés au XXème siècle. En 1975 il réalise, en collaboration avec l'ingénieur Edmund Happold du bureau Arup, un projet expérimental de grande ampleur~: la multi-halle de Mannheim. Cette réalisation emblématique ancrera durablement les gridshells dans le paysage des typologies structurelles candidates à l’avènement de géométries non-standard, caractérisées par l'absence d'orthogonalité. Cette capacité à \guil{former la forme} de façon efficiente prend tout son sens dans le contexte actuel où, d’une part la forme s'impose comme une composante pré-dominante de l'architecture moderne (Gehry, Hadid, ...), et d’autre part l'enveloppe s'affirme comme le lieu névralgique de la performance environnementale des bâtiments.
%
%Dans ce manuscrit,
\vspace{12pt}\\
\textbf{Keywords}~: gridshell, form-finding, active-bending, free-form, torsion, elastic rod
\end{otherlanguage}
