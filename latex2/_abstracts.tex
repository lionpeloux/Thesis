% !TEX encoding = UTF-8 Unicode

% ==============
% English abstract
% ==============
\cleardoublepage
\chapter*{Abstract}
\markboth{Abstract}{Abstract}
\addcontentsline{toc}{chapter}{Abstract (English/Français)}

\vspace{-16pt}
{
\small
An \emph{elastic gridshell} is a freeform structure, generally doubly curved, but formed out through the reversible deformation of a regular and initially flat structural grid. Building curved shapes that may seems to offer the best of both worlds~: shell structures are amongst the most performant mechanically speaking while planar and orthogonal constructions are much more efficient and economic to produce than curved ones. This ability to \textquote{form a form} efficiently is of peculiar importance in the current context where morphology is a predominant component of modern architecture, and envelopes appear to be the neuralgic point for building performances.

The concept was invented by Frei Otto, a German architect and structural engineer who devoted many years of research to gridshells. In 1975 he designed the Multihalle of Mannheim, a \SI{7500}{m^2} wooden shell which demonstrated the feasibility of this technology and made it famous to a wide audience. However, despite their potential, very few projects of this kind were built after this major realization. And for good reason, the resources committed at that time cannot guarantee the replicability of this experiment for more standard projects, especially on the economic level. Moreover, the technics and methods developed by Otto's team in the 1960s have mostly fall into disuse or are based on disciplines that have considerably evolved. New materials, such as composite materials, have recently emerged. They go beyond the limitations of conventional materials such as timber and offer at all levels much better technical performances for this kind of application. Finally, it should be noted that the regulatory framework has also deeply changed, bringing a certain rigidity to the penetration of innovations in the building industry. Therefore, the design of gridshells arises in new terms for current architects and engineers and comes up against the inadequacy of existing tools and methods.

In this thesis, which marks an important step in a personal research adventure initiated in 2010, we try to embrace the issue of the design of elastic gridshells in all its complexity, addressing both theoretical, technical and constructive aspects. In a first part, we deliver a thorough review of this topic and we present in detail one of our main achievements, the ephemeral cathedral of Créteil, built in 2013 and still in service. In a second part, we develop an original discrete beam element with a minimal number of degrees of freedom adapted to the modeling of bending and torsion inside gridshell members with anisotropic cross-section. Enriched with a ghost node, it allows to model more accurately physical phenomena that occur at connections or at supports. Its numerical implementation is presented and validated through several test cases. Although this element has been developed specifically for the study of elastic gridshells, it can advantageously be used in any type of problem where the need for an interactive computation with elastic rods taking into account flexion-torsion couplings is required.

\textbf{Keywords}~: gridshell, form-finding, active-bending, free-form, torsion, elastic rod, coupling, fibreglass, composite material.
}

% ==============
% French abstract
% ==============
\cleardoublepage
\begin{otherlanguage}{french}

\chapter*{Résumé}
\markboth{Résumé}{Résumé}
\vspace{-16pt}
{
\small
Les structures de type \emph{gridshell élastique} permettent de réaliser des enveloppes courbes par la déformation réversible d'une grille structurelle régulière initialement plane. Cette capacité à \textquote{former la forme} de façon efficiente prend tout son sens dans le contexte actuel où, d’une part la forme s'impose comme une composante prédominante de l'architecture moderne, et d’autre part l'enveloppe s'affirme comme le lieu névralgique de la performance des bâtiments.

Fruit des recherches de l'architecte et ingénieur allemand Frei Otto dans les années 1960, elles ont été rendues populaires par la construction de la Multihalle de Mannheim en 1975. Cependant, en dépit de leur potentiel, très peu de projets de ce type ont vu le jour suite à cette réalisation emblématique qui en a pourtant démontré la faisabilité à grande échelle. Et pour cause, les moyens engagés à l'époque ne sauraient assurer la reproductibilité de cette expérience dans un contexte plus classique de projet, notamment sur le plan économique. Par ailleurs, les techniques et les méthodes développées alors sont pour la plus part tombées en désuétude ou reposent sur des disciplines scientifiques qui ont considérablement évoluées. Des matériaux nouveaux, composites, ont vu le jour. Ils repoussent les limitations intrinsèques des matériaux usuels tel que le bois et offrent des performances techniques bien plus intéressantes pour ce type d'application. Enfin, notons que le cadre réglementaire a lui aussi profondément muté, apportant une certaine rigidité vis-à-vis de la pénétration des innovations. Ainsi la conception des gridshells se pose-t-elle en des termes nouveaux aux architectes et ingénieurs actuels et se heurte à l'inadéquation des outils et méthodes existant.

Dans cette thèse, qui marque une étape importante dans une aventure de recherche personnelle initiée en 2010, nous tentons d'embrasser la question de la conception des gridshells élastiques dans toute sa complexité, en abordant aussi bien les aspects théoriques que techniques et constructifs. Dans une première partie, nous livrons une revue approfondie de cette thématique et nous présentons de façon détaillée l'une de nos principales réalisation, la cathédrale éphémère de Créteil, construite en 2013 et toujours en service. Dans une seconde partie, nous développons un élément de poutre discret original avec un nombre minimal de degrés de liberté adapté à la modélisation de la flexion et de la torsion dans les gridshells constitués de poutres de section anisotrope. Enrichi d'un noeud fantôme, il permet de modéliser plus finement les phénomènes physiques au niveau des connexions et des appuis. Son implémentation numérique est présentée et validée sur quelques cas tests. Bien que cet élément ait été développé spécifiquement pour l'étude des gridshells élastiques, il pourra avantageusement être utilisé dans tout type de problème où la nécessité d'un calcul interactif avec des tiges élastiques prenant en compte les couplages flexion-torsion s'avère nécessaire.

\textbf{Keywords}~: gridshell, form-finding, active-bending, free-form, torsion, elastic rod, coupling, fibreglass, composite material.
}
\end{otherlanguage}
