% Author :  Lionel du Peloux
% Contact : lionel.dupeloux@gmail
% Year : 2017

\chapter{Numerical Model} \label{chp:numerical_model}

\section{Introduction}

Penser à expliquer le dualisme edge / vertex. En pratique, on a envie de contrôler le model par les noeuds et les propriétés sur les bords. Ce que peu de models savent faire. C'est aussi mieux pour les conditions de bords. C'est le vrai avantage de ce modèle.

Expliquer l'interpolation des efforts. Il y a un choix a faire dans la DR. Soit on écrit l'équilibre sur le noeud (de dimension null). Soir on écrit l'équilibre sur une tranche de dimension $l_i+l{i+1}/2$. Mais si l'on veut maintenir la précision des conditions de passage, la DR fonctionne avec des noeuds et pas des tranches, le mieux est d'opter pour un équilibre du noeud. La DR va minimiser les résidus aux noeuds et donc on aura les bonnes conditions de passage sous Fext et Mext. J'ai montré que la convergence était plus rapide et précise. On a pas de décalage dans les diagrammes des efforts (si il y en a, ce sont uniquement les défauts de convergence, le résidu , n'étant pas strictement nul au noeud à la fin de la DR).

\subsection{Overview}

\subsection{Contributions}
\begin{itemize}
	\item We use the parallel transport in time and not in space.
\end{itemize}

\subsection{Related works}

\section{Main hypothesis}

On néglige les forces d'inertie liées à la rotation de l'élément  (devant quoi ?? traitement quasi-statique par rapport à la rotation). Cette hypothèse est faite explicitement chez Florence Bertail~:



Cette hypothèse est faite mais passée sous silence chez Douthe, Adriaenssen, D'Amico lorsqu'ils déduisent l'effort tranchant du moment de flexion.

Principe~:

- les équations constitutives permettent le calcul de $M_1$, $M_2$, $Q$ à partir de la géométrie $\{\vect{x},\theta\}$.

- La seconde loi de kirchhoff projetée sur les axes matériels 1 et 2 de la section me donnent accès aux efforts tranchants $T_1$ et $T_2$.

- La seconde loi de kirchhoff projetée sur les axes matériel 3 (tangente à la centerline) de la section me donnent l"hypothèse quasi-statique de Audoly.

%\begin{figure}[t]
%	\centering
%	\includegraphics[]{discrete_model.pdf}
%	\caption{Biarc model for a discrete beam. The centerline is divided into curved segments (grey solid hatch). Each segment is defined as a 3-noded element with uniform material and section properties. It has two end vertices (white) called \emph{handle} as they are used to interact with the model, for instance to apply loads or restrains. It has one mid vertex (grey) called \emph{ghost} as it is used only to enrich the segment kinematics and is not accessible to the end user.}
%	\label{fig:discrete_model}
%\end{figure} 


\section{Discret beam model}
% =======================
Let's introduce the discrete biarc model to describe the configuration of a beam. It is composed of a discrete curve called \emph{centerline} ($\Gamma$) and a discrete adapted frame called \emph{material frame} as its axes are chosen to be the principal axes of the beam cross-section (\cref{fig:discrete_model}). The centerline itself is organized in $n_s$ consecutive adjacent segments which are three-vertices and two-edges elements with uniform material and section properties.

Beams can either be closed or open. The corresponding number of vertices, edges and segments are reported in \cref{tab:count}.
\begin{figure}[p]
\begin{fullpage}
	\captionsetup[subfloat]{captionskip=20pt}
     	\centering
     	\subfloat[][Centerline of the discrete biarc model.]{\includegraphics{discrete_model.pdf}\label{fig:discrete_model}} \\
	\vspace{30pt}
	\subfloat[][Number of segments, edges and vertices whether the centerline is closed or open.]{
		 \ra{1.2}
		 \begin{tabularx}{0.45\textwidth}{@{} X l r  r@{}}
		 \toprule
		 		&			& \multicolumn{2}{c}{Centerline} \\ \cmidrule(l){3-4}
		 Item 	&  Symbol 	& Open 	& Closed \\
		 \midrule
		segments & $n_s$ & $n_s$ & $n_s$ \\
		edges & $n_e$ & $2 n_s$ & $2 n_s$ \\
		vertices & $n$ & $2 n_s + 1$ & $2 n_s$ \\
		ghosts & $n_g$ & $n_s$ & $n_s$ \\
		handles & $n_h$ & $n_s+1$ & $n_s$ \\
		\bottomrule
	\end{tabularx}
	\label{tab:count}}
	\vspace{10pt}
	\caption{Biarc model for a discrete beam. The centerline is divided into curved segments (grey solid hatch). Each segment is defined as a three-noded element with uniform material and section properties. It has two end vertices (white) called \emph{handle} as they are used to interact with the model, for instance to apply loads or restrains. It has one mid vertex (grey) called \emph{ghost} as it is used only to enrich the segment kinematics and is not accessible to the end user.}
\end{fullpage}
\end{figure}

\subsubsection{Centerline}
% --------------------------------
The discrete centerline is a polygonal space curve (\cref{fig:discrete_model}) defined as an ordered sequence of $n+1$ pairwise disjoint \emph{vertices}~: $\Gamma = (\vect{x}_0,  \vect{x}_1, \ldots, \vect{x}_n) \in \mathbb{R}^{3(n+1)}$. Consecutive pairs of vertices define $n$ straight segments $(\vect{e}_0,  \vect{e}_1, \ldots, \vect{e}_{n-1})$ called \emph{edges} and pointing from one vertex to the next one~: 
\begin{subequations}
	\begin{alignat}{1}
	\vect{e}_i 	&= \vect{x}_{i+1} - \vect{x}_{i}
	\\
	l_i 		&= \norm{\vect{e}_i} 
	\\
	\vect{u}_i 	&= \vect{e}_i / l_i = \vect{d}_{3, i+1/2} 
	\end{alignat}
\end{subequations}
The length of the $i$th edge is denoted $l_i $ and its normalized direction vector is denoted $\vect{u}_i$. The arc length of the $i$th vertex is denoted $s_i$ and is given by~: 
\begin{equation}
	\left\{
	\begin{aligned}
		s_0 	&= 0 				& 	&i = 0		\\
		s_i 	&= \sum_{k=0}^{i-1} l_k	&	&i \in \llbracket 1, n-1 \rrbracket	\\
		s_n 	&=  L 				&	&i = n		\\
	\end{aligned}
	\right.
\end{equation}
Thus, the centerline is parametrized by arc length and $\Gamma(s_i) = \vect{x}_i$. Additionally, we define the vertex-based mean length at vertex $\vect{x}_i$~: 
\begin{equation}
%\setlength{\jot}{8pt}
	\left\{
	\begin{aligned}
		\rconf{l}_0 	& =  \tfrac{1}{2}l_0				&		&i = 0					\\
		\rconf{l}_i	& =  \tfrac{1}{2}(l_{i-1} + l_i)		&		&i \in \llbracket 1, n-1 \rrbracket	\\
		\rconf{l}_n 	& =  \tfrac{1}{2}l_{n-1} 			&		&i = n					\\
	\end{aligned}
	\right.
\end{equation}

\subsubsection{Segments}
% ---------------------------------
The discrete centerline is divided into $n_s$ curved segments. Each segment is a three-noded element -- see \cref{fig:discrete_model} where the area covered by a segment is represented as a grey solid hatch. The $i$th segment is composed of three vertices ($\vect{x}_{2i}, \vect{x}_{2i+1},  \vect{x}_{2i+2}$) spanning two edges ($\vect{e}_{2i}, \vect{e}_{2i+1}$). The (i-1)th segment and the $i$th segment share the same vertex $\vect{x}_{2i}$ at arc length $s_{2i}$.

Each segment has two end vertices called \emph{handle} ($\vect{x}_{2i}, \vect{x}_{2i+2}$) and one mid vertex called \emph{ghost} ($\vect{x}_{2i+1}$) as this one is not accessible to the end user in order to interact with the model (link, restrain, loading, \dots). Ghost vertices are used only for internal purpose to give a higher richness in the kinematic description of a segment than a two-noded segment would.

Finally, we define the \emph{chord length} of the $i$th segment as the distance between $\vect{x}_{2i}$ and $\vect{x}_{2i+2}$~:
\begin{equation}
	L_i = \norm{\vect{e}_{2i} + \vect{e}_{2i+1}} \quad , \quad i \in \llbracket 0, n_s-1 \rrbracket
\end{equation}

\subsubsection{Material and section properties}
% ------------------------------------------------------------
In addition, the model assumes that a segment has uniform section ($S$, $I_1$, $I_2$, $J$)\footnote{$S$ is the cross-section area ; $I_1$, $I_2$ and $J$ are the principal moments of inertia of the cross-section.} and material ($E$, $G$)\footnote{$E$ is the elastic modulus and $G$ is the shear  modulus for the considered material} properties over its length $s \in ]s_{2i},s_{2i+2}[$. For the sake of simplicity, we introduce for further calculations the \emph{material stiffness matrix} ($\mat{B}_i$) attached to each segment. It has the following form in the material frame basis~:
\begin{equation}
	\mat{B}_i = \begin{bmatrix} 
			EI_1		&	0		&	0		\\
			0		&	EI_{2}	&	0		\\
			0		&	0		&	GJ_{}	\\
		\end{bmatrix}_i
	\quad , \quad i \in \llbracket 0, n_s-1 \rrbracket
	\label{eq:stiffnessmatrix}
\end{equation}
where $EI_1$ and $EI_2$ are the bending stiffnesses and $GJ$ is the torsional stiffness. The axial stiffness of the $i$th segment is denoted by :
\begin{equation}
	ES_i 	\quad , \quad i \in \llbracket 0, n_s-1 \rrbracket
\end{equation}

\subsubsection{Distributed loads}
% -----------------------------------------
The model assumes that each segment can be loaded with some distributed forces ($\vect{f}^{ext} = f_k\vect{d}_k$) and moments ($\vect{m}^{ext} = m_k\vect{d}_k$). These forces and moments are required to be uniform over each segment but can vary from one segment to another. They can represent body loads such as self weight or thermal loads or external loads such as wind, snow, pressure,~\dots

\subsubsection{Concentrated loads}
% ---------------------------------------
Additional External concentrated forces ($\vect{F}^{ext}$) and moments ($\vect{M}^{ext}$) are applied to the segment's end vertices ($\vect{x}_{2i}$,  $\vect{x}_{2i+2}$). Note that the model does not allow to load ghost vertices, and this is precisely why they are called \textquote{ghost}.

 \subsubsection{Internal forces and moments}
% ---------------------------------------------------------
Under deformations the discrete rod is subject to internal forces and moments. Their components in the material frame basis are named as follow :
\begin{itemize}
\item The shear force : $\vect{F}^{\perp} = F_1 \vect{d}_1 + F_2 \vect{d}_2$ 
\item The axial force : $\vect{N} = N \vect{d}_3$
\item The bending moment : $\vect{M} = M_1 \vect{d}_1 + M_2 \vect{d}_2$ 
\item The twisting moment : $\vect{Q} = Q \vect{d}_3$ 
\end{itemize}

% Modeling discontinuities
% ------------------------------------
 \subsection{Modeling discontinuities}
The model assumes that cross-section and material properties as well as distributed loads are uniform over each segment. Referring to the structure of the equations of motion, and because the centerline is required to be a regular curve in the stress-free configuration, strains, stresses, displacements, internal forces and internal moments must be piecewise continuous functions of the arc length parameter, continuous over each segment $]s_{2i},s_{2i+2}[$. Discontinuities of these functions might occur at handle vertices ($\vect{x}_{2i}$), for instance if there is a jump in material or cross-section properties or if concentrated loads are applied at handle vertices. Moreover, the centerline curve itself will stay $\mathcal{C}^1$ during the motion, as it is chosen to be $\mathcal{C}^1$ in the reference configuration.\footnote{This preclude the modeling of beams with kinks as the tangent vector would not be continuously defined at those points. In such a case, the beam should be modeled in two separate parts linked together in a rigid manner.}\textsuperscript{,}\footnote{The centerline is not $\mathcal{C}^2$ as discontinuities in curvature may occur. For instance, if no punctual loads are applied, the bending moment is continuous over the rod. As the bending moment is linked to the curvature through the constitutive equation $M = EI\varkappa$, a discontinuity in $I$ will lead to a discontinuity in $\varkappa$. Conversely, a discontinuity in $\varkappa$ will lead to a discontinuity in $I$.}

Here and subsequently, for such a function, the left and right limits at handle vertices ($s_{2i}$) will be denoted with superscripts $f_{2i}^-$ and $f_{2i}^+$. Possibly, the function is continuous so that the left and right limits agree ($f_{2i}^- = f_{2i}^+$).

% Matrix notation
% --------------------
\subsection{Matrix notation}
Here and subsequently, matrix notation will often be used to provide compact expressions for the equations, where the components of vector-valued functions are given in the material frame basis. This notation will be mixed with the vector notation employed  more generally throughout this document. Usually, if there is no comment in the manuscript, the meaning should be obvious and with no ambiguity to the reader.

For instance, all this expressions for the curvature binormal vector and the material curvatures vector will be considered equivalent and could be mixed together in the same equation~:
\begin{subequations}
	\begin{alignat}{1}
	&\vect{\kappa b} 
	= \kappa_1 \vect{d}_1 +  \kappa_2 \vect{d}_2 
	= \begin{bmatrix} \kappa_1 & \kappa_2 & 0 \end{bmatrix}^T
	\\
	&\vect{\varkappa} 
	= \varkappa_1 \vect{d}_1 +  \varkappa_2 \vect{d}_2 +  \varkappa_3 \vect{d}_3 
	= \begin{bmatrix} \varkappa_1 & \varkappa_2 & \varkappa_3 \end{bmatrix}^T
	= (1+\epsilon) \begin{bmatrix} \kappa_1 & \kappa_2 & \tau \end{bmatrix}^T
	\end{alignat}
\end{subequations}
The force strains vector is given by :
\begin{equation}
	\vect{\eta} = (1+\epsilon)\vect{d}_3 = \begin{bmatrix} 0 & 0 & 1+\epsilon \end{bmatrix}^T
\end{equation}
Internal forces are composed of a shear force and an axial force given by~:
\begin{subequations}
	\begin{alignat}{3}
	&\vect{F} 
	&&= F_1 \vect{d}_1 +  F_2 \vect{d}_2 + N \vect{d}_3 
	= \begin{bmatrix} F_1 & F_2 & N \end{bmatrix}^T
	\\
	&\vect{F}^{\perp} 
	&&= F_1 \vect{d}_1 +  F_2 \vect{d}_2
	= \begin{bmatrix} F_1 & F_2 & 0 \end{bmatrix}^T
	\\
	&\vect{F}^{\para} &&= \vect{N}
	= N \vect{d}_3 
	= \begin{bmatrix} 0 & 0 & N \end{bmatrix}^T
	\end{alignat}
\end{subequations}
Internal moments are composed of a bending moment and a twisting moment given by~:
\begin{subequations}
	\begin{alignat}{3}
	&\vect{M} 
	&&= M_1 \vect{d}_1 +  M_2 \vect{d}_2 + Q \vect{d}_3 
	= \begin{bmatrix} M_1 & M_2 & Q \end{bmatrix}^T
	\\
	&\vect{M}^{\perp} 
	&&= M_1 \vect{d}_1 +  M_2 \vect{d}_2
	= \begin{bmatrix} M_1 & M_2 & 0 \end{bmatrix}^T
	\\
	&\vect{M}^{\para} &&= \vect{Q}
	= Q \vect{d}_3 
	= \begin{bmatrix} 0 & 0 & Q \end{bmatrix}^T
	\end{alignat}
\end{subequations}
With the help of the matrix notation, the constitutive equations \cref{eq:constitutive_b,eq:constitutive_c,eq:constitutive_d} together write in a single equation~:
\begin{equation}
	\mat{M} = \mat{B} (\vect{\varkappa} - \rconf{\vect{\varkappa}})
	= EI_1(\varkappa_1 - \rconf{\varkappa}_1) \vect{d}_1
	+ EI_2(\varkappa_2 - \rconf{\varkappa}_2) \vect{d}_2
	+ GJ (\varkappa_3 - \rconf{\varkappa}_3) \vect{d}_3
\end{equation}

%We also introduce the matrix notation for the derivatives of the internal forces and moments referring to the inertial frame of reference $\{\vect{x},\vect{y},\vect{z}\}$ and to the local frame $\{\vect{d}_{1},\vect{d}_{2},\vect{d}_{3}\}$. The (spatial) velocity of $\{\vect{d}_{1},\vect{d}_{2},\vect{d}_{3}\}$ implies that~:
%\begin{subequations}
%	\begin{alignat}{3}
%	&\vect{F}' = \vect{F}^{'|xyz} = \vect{F}^{'|123} + \vect{\varkappa} \times \vect{F}
%	\\
%	&\vect{M}' = \vect{M}^{'|xyz} = \vect{M}^{'|123} + \vect{\varkappa} \times \vect{M}
%	\end{alignat}
%\end{subequations}
%where $\vect{F}_{xyz}'$ and $\vect{M}_{xyz}'$ denote the derivative with respect to $s$ in the inertial frame of reference $\{\vect{x},\vect{x},\vect{z}\}$ whereas $\vect{F}_{123}'$ and $\vect{M}_{123}'$ denote the derivative with respect to $s$ in the local material frame basis $\{\vect{d}_{1},\vect{d}_{2},\vect{d}_{3}\}$. Thus the matrix notation for $\vect{F}_{123}'$ and $\vect{M}_{123}'$ are given by :
%\begin{subequations}
%	\begin{alignat}{3}
%	&\vect{F}^{'|123} 
%	&&= F_1' \vect{d}_1 +  F_2' \vect{d}_2 + N' \vect{d}_3 
%	= \begin{bmatrix} F_1' & F_2' & N' \end{bmatrix}^T
%	\\
%	&\vect{M}^{'|123} 
%	&&= M_1' \vect{d}_1 +  M_2' \vect{d}_2 + Q' \vect{d}_3 
%	= \begin{bmatrix} M_1' & M_2' & Q' \end{bmatrix}^T
%	\end{alignat}
%\end{subequations}

% Discret extension and axial force
% --------------------------------------------
\subsection{Discret extension and axial force}
We assume the axial force ($\vect{N}$) to vary linearly over $]\vect{x}_{2i},  \vect{x}_{2i+2}[$ regarding the arc length parameter. The variation occurs if the segment is subject to a uniform distributed load $f_3$ over the segment. Consequently, the axial strain ($1 + \epsilon$) is also required to vary linearly. The value of the axial extension at mid span of each edge are given by~:
\begin{equation}
	\epsilon_{i+1/2} = {l_{i}}\,/\,{\rconf{l}_i} - 1 \quad , \quad i \in \llbracket 0, n_e-1 \rrbracket
\end{equation}
Consequently, the axial force at mid span of each edge is computed directly with the constitutive equation \cref{eq:constitutive_a} as~:
\begin{subequations}
	\begin{alignat}{3}
	\vect{N}_{2i+1/2} &= N_{2i+1/2}\, \vect{u}_{2i} \quad &&\text{where} \quad {N_{2i+1/2}} = ES_i \,\epsilon_{2i+1/2}
	\\
	\vect{N}_{2i+3/2} &= N_{2i+3/2}\, \vect{u}_{2i+1}\quad &&\text{where} \quad {N_{2i+3/2}} = ES_i \,\epsilon_{2i+3/2}
	\end{alignat}
\end{subequations}
Remark the sign convention~: as expected, when edge $\vect{e}_i$ suffers a positive extension ($\epsilon_{i+1/2} > 0$), vertex $\vect{x}_{i+1}$ \textquote{attracts} vertex $\vect{x}_{i}$ to it as $\vect{d}_{3,i+1/2} = \vect{u}_{i}$ is pointing from $\vect{x}_{i}$ towards $\vect{x}_{i+1}$. Remark also that $\epsilon_{i+1/2} = 0 \Leftrightarrow l_i = \rconf{l}_i$ when the rod is not stretched.

% Discret bending moments and curvatures
% -------------------------------------------------------
\subsection{Discret curvature and bending moment}
We assume that the internal bending moment and curvature are quadratic functions of the arc length parameter over $]\vect{x}_{2i},  \vect{x}_{2i+2}[$.
Although they must be continuous over this interval, they might be discontinuous at handle vertices and be subjected to jump discontinuities in direction and magnitude.

%\clearpage
\def\tabularxcolumn#1{m{#1}} % vertical center in X column

\subsubsection{Geometric curvature at ghost vertices}
% --------------------------------------------------------
For a given geometry of the centerline, the curvature binormal vector at ghost vertex  $\vect{x}_{2i-1}$ (resp. $\vect{x}_{2i+1}$) is computed considering the circumscribed osculating circle passing through the vertices ($\vect{x}_{2i-2}, \vect{x}_{2i-1},  \vect{x}_{2i}$) of the ($i-1$)th segment -- resp. through the vertices ($\vect{x}_{2i}, \vect{x}_{2i+1},  \vect{x}_{2i+2}$) of the $i$th segment.

\begin{tabularx}{\textwidth}[t]{>{\centering\arraybackslash}m{0.48\textwidth} >{\centering\arraybackslash}X} % >{\centering\arraybackslash}
	\includegraphics[]{E1.pdf}
	& 
	$\begin{aligned}[t] % placement: default is "center", options are "top" and "bottom"
	\vect{\kappa b}_{2i-1} 	& =  \frac{2}{L_{i-1}} \vect{u}_{2i-2} \times \vect{u}_{2i-1}\\[0.5em]
	\vect{\kappa b}_{2i+1} 	& =  \frac{2}{L_{i}} \vect{u}_{2i} \times \vect{u}_{2i+1} 
	\end{aligned}$
\end{tabularx}


\subsubsection{Unit tangent vector at ghost vertices}
% ---------------------------------------------------------------------

This definition of the curvature leads to a natural definition of the unit tangent vector at ghost vertex $\vect{x}_{2i-1}$ (resp. $\vect{x}_{2i+1}$), as the unit vector tangent to the osculating circle of the ($i-1$)th segment (resp. $i$th segment) at that point. 

\begin{tabularx}{\textwidth}[t]{>{\centering\arraybackslash}m{0.48\textwidth} >{\centering\arraybackslash}X} % >{\centering\arraybackslash}
	\includegraphics[]{E2.pdf}
	& 
	$\begin{aligned}[t] % placement: default is "center", options are "top" and "bottom"
	\vect{t}_{2i-1}	&=  \frac{l_{2i-1}}{L_{i-1}} \vect{u}_{2i-2}	+ 	\frac{l_{2i-2}}{L_{i-1}} \vect{u}_{2i-1} 	\\[0.5em]
	\vect{t}_{2i+1} 	&=  \frac{l_{2i+1}}{L_{i}} \vect{u}_{2i}		+ 	\frac{l_{2i}}{L_{i}} \vect{u}_{2i+1} 	
	\end{aligned}$
\end{tabularx}

\subsubsection{Left/right unit tangent vector at handle vertices}
% -----------------------------------------------------------------------------------
Equivalently, the definition of the osculating circles of the ($i-1$)th and $i$th segments leads to a natural definition of the left ($\vect{t}_{2i}^-$) and right ($\vect{t}_{2i}^+$) unit tangent vectors at handle vertex $\vect{x}_{2i}$, for segments of uniform curvature. When both segments have the same curvature, left and right vectors agree.

\begin{tabularx}{\textwidth}[t]{>{\centering\arraybackslash}m{0.48\textwidth} >{\centering\arraybackslash}X} % >{\centering\arraybackslash}
	\includegraphics[]{E3.pdf}
	& 
	$\begin{aligned}[t] % placement: default is "center", options are "top" and "bottom"
	\vect{t}_{2i}^- 	&= 2 (\vect{t}_{2i-1} \cdot \vect{u}_{2i-1}) \vect{u}_{2i-1} - \vect{t}_{2i-1} \\[0.5em]
	\vect{t}_{2i}^+ 	&= 2 (\vect{t}_{2i+1} \cdot \vect{u}_{2i}) \vect{u}_{2i} - \vect{t}_{2i+1}
	\end{aligned}$
\end{tabularx}

\subsubsection{Unit tangent vector at handle vertices}
% ----------------------------------------------------------------------
The unit tangent vector $\vect{t}_{2i}$ -- that is the beam section normal -- at handle vertex $\vect{x}_{2i}$ is chosen to be the mean of the left and right unit tangent vectors at that vertex.\footnote{Consequently, this model assumes that the field of tangents along the centerline is continuous and is thus unable to model cases where the centerline is not at least $\mathcal{C}^1$. In such case the beam must be considered as two parts glued together.}

\begin{tabularx}{\textwidth}[t]{>{\centering\arraybackslash}m{0.48\textwidth} >{\centering\arraybackslash}X} % >{\centering\arraybackslash}
	\includegraphics[]{E4.pdf}
	& 
	$\begin{aligned}[t] % placement: default is "center", options are "top" and "bottom"
	\vect{t}_{2i} 	&= \frac{\vect{t}_{2i}^- + \vect{t}_{2i}^+ }{\norm{\vect{t}_{2i}^- + \vect{t}_{2i}^+}}
	\end{aligned}$
\end{tabularx}

This way, the determination of the tangent vectors -- or equivalently the section normals -- in the static equilibrium configuration will be done in the flow of the dynamic relaxation process, without the need of introducing any additional degrees of freedom (for instance the usual Euler angles). The position of the vertices rules the orientation of the section normals.

\subsubsection{Left/right bending moment at handle vertices}
% ---------------------------------------------------------------------------------
Given the unit tangent vector $\vect{t}_{2i}$, one can define the left ($\vect{\kappa}_{2i}^-$) and right ($\vect{\kappa}_{2i}^+$) curvature at handle vertex $\vect{x}_{2i}$. The left curvature is initially evaluated from the left osculating circle, defined as the circle passing through $\vect{x}_{2i-1}$ and $\vect{x}_{2i}$ and tangent to $\vect{t}_{2i}$ at $\vect{x}_{2i}$. The right curvature is initially evaluated from the right osculating circle, defined as the circle passing through $\vect{x}_{2i}$ and $\vect{x}_{2i+1}$ and tangent to $\vect{t}_{2i}$ at $\vect{x}_{2i}$.\footnote{Remark that the centerline is now approximated with a biarc in the vicinity of $\vect{x}_{2i}$. This is the reason why this model is called the \textquote{biarc model}.}${}^,$\footnote{This model offers the ability to represent discontinuities in curvature -- thus in bending moment -- at handle vertices as the left and right curvatures does not necessarily agree. This is quite different from the classical 3-dof element~\cite{Barnes1999, Adriaenssens1999, Douthe2006} which assumes that the curvature -- thus the bending moment -- is $\mathcal{C}^0$ and can be evaluated at every vertices from the circumscribed osculating circle.}

\begin{tabularx}{\textwidth}[t]{>{\centering\arraybackslash}m{0.48\textwidth} >{\centering\arraybackslash}X} % >{\centering\arraybackslash}
	\includegraphics[]{E5.pdf}
	& 
	$\begin{aligned}[t] % placement: default is "center", options are "top" and "bottom"
	\vect{\kappa b}_{2i}^- &=  \frac{2}{l_{2i-1}} \vect{u}_{2i-1} \times \vect{t}_{2i}
	\\[0.5em]
	\vect{\kappa b}_{2i}^+ &=  \frac{2}{l_{2i}} \vect{t}_{2i} \times \vect{u}_{2i}
	\end{aligned}$
\end{tabularx}

However, this values need to be adjusted so that the static condition for rotational equilibrium is satisfied at all time ($\vect{M}^{ext}  + \vect{M}^+ - \vect{M}^- = 0$). Then, this condition will be satisfied in particular at the end of the solving process. To achieve this goal, we first compute a realistic mean value ($\vect{M}_{2i}$) for the internal bending moment as~:
\begin{equation}
		\vect{M}_{2i}^{\perp}	=  \frac{1}{2} \mat{B}_{i-1}(\vect{\kappa b}_{2i}^- - \rconf{\vect{\kappa b}}_{2i}^-)
					+  \frac{1}{2} \mat{B}_{i}(\vect{\kappa b}_{2i}^+ - \rconf{\vect{\kappa b}}_{2i}^+)
\end{equation}
To enforce the jump discontinuity in bending moment ($\vect{M}^{ext} = \vect{M}^- - \vect{M}^+$) across the handle vertex, we define the left and right bending moments at $\vect{x}_{2i}$ as~:
\begin{subequations}
	\begin{align}
		{\vect{M}_{2i}^{\perp}}^{-}	&=  \vect{M}_{2i} + \frac{1}{2} \vect{M}_{2i}^{ext} 
		\\[0.5em]
		{\vect{M}_{2i}^{\perp}}^{+}	&=  \vect{M}_{2i} - \frac{1}{2} \vect{M}_{2i}^{ext} 
	\end{align}
\end{subequations}
Note that in the case where no external concentrated bending moment is applied to the handle vertex, the internal bending moment is continuous across the vertex.

\subsubsection{Left/right curvature at handle vertices}
% --------------------------------------------------------------------
Finally, the left and right curvature at handle vertex $\vect{x}_{2i}$ are computed back with the constitutive law~:
\begin{subequations}
	\begin{align}
		\vect{\kappa b}_{2i}^-  &=  \mat{B}_{i-1}^{\;-1} {\vect{M}_{2i}^{\perp}}^{-} + \rconf{\vect{\kappa b}}_{2i}^- 
		\\[0.5em]
		\vect{\kappa b}_{2i}^+  &=  \mat{B}_{i}^{\;-1} {\vect{M}_{2i}^{\perp}}^{+} + \rconf{\vect{\kappa b}}_{2i}^+ 
	\end{align}
\end{subequations}

\subsubsection{Bending moment at ghost vertices}
% -----------------------------------------------------------------
The internal bending moment at ghost vertices is simply given by the constitutive law as~:
\begin{subequations}
	\begin{align}
		\vect{M}_{2i-1}^{\perp} &=  \mat{B}_{i-1}(\vect{\kappa b}_{2i-1} - \rconf{\vect{\kappa b}}_{2i-1})
		\\[0.5em]
		\vect{M}_{2i+1}^{\perp} &=  \mat{B}_{i}(\vect{\kappa b}_{2i+1} - \rconf{\vect{\kappa b}}_{2i+1})
	\end{align}
\end{subequations}

\subsection{Discret rate of twist and twisting moment}
% =======================================
We assume the twisting moment and the rate of twist to vary linearly over $]\vect{x}_{2i},  \vect{x}_{2i+2}[$.
Thus, the material twist of the rod at mid edge is given by~:
\begin{subequations}
	\begin{align}
		\tau_{i+1/2} = \frac{\Delta\theta_{i}}{{l}_i}\label{eq:d_twist}
		\\[0.5em]
		\varkappa_{3,i+1/2} = \frac{\Delta\theta_{i}}{\rconf{l}_i}\label{eq:d_twist}
	\end{align}
\end{subequations}
To compute $\Delta\theta_{i} = \theta_{i+1} - \theta_{i}$ imagine that the rod si framed with a Bishop frame $\{\vect{u},\vect{v},\vect{t}\}$. Because the material frame $\{\vect{d}_1,\vect{d}_2,\vect{d}_3\}$ is also adapted to the rod centerline, it can be transformed into the Bishop frame with a single rotation of angle $\theta(s)$ around $\vect{d}_3(s) =\vect{t}(s)$. Because the Bishop frame does not twist around $\vect{d}_3$, the rate of change of angle $\theta$ along the curve directly leads to the computation of the rate of twist as exposed in \cref{eq:d_twist}.

In the discrete case, although it is possible to frame the whole curve with a Bishop frame to achieve the computation of the rate of twist, it is more convenient to measure $\Delta\theta_{i}$ step by step using the existing material frames at vertices $\vect{x}_i$ and $\vect{x}_{i+1}$. This is done in a to step process : 
\begin{enumerate}
\item 
Parallel transport the material frame $\{\vect{d}_{1,i},\vect{d}_{2,i},\vect{d}_{3,i}\}$ at vertex $\vect{x}_i$ onto vertex $\vect{x}_{i+1}$. We call $\{\vect{d}_{1,i}^{\para}, \vect{d}_{2,i}^{\para},\vect{d}_{3,i}^{\para}\}$ the resulting framed positioned at $\vect{x}_{i+1}$ such that $\vect{d}_{3,i}^{\para} = \vect{d}_{3,i+1}$.
\item 
Measure $\Delta\theta_{i} = \angle{(\vect{d}_{1,i}^{\para},\vect{d}_{1,i+1})} = \angle{(\vect{d}_{2,i}^{\para},\vect{d}_{2,i+1})}$ as the oriented angle needed to align $\vect{d}_{1,i}^{\para}$ with $\vect{d}_{1,i+1}$ (or $\vect{d}_{2,i}^{\para}$ with $\vect{d}_{2,i+1}$) by a rotation of angle $\Delta\theta_{i}$ around $\vect{d}_{3,i+1} = \vect{t}_{i+1}$.
\end{enumerate}
Consequently, the twisting moment at mid span of each edge is computed directly with the appropriate constitutive equation :
\begin{subequations}
	\begin{alignat}{3}
	\vect{Q}_{2i+1/2} &= Q_{2i+1/2}\, \vect{u}_{2i} \quad &&,\quad {Q_{2i+1/2}} = GJ_i(\varkappa_{3,2i+1/2} - \rconf{\varkappa}_{3,2i+1/2})
	\\
	\vect{Q}_{2i+3/2} &= Q_{2i+3/2}\, \vect{u}_{2i+1}\quad &&, \quad {Q_{2i+3/2}} = GJ_i(\varkappa_{3,2i+3/2} - \rconf{\varkappa}_{3,2i+3/2})
	\end{alignat}
\end{subequations}
Remark the sign convention~: as expected, when edge $\vect{e}_i$ suffers a positive twist ($\varkappa_{3,i+1/2} > 0$), frame $\{\vect{d}_{1,i+1},\vect{d}_{2,i+1},\vect{d}_{3,i+1}\}$ makes frame $\{\vect{d}_{1,i},\vect{d}_{2,i},\vect{d}_{3,i}\}$ to rotate positively around $\vect{u}_i$ as $Q_{i+1/2} > 0$.

% Discret shear force
% --------------------------
\subsection{Discret shear force}
Recall that in Kirchhoff's theory the shear force is a reacting parameter, computed from the equilibrium equations and not from a constitutive law. Firstly, remark that the shear force can be factorized under the following expression~:
\begin{equation}
	\vect{F}^{\perp} = F_1 \vect{d}_1 + F_2 \vect{d_2} = - \vect{d}_3 \times (\vect{d}_3 \times \vect{F}) \label{eq:shearid}
\end{equation}
Then, combining \cref{eq:motion_4,eq:motion_5} --~where the inertial terms are neglected~-- with \cref{eq:shearid} leads to the following vectoriel form of the shear force~:
\begin{equation}
	\vect{F}^{\perp} 
	= (1+\epsilon)^{-1} \vect{d}_3 \times \left(\frac{\partial \vect{M}}{\partial s} + \vect{m} \right)
	=  \vect{d}_3 \times \left(\frac{\partial \vect{M}}{\partial s_t} + \frac{\vect{m}}{1+\epsilon} \right)
	\label{eq:shear} 
\end{equation}
In the discrete case, the shear force is evaluated at mid span of each edge by :
\begin{subequations}
	\begin{alignat}{3}
		&\vect{F}^{\perp}_{2i+1/2}
		=  \vect{u}_{2i} &&\times \left( \frac{\vect{M}_{2i+1} - \vect{M}_{2i}^{+}}{l_{2i}}+ \frac{\rconf{l}_{2i}}{l_{2i}} \vect{m}_i \right)
		\label{eq:dshear_a}
		\\[0.5em]
		&\vect{F}^{\perp}_{2i+3/2}
		=  \vect{u}_{2i+1} &&\times \left( \frac{\vect{M}_{2i+2}^- - \vect{M}_{2i+1}}{l_{2i+1}}+ \frac{\rconf{l}_{2i+1}}{l_{2i+1}} \vect{m}_i \right)
		\label{eq:dshear_b}
	\end{alignat}
\end{subequations}
Remark that the derivative of the internal moments at mid edge is evaluated by the finite difference of the moment between the two closest vertices. This is in accordance with the quadratic interpolation method of a vector-valued function given in \cref{ch:interpolation}. 

Expressed in the form of \cref{eq:dshear_a,eq:dshear_b}, the discrete shear force has the interesting property to remain strictly orthogonal to $\vect{d}_{3,i+1/2} = \vect{u}_i$. While this is true in the continuous world, this property can easily be lost in the discrete case where mean values and derivatives are evaluated through finite summations or finite differences.

\subsubsection{Matrix notation}
Because there is a derivation with respect to $s$ in \cref{eq:dshear_a,eq:dshear_b}, one must be very careful when writing these equations in matrix notation. Indeed, their counterparts will translate differently wether the symbols will be decomposed in the \emph{global} frame basis or in the \emph{material} frame basis.

If the symbols are decomposed in the \emph{global} frame basis the translation is straightforward as the derivative of a vector is the vector of the derived components~:
\begin{equation}
	\vect{M} = {\begin{bmatrix}M_x \\ M_y \\ M_z\end{bmatrix}}
	\quad , \quad
	\vect{M}'
	=
	{\begin{bmatrix}M_x \\ M_y \\ M_z\end{bmatrix}}^{'}
	=
	\begin{bmatrix}M_x' \\ M_y' \\ M_z'\end{bmatrix}
\end{equation}
Thus, in the discrete case, the evaluation of the derivative of the moment at mid-edge is achieved thanks to the finite difference formula~:
\begin{equation}
	\vect{M}'_{2i+1/2}
	\simeq
	\frac{1}{\rconf{l}_{2i}}\left( {\begin{bmatrix}M_x & M_y & M_z\end{bmatrix}}_{2i+1}^T - {\begin{bmatrix}M_x & M_y & M_z\end{bmatrix}}_{2i}^T \right)
\end{equation}
However, if the symbols are given in the \emph{material} frame basis, the derivation must take into account the spatial velocity $\vect{\varkappa}$ of the material frame~:
\begin{equation}
	\vect{M} = {\begin{bmatrix}M_1 \\ M_2 \\ Q \end{bmatrix}}
	\quad , \quad
	\vect{M}'
	=
	{\begin{bmatrix}
	M_1 \\ M_2 \\ Q
	\end{bmatrix}}^{'}
	=
	\begin{bmatrix}
	M_1' \\ M_2' \\ Q'
	\end{bmatrix}
	+
	\begin{bmatrix}
	\varkappa_1 \\ {\varkappa_2} \\ \varkappa_3
	\end{bmatrix}
	\times
	\begin{bmatrix}
	M_1 \\ M_2 \\ Q
	\end{bmatrix}
\end{equation}
Thus, in the discrete case, the evaluation of the derivative of the moment at mid-edge is still achieved thanks to the finite difference formula, but takes a very different matrix form~:
\begin{equation}
	\begin{aligned}
	\vect{M}'_{2i+1/2}
	&\simeq
	&&\frac{1}{\rconf{l}_{2i}}\left( {\begin{bmatrix}M_1 & M_2 & Q\end{bmatrix}}_{2i+1}^T - {\begin{bmatrix}M_1 & M_2 & Q\end{bmatrix}}_{2i}^T \right)
	\\
	&+ &&\frac{1}{2}\left( {\begin{bmatrix}\varkappa_1 & \varkappa_2 & \varkappa_3\end{bmatrix}}_{2i}^T + {\begin{bmatrix}\varkappa_1 & \varkappa_2 & \varkappa_3\end{bmatrix}}_{2i+1}^T \right)
	\\
	&\times 
	&& \frac{1}{2} \left( {\begin{bmatrix}M_1 & M_2 & Q\end{bmatrix}}_{2i}^T + {\begin{bmatrix}M_1 & M_2 & Q\end{bmatrix}}_{2i+1}^T \right)
	\end{aligned}
\end{equation}
Although this paragraph could seem superfluous to the reader, this point is a matter of concern when implementing the model into an algorithm. Indeed, the developper always has the choice between two natural data structures where vectors are represented by a triplet either in the global frame basis or in the material frame basis (see \cref{eq:stiffnessmatrix}). Even more, he can decide to mix the two for practical reasons, for instance if it leads to less arithmetic computations. In particular, the stiffness matrix has a nice diagonal shape when written in the material frame basis. Thus it seems desirable to do the computation of the bending moment in this basis. On the contrary, we've just seen that it seems easier to compute the shear force in the global frame basis.

\subsection{Interpolation}

\clearpage
\section{Dynamic relaxation}

Dynamic relaxation~:
\cite{Lewis2003}
En particulier, voir pour un comparatif avec une méthode implicite.

\IncMargin{2em}
\begin{figure}[p]
\begin{fullpage}
\begin{algorithm}[H]
\SetKwInOut{Input}{input}
\SetKwInOut{Output}{output}
\SetKwProg{Fn}{Function}{:}{}
\SetKwFunction{Solve}{Solve}
\SetKwFunction{Init}{Init}
\SetKwFunction{Run}{Run}
\SetKwFunction{Move}{Move}
\SetKwFunction{CalculateF}{CalcInternalForces}
\SetKwFunction{CalculateM}{CalcInternalMoments}
\SetKwFunction{CalculateA}{CalcAcceleration}
\SetKwFunction{CalculateV}{CalcVelocity}
\SetKwFunction{CalculateE}{CalcKEnergy}
\SetKwFunction{SumFM}{AggregateForcesAndMoments}
\SetKwFunction{SyncFM}{SynchronizeForcesAndMoments}
\SetKwFunction{SumM}{AggregateMasses}
\SetKwFunction{SyncM}{SynchronizeMasses}
\SetKwFunction{InterpX}{InterpolatePosition}
\SetKwFunction{InterpR}{InterpolatePosition}
\SetKwFunction{Reset}{Reset}

\SetKwData{Model}{model}
\SetKwData{Node}{node}
\SetKwData{Joint}{joint}
\SetKwData{Element}{element}
\SetKwData{CVG}{cvg}
\SetKwData{MAX}{maxIteration}

%\Input{A model to analyzed}
%\Output{An analyzed model}

% SOLVE
\Fn{\Solve{\Model, \CVG, \MAX}}{
\BlankLine
\Init{} \;
\Repeat{convergence criterion reached}{
\Run{}\;
}
}{\KwRet}
\BlankLine\BlankLine

\caption{disjoint decomposition}\label{algo_disjdecomp}
\end{algorithm}
\end{fullpage}
\end{figure}
\DecMargin{2em}

% ======== ALGO RUN


\IncMargin{2em}
\begin{figure}[p]
\begin{fullpage}
\begin{algorithm}[H]
\SetKwInOut{Input}{input}
\SetKwInOut{Output}{output}
\SetKwProg{Fn}{Function}{:}{}
\SetKwFunction{Solve}{Solve}
\SetKwFunction{Init}{Init}
\SetKwFunction{Run}{Run}
\SetKwFunction{Move}{Move}
\SetKwFunction{CalculateF}{CalcInternalForces}
\SetKwFunction{CalculateM}{CalcInternalMoments}
\SetKwFunction{CalculateA}{CalcAcceleration}
\SetKwFunction{CalculateV}{CalcVelocity}
\SetKwFunction{CalculateE}{CalcKEnergy}
\SetKwFunction{SumFM}{AggregateForcesAndMoments}
\SetKwFunction{SyncFM}{SynchronizeForcesAndMoments}
\SetKwFunction{SumM}{AggregateMasses}
\SetKwFunction{SyncM}{SynchronizeMasses}
\SetKwFunction{InterpX}{InterpolatePosition}
\SetKwFunction{InterpR}{InterpolatePosition}
\SetKwFunction{Reset}{Reset}

\SetKwData{Model}{model}
\SetKwData{Node}{node}
\SetKwData{Joint}{joint}
\SetKwData{Element}{element}
\SetKwData{CVG}{cvg}
\SetKwData{MAX}{maxIteration}

%\Input{A model to analyzed}
%\Output{An analyzed model}

% RUN
\Fn{\Run{}}{
\BlankLine
\ForEach(\tcc*[f]{$\vect{dx} = \vect{v_x}  dt$, $\vect{d\theta} = \vect{v_{\theta}}  dt$}){\Node in \Model}{
	\Move{$\vect{dx}, \vect{d\theta}$} \;
}
\BlankLine
\tcc{Elements calculate internal forces and moments}
\ForEach(){\Element in \Model}{
\CalculateF{$\vect{x}, \vect{d}_1,  \vect{d}_2,  \vect{d}_3$}
\tcc*[r]{$\vect{F}^{int}(\vect{x}, \vect{d}_1,  \vect{d}_2,  \vect{d}_3)$}

\CalculateM{$\vect{x}, \vect{d}_1,  \vect{d}_2,  \vect{d}_3$}
\tcc*[r]{$\vect{M}^{int}(\vect{x}, \vect{d}_1,  \vect{d}_2,  \vect{d}_3)$}
}
\BlankLine
\tcc{Joints coordinate the dynamic of several nodes}
\ForEach(){\Joint in \Model}{
\SumFM{} \;
\SumM{} \;
\SyncFM{} \;
\SyncM{} \;
}
\BlankLine
\tcc{Calculate translational kinetic energy}
\ForEach(){\Node in \Model}{
	\CalculateA{$m_x, \vect{F}$}
	\tcc*[f]{$\vect{a_x}(t) = \vect{R_x}/m_x$}
	
	\CalculateV{$\vect{a_x}, dt$}
	\tcc*[f]{$\vect{v_x}(t+\tfrac{dt}{2}) = \vect{v_x}(t-\tfrac{dt}{2}) + dt \vect{a_x}(t)$}
	
	\CalculateE{$\vect{v_x}$}
	\tcc*[f]{$E_x(t+\tfrac{dt}{2}) =  \tfrac{1}{2}\sum m_x \vect{v_x}^2(t+\tfrac{dt}{2})$}
}
\BlankLine
\tcc{Detect pic of kinetic energy}
\If{$E_x(t+\tfrac{dt}{2}) < E_x(t-\tfrac{dt}{2})$}{
	\InterpX{} \;
	\Reset{} \;
}

\BlankLine
\tcc{Calculate rotational kinetic energy}
\ForEach(){\Node in \Model}{
	\CalculateA{$m_{\theta}, \vect{M}$}
	\tcc*[f]{$\vect{a_{\theta}}(t) = \vect{R_x}/m_{\theta}$}
	
	\CalculateV{$\vect{a_{\theta}}, dt$}
	\tcc*[f]{$\vect{v_{\theta}}(t+\tfrac{dt}{2}) = \vect{v_{\theta}}(t-\tfrac{dt}{2}) + dt \vect{a_{\theta}}(t)$}
	
	\CalculateE{$\vect{v_{\theta}}$}
	\tcc*[f]{$E_{\theta} = \tfrac{1}{2}\sum m \vect{v_{\theta}}^2(t+\tfrac{dt}{2})$}
}
\BlankLine
\tcc{Detect pic of kinetic energy}
\If{$E_{\theta}(t+\tfrac{dt}{2})  < E_{\theta}(t-\tfrac{dt}{2})$}{
	\InterpR{} \;
	\Reset{} \;
}

}{\KwRet}

\caption{disjoint decomposition}\label{algo_disjdecomp}
\end{algorithm}
\end{fullpage}
\end{figure}
\DecMargin{2em}



\section{Conclusion}
Remind that the beam is subject to a distributed external force $\vect{f}_{ext}$ and a distributed external moment $\vect{m}_{ext}$.

We neglect rotational inertial effects on $\vect{d}_1$ et $\vect{d}_2$ in \eqref{eq:3_16_b} and \eqref{eq:3_16_c} which leads to the following shear force~:
\begin{equation}
	\vect{F}^{\perp}(s) = \vect{d}_3 \times (\vect{M}' + \vect{\varkappa} \times \vect{M} + \vect{m}_{ext})
\end{equation}
\begin{equation}
	\vect{F}^{\parallel}(s) = N \vect{d}_3
\end{equation}
\note{We may neglect as well the last term ($\tau \vect{M}$) and get back to the shear force obtained by the variational approach.} The total internal force acting on the beam is hence given by~:
\begin{equation}
	\vect{F}(s) = \vect{N}(s) + \vect{T}(s)
\end{equation}
Sections are subject to the following rotational moment around the centerline~:
\begin{equation}
	\begin{aligned}
	\vect{\Gamma}(s) = Q' + \vect{d}_3 \cdot ( \kappa\vect{b} \times \mat{M} + \vect{m}_{ext})
	\end{aligned}
\end{equation}

