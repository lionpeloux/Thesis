% !TEX encoding = UTF-8 Unicode
%%%%%%%%%%%%%%%%%%%%%%%%%%%%%%%%%%%%%%%%%%%%%%
%
%		Thesis Settings
%
%		EDOC Template
%		2011
%
%%%%%%%%%%%%%%%%%%%%%%%%%%%%%%%%%%%%%%%%%%%%%%
\documentclass[a4paper,11pt,fleqn]{book}
%\documentclass[a4paper,11pt,fleqn,draft]{book}

%\renewcommand*\ttdefault{txtt}

%\renewcommand*\familydefault{\ttdefault} %% Only if the base font of the document is to be typewriter style

\usepackage[T1]{fontenc}
\usepackage[utf8]{inputenc}
\usepackage[french,english]{babel}

% File Metadata



%%%%%%%%%%%%%%%%%%%%%%%%%%%%%%%%%%%%%%%%%%%%%%%
%% EDOC THESIS TEMPLATE: Variant 1.0 -> Latin modern, large text width&height
%%%%%%%%%%%%%%%%%%%%%%%%%%%%%%%%%%%%%%%%%%%%%%%

\usepackage[a4paper,top=32mm,bottom=28mm,inner=35mm,outer=25mm]{geometry}


% FONT
% ===========================
% see : https://en.wikibooks.org/wiki/LaTeX/Fonts
% see : https://tex.stackexchange.com/questions/213835/using-many-typewriter-fonts-in-a-single-document

% SELECT ROMAN FONT (Body)
% lmr (Latin Modern)
%\usepackage{lmodern}
%\renewcommand*{\familydefault}{lmr} 
\usepackage[scaled=0.85]{beraserif}

% SELECT TYPEWRITER FONT (Headings and Captions)
% see : http://www.tug.dk/FontCatalogue/typewriterfonts.html
% txtt (TXTT), cm(c)tt (Computer Modern), lm(v)tt (Latin Modern), qcr (Gyre Cursor), pcr (Adobe Courier)
\usepackage[scaled=0.85]{beramono}
%\renewcommand*{\ttdefault}{lmvtt} 

% HYP REF LINK IN PDF
\usepackage{hyperxmp} % pdf metadada (LDP)
%\usepackage[a-1b]{pdfx}
\usepackage{hyperref} % pdf metadada (LDP)

\hypersetup{pdfborder={0 0 0},
	colorlinks=true,
	linktoc=page,
	linkcolor=Tblue,
	citecolor=Tblue,
	urlcolor=Tblue} 
\urlstyle{same}

% HEADINGS
%\renewcommand*\ttdefault{cmvtt} % cmvtt or pcr
\usepackage{sectsty}
\newcommand{\headingfont}[1]{\ttfamily #1}
\allsectionsfont{\ttfamily}

% PARAGRAPH
% ===========================
\setlength{\parindent}{0pt}
\usepackage{setspace} % increase interline spacing slightly
\setstretch{1.1}

% TOCLOFT
% ===========================
\usepackage[titles]{tocloft}

% PART
\setlength{\cftbeforepartskip}{3em}
\renewcommand{\cftpartfont}{\ttfamily\LARGE\bfseries\scshape}
\renewcommand{\cftpartpagefont}{\ttfamily\LARGE\bfseries\scshape} 

% CHAP
\usepackage{etoolbox}

% frontmatter
\renewcommand{\cftchapfont}{\ttfamily\large\scshape}
\renewcommand{\cftchappagefont}{\ttfamily\large\scshape}
\let\cftchapfontbase\cftchapfont
\let\cftchappagefontbase\cftchappagefont
\preto{\frontmatter}{
\addtocontents{toc}{\protect\setlength{\cftbeforechapskip}{0.0em}}
}

% mainmatter
\let\cftchapfontnew\cftchapfont
\let\cftchappagefontnew\cftchappagefont
\renewcommand{\cftchapfontnew}{\ttfamily\large\scshape\bfseries}
\renewcommand{\cftchappagefontnew}{\ttfamily\large\scshape\bfseries}
\preto{\mainmatter}{
\addtocontents{toc}{\protect\addvspace{1em}}
\addtocontents{toc}{\protect\setlength{\cftbeforechapskip}{1.0em}}
\addtocontents{toc}{\protect\renewcommand{\protect\cftchapfont}{\protect\cftchapfontnew}}
\addtocontents{toc}{\protect\renewcommand{\protect\cftchappagefont}{\protect\cftchappagefontnew}}
}
\preto{\appendix}{
\addtocontents{toc}{\protect\setlength{\cftbeforechapskip}{1.0em}}
}

% backmatter
\preto{\backmatter}{
\addtocontents{toc}{\protect\addvspace{2em}}
\addtocontents{toc}{\protect\setlength{\cftbeforechapskip}{0.0em}}
\addtocontents{toc}{\protect\renewcommand{\protect\cftchapfont}{\protect\cftchapfontbase}}
\addtocontents{toc}{\protect\renewcommand{\protect\cftchappagefont}{\protect\cftchappagefontbase}}
}

% SEC
\renewcommand{\cftsecfont}{\ttfamily\normalsize\scshape}
\renewcommand{\cftsecpagefont}{\ttfamily\normalsize\scshape}

% SUBSEC
\renewcommand{\cftsubsecfont}{\ttfamily\small\scshape}
\renewcommand{\cftsubsecpagefont}{\ttfamily\normalsize\scshape}

% FIG
\setlength{\cftfigindent}{0em}
\setlength{\cftfignumwidth}{\cftsubsecnumwidth}
\renewcommand{\cftfigfont}{\ttfamily\small\scshape}
\renewcommand{\cftfigpagefont}{\ttfamily\small\scshape}

% TAB
\setlength{\cfttabindent}{0em}
\setlength{\cftfignumwidth}{\cftsubsecnumwidth}
\renewcommand{\cfttabfont}{\ttfamily\small\scshape}
\renewcommand{\cfttabpagefont}{\ttfamily\small\scshape}
 
 
 %%%%%%%%%%%%%%%%%%%%%%%%%%%%%%%%%%%%%%%%%%%%%%%



%%%%%%%%%%%%%%%%%%%%%%%%%%%%%%%%%%%%%%%%%%%%%%
% EDOC THESIS TEMPLATE: Variant 2.0 -> Utopia, Gabarrit A (lighter pages)
%%%%%%%%%%%%%%%%%%%%%%%%%%%%%%%%%%%%%%%%%%%%%%
%\usepackage{fourier} % Utopia font-typesetting including mathematical formula compatible with newer TeX-Distributions (>2010)
%\usepackage{utopia} % on older systems -> use this package instead of fourier in combination with mathdesign for better looking results
%\usepackage[adobe-utopia]{mathdesign}
%\setlength{\textwidth}{146.8mm} % = 210mm - 37mm - 26.2mm
%\setlength{\oddsidemargin}{11.6mm} % 37mm - 1in (from hoffset)
%\setlength{\evensidemargin}{0.8mm} % = 26.2mm - 1in (from hoffset)
%\setlength{\topmargin}{-2.2mm} % = 0mm -1in + 23.2mm
%\setlength{\textheight}{221.9mm} % = 297mm -29.5mm -31.6mm - 14mm (12 to accomodate footline with pagenumber)
%\setlength{\headheight}{14pt}
%%%%%%%%%%%%%%%%%%%%%%%%%%%%%%%%%%%%%%%%%%%%%%





\makeatletter
\setlength{\@fptop}{0pt}  % for aligning all floating figures/tables etc... to the top margin
\makeatother


\usepackage{graphicx}
\usepackage[table]{xcolor}
\graphicspath{{images/}}

\usepackage{subfig}
\usepackage{booktabs}
\usepackage{lipsum}
\usepackage{microtype}
\usepackage{url}
\usepackage[final]{pdfpages}

\usepackage{fancyhdr}
\renewcommand{\sectionmark}[1]{\markright{\thesection\ #1}}
\pagestyle{fancy}
	\fancyhf{}
	\renewcommand{\headrulewidth}{0.4pt}
	\renewcommand{\footrulewidth}{0pt}
	\fancyhead[OR]{\ttfamily \nouppercase{\rightmark}}
	\fancyhead[EL]{\ttfamily \nouppercase{\leftmark}}
	\fancyfoot[EL,OR]{\ttfamily\thepage}
\fancypagestyle{plain}{
	\fancyhf{}
	\renewcommand{\headrulewidth}{0pt}
	\renewcommand{\footrulewidth}{0pt}
	\fancyfoot[EL,OR]{\ttfamily\thepage} % \ttfamily\large\scshape\bfseries
}
\fancypagestyle{addpagenumbersforpdfimports}{
	\fancyhead{}
	\renewcommand{\headrulewidth}{0pt}
	\fancyfoot{}
	\fancyfoot[RO,LE]{\thepage}
}

\usepackage{listings}
\lstset{language=[LaTeX]Tex,tabsize=4, basicstyle=\scriptsize\ttfamily, showstringspaces=false, numbers=left, numberstyle=\tiny, numbersep=10pt, breaklines=true, breakautoindent=true, breakindent=10pt}






\makeatletter
\def\cleardoublepage{\clearpage\if@twoside \ifodd\c@page\else
    \hbox{}
    \thispagestyle{empty}
    \newpage
    \if@twocolumn\hbox{}\newpage\fi\fi\fi}
\makeatother \clearpage{\pagestyle{plain}\cleardoublepage}


%%%%% CHAPTER HEADER %%%%
\usepackage{color}
\usepackage{tikz}
\usepackage[explicit]{titlesec}
\newcommand*\chapterlabel{}
%\renewcommand{\thechapter}{\Roman{chapter}}
\titleformat{\chapter}[display]  % type (section,chapter,etc...) to vary,  shape (eg display-type)
	{\headingfont\bfseries\Huge} % format of the chapter
	{\gdef\chapterlabel{\thechapter\ }}     % the label
 	{0pt} % separation between label and chapter-title
 	  {\tikzset{external/export next=false} % LDP
	  \begin{tikzpicture}[remember picture,overlay]
    \node[yshift=-8cm] at (current page.north west)
      {\begin{tikzpicture}[remember picture, overlay]
        \draw[draw=Tblue, fill=Tblue] (0,0) rectangle(35.5mm,15mm); % color box
        \node[anchor=north east,yshift=-7.2cm,xshift=34mm,minimum height=30mm,inner sep=0mm] at (current page.north west)
        {\parbox[top][30mm][t]{15mm}{\raggedleft $\phantom{\texttt{l}}$\color{white}\chapterlabel}};  %the black l is just to get better base-line alignement
        \node[anchor=north west,yshift=-7.2cm,xshift=37mm,text width=\textwidth,minimum height=30mm,inner sep=0mm] at (current page.north west)
              {\parbox[top][30mm][t]{\textwidth}{\color{black}#1}}; % color title
       \end{tikzpicture}
      };
   \end{tikzpicture}
   \gdef\chapterlabel{}
  } % code before the title body
  
\titlespacing*{\chapter}{0pt}{50pt}{80pt} % LDP : change 30pt to 80pt after margin to avoid collision for not numbered chapters
\titlespacing*{\section}{0pt}{13.2pt}{*0}  % 13.2pt is line spacing for a text with 11pt font size
\titlespacing*{\subsection}{0pt}{13.2pt}{*0}
\titlespacing*{\subsubsection}{0pt}{13.2pt}{*0}

\newcounter{myparts}
\newcommand*\partlabel{}
\titleformat{\part}[display]  % type (section,chapter,etc...) to vary,  shape (eg display-type)
	{\headingfont\bfseries\Huge} % format of the part
	{\gdef\partlabel{\thepart\ }}     % the label
 	{0pt} % separation between label and part-title
 	  {\setlength{\unitlength}{20mm}
	  \addtocounter{myparts}{1}
	  \begin{tikzpicture}[remember picture,overlay]
    \node[anchor=north west,xshift=-65mm,yshift=-6.9cm-\value{myparts}*20mm] at (current page.north east) % for unknown reasons: 3mm missing -> 65 instead of 62
      {\tikzset{external/export next=false} % LDP
      \begin{tikzpicture}[remember picture, overlay]
        \draw[draw=Tblue, fill=Tblue] (0,0) rectangle(62mm,20mm);   % -\value{myparts}\unitlength
        \node[anchor=north west,yshift=-6.1cm-\value{myparts}*20mm,xshift=-55.0mm,minimum height=30mm,inner sep=0mm] at (current page.north east)
        {\parbox[top][30mm][t]{55mm}{\raggedright \color{white}Part \partlabel $\phantom{\texttt{l}}$}};  %the phantom l is just to get better base-line alingement / textt vs. texrm
        \node[anchor=north east,yshift=-6.1cm-\value{myparts}*20mm,xshift=-63.5mm,text width=\textwidth,minimum height=30mm,inner sep=0mm] at (current page.north east)
              {\parbox[top][30mm][t]{\textwidth}{\raggedleft \color{black}#1}};
       \end{tikzpicture}
      };
   \end{tikzpicture}
   \gdef\partlabel{}
  } % code before the title body

\usepackage{amsmath}
% Fix the problem with delimiter size caused by fourier and amsmath packages.
\makeatletter
\def\resetMathstrut@{%
  \setbox\z@\hbox{%
    \mathchardef\@tempa\mathcode`\(\relax
      \def\@tempb##1"##2##3{\the\textfont"##3\char"}%
      \expandafter\@tempb\meaning\@tempa \relax
  }%
  \ht\Mathstrutbox@1.2\ht\z@ \dp\Mathstrutbox@1.2\dp\z@
}
\makeatother
