\chapter{Geometry of smooth and discret space curves}
\section{Introduction}

Attention à la terminologie smooth vs. continious :

\emph{A smooth curve is a curve which is a smooth function, where the word "curve" is interpreted in the analytic geometry context. In particular, a smooth curve is a continuous map f from a one-dimensional space to an n-dimensional space which on its domain has continuous derivatives up to a desired order} 
\footnote{Definition of a smooth curve from mathworld : \url{http://mathworld.wolfram.com/SmoothCurve.html}}


\subsection{Goals and contributions}
Dans ce chapitre, après un bref rappel sur le cadre mathématique d'étude des courbes paramétrique de l'espace, on présente les notions de courbures et de torsion géométrique associées au repère de Frenet. On montre ensuite le cas plus général d'un repère mobile quelconque attaché à une courbe gamma. On définit enfin la particularité d'un repère mobile adapté à un courbe, et on présente, en sus du repère de Frenet, une approche différente pour accrocher des repères le long d'une courbe (Bishop / RMF / Zéro-twisting frame)

Contributions : présentation et comparaison de différentes façons d'estimer la courbure discrete

\subsection{Related work}

\cite{Bishop1975}
\cite{Bergou2008}
\cite{Hoffmann2008}
\cite{Bluth2014}
\cite{Frenet1852}
\cite{Delcourt2007}
\cite{Farouki2014}
\cite{Guggenheimer1989}
\cite{Klok1986}

\subsection{Overview}

% --------------------------------------------------------------------------------------------------------------------------------------------
% SMOOTH SPACE CURVE
% --------------------------------------------------------------------------------------------------------------------------------------------


\section{Paramectric curves}

% parametric curve
\subsection{Definition}
Let $I$ be an interval of $\mathbb{R}$ and $F\colon t \mapsto F(t)$ be a map of ${\mathcal{C}}^{}(I,{\mathbb{R}}^3)$. Then $\gamma=(I,F)$ is called a \emph{parametric curve} and :
\begin{itemize}
\item The 2-uplet $(I,F)$ is called a \emph{parametrization} of $\gamma$
\item $\gamma = F(I) = \{F(t), t \in I\}$ is called the \emph{graph} or \emph{trace} of $\gamma$
\item $\gamma$ is said to be ${\mathcal{C}}^{k}$ if $F \in {\mathcal{C}^{k}}^{}(I,{\mathbb{R}}^3)$
\end{itemize}

\begin{myrk}
Note that for a given graph in ${\mathbb{R}}^3$ they may be different possible parameterizations. From now, $\gamma$ will simply refer to $F(I)$, its graph.
\end{myrk}

% regularity
\subsection{Regularity}
Let $\gamma=(I,F)$ be a parametric curve, and $t_0 \in I$ a parameter.
\begin{itemize}
\item A point of parameter $t_0$ is called \emph{regular} if $F'(t_0) \neq 0$.
\\The curve $\gamma$ is called \emph{regular} if $\gamma$ is $\mathcal{C}^{1}$ and $F'(t) \neq 0, \forall t \in I$
\item A point of parameter $t_0$ is called \emph{biregular} if $F'(t_0)$ and $F''(t_0)$ are not collinear
\\The curve $\gamma$ is called \emph{biregular} if $\gamma$ is $\mathcal{C}^{2}$ and  $F'(t)\cdot    F''(t) \neq 0, \forall t \in I$
\end{itemize}

% reparametrization
\subsection{Reparametrization}
Let $\gamma=(I,F)$ be a parametric curve of class ${\mathcal{C}}^{k}$, $J \in {\mathbb{R}}^{3}$ an interval, and $\varphi\colon I\mapsto J$ a ${\mathcal{C}}^{k}$ diffeomorphisme. Lets define $G=F\circ\varphi$. Then :
\begin{itemize}
\item $G\in{\mathcal{C}}^{k}(J,{\mathbb{R}}^3)$
\item $G(J)=F(I)$
\item $\varphi$ is said to be an admissible \emph{change of parameter} for $\gamma$
\item  $(J,G)$ is said to be another \emph{admissible parametrization} for $\gamma$
\end{itemize}

% natural parametrization
\subsection{Natural parametrization}
Let $\gamma$ be a space curve of class ${\mathcal{C}}^{1}$. A parametrization $(I,F)$ of $\gamma$ is called \emph{natural} if $\|F'(t)\| = 1, \forall t \in I$. Thus :
\begin{itemize}
\item The curve is necessarily regular
\item F is strictly monotonic
\end{itemize}

% curve length
\subsection{Curve length}
Let $\gamma=(I,F)$ be a parametric curve of class ${\mathcal{C}}^{1}$. The length of $\gamma$ is define as :
\begin{equation}
L=\int_{I}\|F'(t)\|dt
\end{equation}
Note that the length of $\gamma$ is invariant under reparametrization.

% arc-length
\subsection{Arc-length parametrization}
Let $\gamma=(I,F)$ be a regular parametric curve of class ${\mathcal{C}}^{1}$. Let $t_0 \in I$ be a given parameter. The following map is said to be the \emph{arc-length of origin $t_0$} of $\gamma$ :
\begin{equation}
s \colon t \mapsto \int_{t_{0}}^{t}\|F'(u)\|du
\quad,\quad
s \in I \times \mathbb{R}
\end{equation}

The arc-length $s \colon I\mapsto s(I)$ is an admissible change of parameter for $\gamma$. Indeed, $s$ is a ${\mathcal{C}}^{1}$ diffeomorphisme because it is bijective ($s'>0$).

Lets define $G=F\circ s^{-1}$ and $J=s(I)$. Thus $(J,G)$ is a natural reparametrization of $\gamma$ and  $\|G'(s)\| = 1, \forall s \in J$.

This parametrization is preferred because the natural parameter s traverses the image of $\gamma$ at unit speed ($\|G'\| = 1$).


% --------------------------------------------------------------------------------------------------------------------------------------------
% FRENET THRIEDRON
% --------------------------------------------------------------------------------------------------------------------------------------------

\section{Frenet's trihedron}
\label{sec:frenet}
The trihedron of Frenet is a fundamental mathematical tool from the field of differential geometry to study local characterization of planar and non-planar space curves. It is a direct orthonormal basis attached to a point $P$ sliding along a parametric curve ($\gamma$). Introduced by Jean-Frédéric Frenet in his thesis upon \emph{curves of double curvature} in 1847, it brings out intrinsic local properties of space curves : the curvature ($\kappa$) which evaluates the deviance of $\gamma$ from being a straight line, and the torsion ($\tau$), which evaluates the deviance of $\gamma$ from being a plane curve. These quantities are also known as \guil{generalized curvatures}.
The \emph{fundamental theorem of space curves} states that a curve is fully determined by its curvature and torsion up to a solid (or euclidean) movement in space. This assertion is equivalent to the well-known \emph{Serret-Frenet formulas}, which give the first-order linear differential equations system that govern the evolution of Frenet's trihedron along a space curve. For a given curvature and torsion, and a given initial trihedron, the geometry of the space curve can be constructed by integration these differential equations.

In this section we consider $\gamma=(J,G)$ to be a regular ($\|\gamma'\|=1$) parametric curve of class ${\mathcal{C}}^{2}$, parametrized by its arc-length (denoted $s$). For the sake of simplicity we will refer to $G(s)$ as $\gamma(s)$.

\begin{figure}[t]
     \centering
     \subfloat[][Curve's tangent.]{\includegraphics{frenet_tangent.pdf}\label{<figure1>}}
     \subfloat[][Curve's normal and osculating circle.]{\includegraphics{frenet_normal.pdf}\label{<figure2>}}
     \caption{Differential definition of Frenet's trihedron at given point $P_0$.}
     \label{fig:3_1}
\end{figure}

% tangent vector
\subsection{Tangent vector}
The first vector of Frenet's trihedron is called the \emph{unit tangent vector} ($\vect{t}$). At any given parameter $s \in J$,  it is defined as :
\begin{equation}
\vect{t}(s) = \frac{\gamma'(s)}{\|\gamma'(s)\|} = \gamma'(s)
\quad,\quad
\|\vect{t}(s)\|=1
\end{equation}
In differential geometry, the tangente to the curve $\gamma$ at point $P_0$ is obtained as the limit of the (normalized) vector $\overrightarrow{P_0 P}$, as $P$ approches $P_0$ on the path $\gamma$ (\autoref{fig:3_1}). For a regular curve, the left-sided and right-sided limits coïncide as $P^-$ and $P^+$ approche $P_0$ respectively from its left and the right sides :
\begin{equation}
	\vect{t}(P_0)
	= \lim_{P \to P_0}\frac{\overrightarrow{P_0 P}}{\norm{\overrightarrow{P_0 P}}}
	= \lim_{P^- \to P_0}\frac{\overrightarrow{P_0 P^-}}{\norm{\overrightarrow{P_0 P^-}}}
	= \lim_{P^+ \to P_0}\frac{\overrightarrow{P_0 P^+}}{\|\overrightarrow{P_0 P^+}\|}
\label{eq:3_4}
\end{equation}

%Normal vector
\subsection{Normal vector}
The second vector of Frenet's trihedron is called the \emph{unit normal vector} ($\vect{n}$). It is constructed from $\vect{t'}$ which is orthogonal to $\vect{t}$ ($\|\vect{t}\|=1 \Rightarrow \vect{t^{'}} \cdot  \vect{t} = 0 \Leftrightarrow  \vect{t^{'}} \perp \vect{t}$). Thus, at any given parameter $s \in J$, it is defined as :
\begin{equation}
\vect{n}(s) = \frac{\vect{t}'(s)}{\|\vect{t}'(s)\|} = \frac{\gamma''(s)}{\|\gamma''(s)\|}
\quad,\quad
\|\vect{n}(s)\|=1
\end{equation}
Remark that the notion of \emph{normal vector} would be ambiguous for non-planar curves as far as there is an infinite number of possible vectors laying in the plane orthogonal to the curve's tangent. In practice, the tangent derivative is a convenient choice as it allows to extend the notion of curvature from planar to non-planar space curves. The tangent unit vector and the normal unit vector $\{\vect{t},\vect{n}\}$ define the so-called \emph{osculating plane}.

Likewise the differential definition of the tangent exposed in \eqref{eq:3_4}, the osculating plane could be seen as the limit of the plane defined by 3 points $P_0$, $P^-$, $P^+$, as $P^-$ and $P^+$ approches $P_0$ respectively from its left and right side.


%Binormal vector and torsion
\subsection{Binormal vector}
The third vector of Frenet's trihedron is called the \emph{unit binormal vector} ($\vect{b}$). It is constructed from $\vect{t}$ and $\vect{n}$ to form an orthonormal direct basis of $\mathbb{R}^{3}$. Thus, at any given parameter $s \in J$, it is defined as :
\begin{equation}
\vect{b}(s) = \vect{t}(s) \times \vect{n}(s)
\quad,\quad
\|\vect{b}(s)\|=1
\end{equation}
\begin{myrk}
The normal unit vector and the binormal unit vector $\{\vect{n},\vect{b}\}$ define the so-called \emph{normal plane}.
The normal tangent vector and the binormal unit vector $\{\vect{t},\vect{b}\}$ define the so-called \emph{rectifying plane}.
\end{myrk}

% Osculating plane
\subsection{Osculating plane}

As reported in \cite[p.45]{Delcourt2007}, the \emph{osculating plane} seems to have been first introduced by Johannis Bernoulli as the plane passing through three infinitely near points on a curve : \blockcquote[p.113]{Bernoulli1728}{Voco autem planum osculans, quod transit per tria curvae quaesitae puncta infinite sibi invicem propinqua}. 

In modern differential geometry the \emph{osculating plane} is defined as the limit of the plane passing through the points $P^-$, $P_0$ and $P^+$, while $P^-$ and $P^+$ approche $P_0$ (\autoref{fig:3_1}).

% --------------------------------------------------------------------------------------------------------------------------------------------
% CURVE INVARIANTS
% --------------------------------------------------------------------------------------------------------------------------------------------

\section{Curves of double curvature}

The study of space curves is a subset of the differential geometry field. The notion of \textquote{curve of double curvature} is attributed to Pitot \cite[p.28]{Delcourt2007}. However, as stated in \cite[p.321]{Coolidge2013}, curvature and torsion where probably first thought by Monge as testified by his paper from 1771 : \blockcquote[p.363]{Monge1785}{Mémoire sur les Développées, les Rayons de Courbure, et les Différents Genres d'Inflexions des Courbes a Double Courbure.}. It is also interesting to note that, at that time, \textquote{curvature} was also referred to as \textquote{flexure}.

Though, space curves were historically understood as \textquote{curves of double curvature} by extension of the case of planar curves, where the curvature measures the deviance of a curve from being a straight line. The second curvature, nowadays known as the \textquote{torsion} or \textquote{second generalized curvature}, measures the deviance of a curve from being plane. This two generalized curvatures, respectively the curvature and the torsion, are intrinsic curve properties and thus invariant regarding the choice of parametrization.
%Du rayon de courbure et des differens genres d"inflexions de courbes à double courbure.
\blockcquote[p.363]{Monge1809}{On appelle point d'inflexion, dans une courbe plane, le point où cette ligne, après avoir été concave dans un sens, cesse de l'être pour devenir concave dans l'autre sens. Il est évident que dans ce point, la courbe perd sa courbure, et que les deux élémens consécutifs sont en ligne droite. Mais une courbe à double courbure peut perdre chacune de ses courbures en particulier, ou les perdre toutes deux dans le même point ; c'est-à-dire, qu'il peut arriver ou que trois élémens consécutifs d'une même courbe à double courbure se trouvent dans un même plan, ou que deux de ces élémens soient en ligne droite. Il suit de là que les courbes à double courbure peuvent avoir deux espèces d'inflexions; la première a lieu lorsque la courbe devient plane, et nous l'appellerons simple inflexion; la seconde, que nous appellerons double inflexion, a lieu lorsque la courbe devient droite dans un de ses points.}

%Curvature
\subsection{First invariant : curvature}
In differential geometry, the \emph{osculating circle} is defined as the limit of the circle passing through the points $P^-$, $P_0$ and $P^+$, while $P^-$ and $P^+$ approche $P_0$ (\autoref{fig:3_1}). This circle lies on the osculating plane. While the tangent gives the best approximation of the curve as a straigth line, the osculating circle gives the best approximation of the curve as an arc (\autoref{fig:3_2}).

As explained by Euler, at a given arc-length parameter ($s$), the osculating plane is the plane in which a curve takes its curvature :  \blockcquote[p.364]{Euler1775}{in quo bina fili elementa proxima in curvantur}.

\subsubsection{Curvature}

The curvature is defined everywhere $\gamma$ is ${\mathcal{C}}^{2}$ as the norm of $\vect{t}'$ :
\begin{equation}
\kappa(s) = \|\vect{t}'(s)\| = \|\gamma''(s)\| \ge 0 
\quad,\quad
\vect{t}'(s) = \kappa(s) \vect{n}(s)
\end{equation}

\subsubsection{Radius of curvature}
One can demonstrate that, from a geometric point of view, $r(s) = 1/\kappa(s)$ represents the radius of the osculating circle of $\gamma$ at the point of parameter $s$ (\autoref{fig:3_2}).

%Osculating circle
\subsubsection{Center of curvature}
The center of curvature is the center of the osculating circle. It is defined as the limit of the intersection of the normal lines to the curve passing through the points $P^-$ and $P^+$, while $P^-$ and $P^+$ approche $P_0$ (\autoref{fig:3_1}).

\subsubsection{Curvature binormal vector}
Finally, following \cite{Bergou2008} we define the \emph{curvature binormal vector} at any given parameter $s \in J$ as :
\begin{equation}
\vect{\kappa b}(s) = \vect{t}(s) \times \vect{t}'(s) = \kappa(s)\cdot\vect{b}(s)
\quad,\quad
\|\vect{\kappa b}(s)\|= \kappa(s)
\end{equation}
This vector will be useful as it embed all the necessary information on the curve's curvature, defining both the direction of the \emph{osculating plane} and the radius of the \emph{osculating circle}.

\begin{figure}[t]
\centering
\includegraphics[]{osculating_circle.pdf}
\caption{Osculating circles for a spiral curve at different vertices.}
\label{fig:3_2}
\end{figure}

% Torsion
\subsection{Second invariant : torsion}

\note{torsion = second invariant = second curvature}

The \emph{torsion} of a curve is a concept from differential geometry.

En géométrie différentielle, la torsion d'une courbe tracée dans l'espace mesure la manière dont la courbe se tord pour sortir de son plan osculateur (plan contenant le cercle osculateur). Ainsi, par exemple, une courbe plane a une torsion nulle et une hélice circulaire est de torsion constante. Prises ensemble, la courbure et la torsion d'une courbe de l'espace en définissent la forme comme le fait la courbure pour une courbe plane. La torsion apparait comme coefficient dans les équations différentielles du repère de Frenet.

The \emph{torsion} measures the deviance of $\gamma$ from being a planar curve and is defined at any given parameter $s \in J$ as :
\begin{equation}
\tau_f(s) = \vect{n}'(s) \cdot \vect{b}(s)
\end{equation}

Cette notion est propre aux courbes gauches et mesure comment la courbe se "tord" en changeant de plan. Dans le trièdre de Frenet, elle correspond à l'angle des plans osculateurs P(s) et P(s + Ds) en deux points infiniment proches M(s) et M(s + Ds), donc à l'angle Dû entre les binormales mesurant comment la courbe se tord en passant de P(s) à P(s + Ds). Ainsi, de façon analogue à la courbure, la torsion T en un point sera, par unité d'arc, la limite lorsque Ds tend vers 0 du rapport Dû/Ds :

\begin{figure}[t]
\centering
\includegraphics[]{frenet_torsion.pdf}
\caption{Geometric torsion and rotation of the osculating plane}
\label{fig:3_3}
\end{figure}

%\begin{myprop}
%Si $F$ est ${\mathcal{C}}^{1}$ et régulière alors nécessairement elle est monotone car $F'$ est de signe constant.
%\end{myprop}

%\begin{figure}[t]
%\centering
%\includegraphics[width=\linewidth]{img_ch1_1_frenet.pdf}
%\caption{Repères de Frenet attachés à $\gamma$.}
%\label{fig:1_1}
%\end{figure}


% --------------------------------------------------------------------------------------------------------------------------------------------
% CURVE FRAMING
% --------------------------------------------------------------------------------------------------------------------------------------------


\section{Curve framing}

\note{
Vraiment repartir de l'intro de Bishop \citep[p.~1]{Bishop1975}


Nous avons précédement définit le repère de Frenet. Introduit les notions de courbure et de torsion, des invariants.

Nous précisions maintenant la notion de repère mobile le long d'une courbe à double courbure, c'est à dire d'une courbe gauche de l'espace.

Cette notion nous permettra, pour l'étude des poutres élancées, de positionner une section le long d'une fibre neutre.
}

% moving frame
\subsection{Moving frame}

Let $\gamma : s \rightarrow \gamma(s)$ be an arc-length parametrized curve. A map $F$ which associates to each point of arc-length $s$ a direct orthonormal trihedron is called a \emph{moving frame} :
\begin{equation}
	\fonctionL{F}{[0,L]}{\mathcal{SO}_{3}(\mathbb{R})}{s}{F(s) = \{\vect{e}_{3}(s),\vect{e}_{1}(s),\vect{e}_{2}(s)\}}
\end{equation}
Thus, inherently, a moving frame $F$ attached to $\gamma$ satisfies for all $s \in [0,L]$ :
\begin{equation}
	\left\{
	\begin{aligned}
		& \| \vect{e}_i(s) \| = 1 \\
		& \vect{e}_i(s) \cdot \vect{e}_j(s) = 0\quad , \quad i \neq j
	\end{aligned}
	\right.
\end{equation}
The terme \guil{moving frame} will refer indifferently to the map ($F$) itself, or to a specific evaluation of the map ($F(s)$).

% vecteur de Darboux
\begin{figure}[t]
\centering
\includegraphics[]{darboux.pdf}
\caption{Geometric interpretation of the Darboux vector of a moving frame.}
\label{fig:3_4}
\end{figure}

% governing equations
\subsubsection{Governing equations}
Computing the derivatives of the previous relationships leads to the following differential equations :
\begin{equation}
	\left\{
	\begin{aligned}
		&\vect{e}'_i(s) \cdot \vect{e}_i(s) = 0 \\
		&\vect{e}'_{i}(s) \cdot \vect{e}_{j}(s) = -\vect{e}_{i}(s) \cdot \vect{e}'_{j}(s)\quad , \quad i \neq j
	\end{aligned}
	\right.
\end{equation}
Thus, there exists 3 scalar functions $\tau(s)$, $k_{1}(s)$, $k_{2}(s)$ such that :
\begin{equation}
	\left\{
	\begin{aligned}
		&\vect{e}'_{3}(s) = k_{2}(s)\vect{e}_{1}(s) - k_{1}(s)\vect{e}_{2}(s) \\
		&\vect{e}'_{1}(s) = -k_{2}(s)\vect{e}_{3}(s) + \tau(s)\vect{e}_{2}(s) \\
		&\vect{e}'_{2}(s) = k_{1}(s)\vect{e}_{3}(s) - \tau(s)\vect{e}_{1}(s)
	\end{aligned}
	\right.
\end{equation}

It is common to rewrite this first-order linear differential equations system as a single matrix equation :
\begin{equation}
	\begin{bmatrix}
		\vect{e}'_{3}(s) \\
		\vect{e}'_{1}(s) \\
		\vect{e}'_{2}(s)
	\end{bmatrix}
	=
	\begin{bmatrix}
		0 & k_{2}(s) & -k_{1}(s) \\
		-k_{2}(s) & 0 & \tau(s) \\
		k_{1}(s) & -\tau(s) & 0
	\end{bmatrix}
	\begin{bmatrix}
		\vect{e}_{3}(s) \\
		\vect{e}_{1}(s) \\
		\vect{e}_{2}(s)
	\end{bmatrix}
\label{eq:3_12}
\end{equation}
Since the progression of any moving frame along $\gamma$ is ruled by a first-order differential equation, a unique triplet $\{\tau, k_{1}, k_{2}\}$ leads to a set of moving frames equal to each other within a constant of integration. Basically, with a given triplet $\{\tau, k_{1}, k_{2}\}$, one would \guil{propagate} a given initial direct orthonormal trihedron (at $s=0$ for instance) through the whole curve by integrating the differential system. In general, a moving frame will be fully determined by $\tau$, $\kappa_{1}$, $\kappa_{2}$ plus $\{\vect{e}_{3}(s=0),\vect{e}_{1}(s=0),\vect{e}_{2}(s=0)\}$.

\note{formules de Darboux-Ribaucour dans le cas des courbes tracées sur des surfaces - cas plus général que les formules de Serret-Frenet}

\subsubsection{Angular velocity : the Darboux vector}

\note{Rigoureusement, le vecteur de Darboux est le vecteur vitesse angulaire du repère de Frenet. Ici, on propose donc une généralisation de l'appelation}

It is relevant to consider the mobile frame's evolution along $\gamma$ introducing the so-called \emph{Darboux vector} ($\vect{\Omega}$), which corresponds to the instantaneous angular velocity of $F$ at each point of arc-length $s$. Thus, the previous differential system governing the evolution of $F(s)$ along $\gamma$ becomes :
\begin{equation}
	\vect{e}'_{i}(s) = \vect{\Omega}(s) \times \vect{e}_{i}(s)
	\quad avec \quad
	\vect{\Omega}(s)
	=
	\begin{bmatrix}
		\tau(s) \\
		k_1(s) \\
		k_2(s)
	\end{bmatrix}
\end{equation}
This result is straightforward deduced from \eqref{eq:3_12}. Note that the cross product \guil{reveals} that the system is skew-symmetric, which could already be seen in \eqref{eq:3_12}.
Geometrically, decomposing the infinitesimal rotation of the moving frame around its directors between arc-length $s$ and $s+ds$ (\autoref{fig:3_2}) shows that the scalar functions $\tau(s)$, $k_{1}(s)$, $k_{2}(s)$ effectively correspond to the angular speed of the frame, respectively around $\vect{e}_{3}(s)$, $\vect{e}_{1}(s)$, $\vect{e}_{2}(s)$ :
\begin{equation}
	\frac{d\theta_3}{dt}(s) = \tau(s)
	\quad,\quad
	\frac{d\theta_1}{dt}(s) = k_{1}(s)
	\quad,\quad
	\frac{d\theta_2}{dt}(s) = k_{2}(s)
\end{equation}

% adapted frame
\begin{figure}[t]
\centering
\includegraphics[]{adapted_moving_frame.pdf}
\caption{Adapted moving frame $F(s) =\{\vect{e}_{3}(s),\vect{e}_{1}(s),\vect{e}_{2}(s)\}$ where $\vect{e}_3(s) = \vect{t}(s)$.}
\label{fig:3_5}
\end{figure}

% ROTATION_MINIMIZING FRAME
\subsection{Rotation-minimizing frame}
Following \cite{Farouki2014} we introduce the notion of \emph{rotation-miniminzing frame}. Both the Frenet and Bishop adapted moving frames are also rotation-minimizing frames.

\subsection{Adapted moving frame}
Let $F$ be a moving frame as defined in the previous section. $F$ is said to be \emph{adapted} to $\gamma$ if at each point $\gamma(s)$, $\vect{e}_{3}(s)$ is tangent to $\gamma$ :
\begin{equation}
	\vect{d_{3}}(s) = \vect{t}(s) = \frac{\gamma^{'}(s)}{\|\gamma(s)\|}, \quad \forall s \in [0,L]
\end{equation}
For an adapted frame, the components $k_1$ and $k_2$ of the Darboux vector are related to the curve's curvature. Indeed, recall from that $\kappa \equiv \|\gamma''\| = \|\vect{t}'\|$. Or $\vect{t} = \vect{d}_3$ for an adapted frame. Thus, the following relation holds :
\begin{equation}
	\kappa = \|\vect{d}'_3\| = \sqrt{{k_{1}}^2 + {k_{2}}^2}
\end{equation}
\note{
La courbure est une quantité géométrique intrinsèque, indépendante du choix du repère mobile attaché à la courbe. C'est donc un invariant. Et donc quelque soit le choix du repère mobile adapté $\|\vect{t'}\| = \sqrt{\kappa_1^2 + \kappa_2^2}$ est un invariant (la courbure).

Faire le lien avec l'énergie de flexion, qui ne dépend donc que de la géométrie de la courbe dans le cas d'une isotropic rod $\mathcal{E}_b = EI\kappa^2$.}

% FRENET FRAME
\subsection{Frenet frame}

% definition
\subsubsection{Definition}
The Frenet frame is a well-known particular adapted moving frame (\autoref{sec:frenet}). At any given regular point $\gamma(s)$ it is define as $\{\vect{t}(s),\vect{n}(s),\vect{b}(s)\}$ where :
\begin{gather}
\vect{t}(s) = \frac{\gamma^{'}(s)}{\|\gamma'(s)\|}
\quad,\quad
\vect{n}(s) = \frac{\vect{t'}(s)}{\vect{\kappa}(s)}
\quad,\quad
\vect{b}(s)= \vect{t}(s)\times\vect{n}(s)
\end{gather}

\subsubsection{Governing equations}
The Frenet frame satisfies the \emph{Frenet-Serret} formulas, which govern the evolution of the frame along the curve $\gamma$ :
\begin{equation}
	\begin{bmatrix}
		\vect{t^{'}}(s) \\
		\vect{n^{'}}(s) \\
		\vect{b^{'}}(s)
	\end{bmatrix}
	=
	\begin{bmatrix}
		0 & \kappa_{}(s) & 0 \\
		-\kappa_{}(s) & 0 & \tau_f(s) \\
		0 & -\tau_f(s) & 0
	\end{bmatrix}
	\begin{bmatrix}
		\vect{t}(s) \\
		\vect{n}(s) \\
		\vect{b}(s)
	\end{bmatrix}
\label{eq:3_12}
\end{equation}
One can remember the generic differential equations of an adapted moving frame attached to a curve, where :
\begin{gather}
\vect{d_{3}}(s) = \vect{t}(s) = \frac{\gamma^{'}(s)}{\|\gamma(s)\|}
\quad,\quad
k_{1}(s) = 0
\quad,\quad
k_{2}(s) = \kappa(s)
\quad,\quad
\tau(s) = \tau_{f}(s)
\end{gather}



\subsubsection{Angular velocity}
Consequently, the Darboux vector ($\vect{\Omega_{f}}$) of the Frenet frame is given by :
\begin{equation}
	\vect{\Omega_f}(s)
	=
	\begin{bmatrix}
		\tau_{f}(s) \\
		0 \\
		\kappa(s)
	\end{bmatrix}
\end{equation}
One can remark that the Frenet frame satisfies $\vect{\Omega_f}(s) \cdot \vect{n}(s) = 0$ and is thus a \emph{rotation-miniminzing} frame regarding the normal vector ($\vect{n}$). The motion of this frame through the curve is known as "\emph{pitch-free}".

\subsubsection{Specific points}

\note{undefined when curvature vanishes : montrer des examples

not related to mechanical torsion

une perturbation de la courbe dans le sens le sens de la courbure engendre une variation de longueur de la courbe proportionnelle à l'inverse de la courbure (au premier ordre) + schéma

une perturbation de la courbe dans le sens de la binormale (en tout point) préserve la longueur de la courbe au 1er ordre : c'est un déplacement qui conserve l'hypothèse d'inextensibilité au premier ordre

Examiner la question de la fermeture sur une boucle fermée. Schéma.
}

% REPERE DE BISHOP
\subsection{Bishop frame}

% definition
\subsubsection{Definition}
Different ways to frame a curve. The usual one is Frenet. But, it could not be as relevant as we want in our field of interest.

The Bishop frame is defined as a a well-known particular adapted moving frame (\autoref{sec:frenet}). At any given regular point $\gamma(s)$ it is define as $\{\vect{t}(s),\vect{n}(s),\vect{b}(s)\}$ where :

\cite{Guggenheimer1989}
\cite{Klok1986}

Although the Frenet frame is not rotation-minimizing with respect to t, one can easily derive such a rotation-minimizing frame from it. New normal-plane vectors (u,v) are specified through a rotation of (p,b) according to u=cos

Bishop frame can be defined relatively to Frenet frame trough a rotation around the unit tangent.
The goal is to nullify the rotation around the tangent. As we know that the Frenet frame is rotating at speed $\tau(s)$ around $\vect{t}(s)$ we just have to rotate back the frenet frame around the tangent vector by the following angle :
\begin{equation}
	\theta(s) = \theta_0(s) - \int_0^s \tau_f(t)dt
\end{equation}
Bishop frame can be expressed relatively to the Frenet frame :
\begin{equation}
	\left\{
	\begin{aligned}
		&\vect{u} = \cos \theta \vect{n} +  \sin \theta \vect{b}\\
		&\vect{v} = -\sin \theta \vect{n} +  \cos \theta \vect{b}
	\end{aligned}
	\right.
\end{equation}
And :
\begin{equation}
	\left\{
	\begin{aligned}
		&\tau =\vect{u^{'}} \cdot \vect{v} = (\vect{\Omega_f} \times \vect{u} + \theta^{'} \vect{v})\cdot  \vect{v} = 0\\
		&k_1 = -\vect{t^{'}} \cdot \vect{v} = -\kappa \vect{n} \cdot \vect{v} = \kappa \sin \theta \\
		&k_2 = \vect{t^{'}} \cdot \vect{u} = \kappa \vect{n} \cdot \vect{u} = \kappa \cos \theta
	\end{aligned}
	\right.
\end{equation}

\subsubsection{Governing equations}
The Bishop frame evolution is governed by the following differential equations :
\begin{equation}
	\begin{bmatrix}
		\vect{t^{'}}(s) \\
		\vect{u^{'}}(s) \\
		\vect{v^{'}}(s)
	\end{bmatrix}
	=
	\begin{bmatrix}
		0 & \kappa(s) \sin \theta(s) & -\kappa(s) \cos \theta(s) \\
		-\kappa(s) \sin \theta(s) & 0 & 0 \\
		\kappa(s) \cos \theta(s) & 0 & 0
	\end{bmatrix}
	\begin{bmatrix}
		\vect{t}(s) \\
		\vect{u}(s) \\
		\vect{v}(s)
	\end{bmatrix}
\label{eq:3_12}
\end{equation}
One can remember the generic differential equations of an adapted moving frame attached to a curve, where :
\begin{gather}
k_{1}(s) = \kappa(s) \sin \theta(s)
\quad,\quad
k_{2}(s) = \kappa(s) \cos \theta(s)
\quad,\quad
\tau(s) = 0
\end{gather}

\subsubsection{Angular velocity}
Consequently, the Darboux vector ($\vect{\Omega_{b}}$) of the Bishop frame is given by :
\begin{equation}
	\vect{\Omega_b}(s) = \vect{\kappa b}(s) 
	=
	\begin{bmatrix}
		0\\
		\kappa(s) \sin \theta(s)\\
		\kappa(s) \cos \theta(s)
	\end{bmatrix}
\end{equation}
One can remark that the Bishop frame satisfies $\vect{\Omega_b}(s) \cdot \vect{t}(s) = 0$ and is thus \emph{rotation-miniminzing} regarding the tangent vector. roll-free motion.

\subsubsection{Specific points}
well defined when curvature vanishes

related to mechanical torsion

\note{expliquer la relation entre bishop et frenet : bishop est obtenu par rotation d'un angle $\alpha = \int \tau_f$ par rapport à frenet.

expliquer la notion de parallèle comme l'a formulé Laurent Hauswirth : la projection de $u'$ et $v'$ dans le plan normal à la tangente $t$ est nulle, cad que d'un plan à un autre la projection de $u$ et $v$ est conservée + faire schéma.

Laurent Hauswirth : la complexité d'un problème est en général proportionnelle à la codimension de l'objet étudié et donc, de ce fait les courbes ($codim = 3-1 = 2$) sont des objets plus compliqués que les surfaces ($codim = 3-2=1$) ds $\mathbb{R}^3$.

Expliquer le défaut de fermeture sur une boucle fermée. Calcul du writhe. Quelle différence avec Frenet ?
}

\subsection{Comparison between Frenet and Bishop frames}

\subsubsection{Application A : circular helix}
\begin{equation}
	\left\{
	\begin{array}{c}
		\rho = a\\
		z = b\theta
	\end{array}\right.
\end{equation}

\subsubsection{Application B : conical helix (spiral)}
\begin{equation}
	\left\{
	\begin{array}{c}
		\rho = a e^{k\theta}\\
		z = \rho \cot{\alpha}\\
	\end{array}\right.
\end{equation}


soit pour une spirale dont on connait

\begin{figure}[H]
\centering
\subfloat[][Frenet frame]{%
  \begin{tikzpicture}
\begin{axis}[
	width = 7.5cm,
	xmin=-0.1, xmax= 1.1, restrict x to domain = 0 : 1,
	ymin=-4e-2, ymax= 7e-2, restrict y to domain = -4e-2:6e-2,
	ytick={-3e-2, 0, 3e-2, 6e-2},
	grid=major,
	]

 	\pgfplotstableread{ch3_geometry/plot/frenet_bishop_darboux/frenet.txt}\frenet;
	\addplot [Tblue, smooth, thick]
       	table [x expr=\thisrowno{0}, y expr=\thisrowno{2}] {\frenet};

	\pgfplotstableread{ch3_geometry/plot/frenet_bishop_darboux/frenet.txt}\frenet;
	\addplot [black, smooth, thick, dashed]
       	table [x expr=\thisrowno{0}, y expr=\thisrowno{3}] {\frenet};

	\pgfplotstableread{ch3_geometry/plot/frenet_bishop_darboux/frenet.txt}\frenet;
	\addplot [black, smooth, thick]
       	table [x expr=\thisrowno{0}, y expr=\thisrowno{4}] {\frenet};

\end{axis}
  \end{tikzpicture}}\quad
\subfloat[][Bishop frame]{%
  \begin{tikzpicture}
\begin{axis}[
	width = 7.5cm,
	xmin=-0.1, xmax= 1.1, restrict x to domain = 0 : 1,
	ymin=-4e-2, ymax= 7e-2, restrict y to domain = -4e-2:6e-2,
	ytick={-3e-2, 0, 3e-2, 6e-2},
	grid=major,
	]

       	\pgfplotstableread{ch3_geometry/plot/frenet_bishop_darboux/bishop.txt}\bishop;
	\addplot [Tblue, smooth, thick]
         table [x expr=\thisrowno{0}, y expr=\thisrowno{2}] {\bishop};

         \pgfplotstableread{ch3_geometry/plot/frenet_bishop_darboux/bishop.txt}\bishop;
	\addplot [black, smooth, thick, dashed]
         table [x expr=\thisrowno{0}, y expr=\thisrowno{3}] {\bishop};

         \pgfplotstableread{ch3_geometry/plot/frenet_bishop_darboux/bishop.txt}\bishop;
	\addplot [black, smooth, thick]
         table [x expr=\thisrowno{0}, y expr=\thisrowno{4}] {\bishop};


\end{axis}
  \end{tikzpicture}}
\caption[]{Comparison between Frenet and Bishop frame velocity for a spirale curve.}
\label{fig:3_6}
\end{figure}


\section{Discret curves}

\subsection{Definition and arc-length parametrization}

\subsection{Discret curvature : an equivocal concept}

Curvature is defined from the osculating circle, which is the best approcximation of a curve by a circle.
We can define such a circle and it's radius will be the curvature at that point. Problem : there are several ways to define such a circle.

\subsubsection{Definition}

\note{
Très intéressant de constater que cette vision 3 verticies  vs. 2 edges est déjà présente dès le début dans l'histoire de la compréhension de la courbure.

Pour Euler, le rayon de courbure est le rapport de l’élément d’arc sur l’angle de contin-
gence entre deux tangentes infiniment proches. Par ailleurs, la définition du plan osculateur n’est pas tout à fait lamêmeque chez Bernoulli, plan passant par trois points consécutifs, puis- qu’Euler dit que ce plan contient deux éléments successifs. Il le définit aussi en disant que c’est le plan où la courbe s’incurve. Pour le dire de façon un peu différente : la tangente contient un élément, c’est le lieu où la courbe est droite, la plan osculateur représente l’étape suivante, c’est le lieu où la courbe est arc de cercle. Nous ne pensons pas trahir Euler en faisant cette présen- tation : cela justifie que, pour lui, il est naturel de se placer sur le plan osculateur pour calculer le rayon de courbure. \cite{Delcourt2007}

}

\cite{Hoffmann2008}

The edge osculating circle.
The vertex osculating circle.
La localité est meilleur dans le cas du vertex-based discret osculating circle.
Pour des anlges élevés, le edge-based discret osculating circle est plus pertinent.
La courbure tend vers l'infini quand les 2 edges deviennent colineaires.

La définition du plan osculateur est univoque dans le cas discret : c'est localement le plan défini par 2 edges consécutifs.

Ce n'est pas le cas de la courbure qui perd son côté intrinsèque.

courbure discrete dans le cas général

\subsubsection{Vertex-based osculating circle}

\begin{equation}
	\kappa_1 = \frac{2 \,sin(\varphi_i)}{\|\vect{e}_{i-1} + \vect{e}_{i} \|},
\end{equation}

\begin{figure}[]
\begin{center}
\includegraphics[]{curvature_vertex.pdf}
\caption{Variation of the vertex-based discrete curvature.}
\label{fig:3_7}
\end{center}
\end{figure}

\subsubsection{Edge-based osculating circle}

\begin{equation}
	\kappa_2 = \frac{tan(\varphi_i/2) + tan(\varphi_{i+1}/2)}{\|\vect{e}_{i}\|}
\end{equation}

The 3 consecutivs

\subsubsection{Osculating circle for an arc-length parametrized curve}

\begin{equation}
	\kappa_3 = \frac{4 \,tan(\varphi_i/2)}{\|\vect{e}_{i-1}\| + \|\vect{e}_{i} \|}
\end{equation}

\begin{equation}
	\kappa_3 = \frac{2 \,tan(\varphi_i/2)}{l} , l = \|\vect{e}_{i} \|
\end{equation}

\note{
Unlike the smooth case we can not reparameterize a curve. A discrete curve is parameterized by arc-length or it is not \citep[p. 10]{Hoffmann2008}.

Cette condition est extrêmement exigente $\|\vect{e}_{i} \| = cst$. Elle est tenable pour des modèles de poutre non connectées (où le pas de disctrétisation peut-être choisi uniform) mais pour en cas de connexion. Ce point n'est pas éclairci dans les articles de Audoly.
}

\begin{figure}[]
\begin{center}
\includegraphics[]{curvature_edge.pdf}
\caption{Variation of the edge-based discrete curvature.}
\label{fig:1_1}
\end{center}
\end{figure}



\begin{figure}[h]
     \centering
     \subfloat[][vertex-based osculating circle]{\includegraphics{osculating_circle_vertex.pdf}\label{<figure1>}}
     \subfloat[][edge-based osculating circle]{\includegraphics{osculating_circle_edge.pdf}\label{<figure2>}}
     \caption{Definition of the osculating circle for discrete curves.}
     \label{steady_state}
\end{figure}

\begin{figure}[H]
\begin{center}
\includegraphics[]{osculating_circle_edge_cst.pdf}
\caption{Another definition of the osculating circle for arc-length parametrized curves.}
\label{fig:1_1}
\end{center}
\end{figure}

\subsection{Variability of discrete curvature regarding $\alpha$}

Qu'on réécrit en posant $\|\vect{e}_{i-1}\| = \alpha \|\vect{e}_{i} \|$, $\alpha \geq 0$ :

\begin{equation}
	\kappa_1 = \frac{2 \,sin(\varphi_i)}{\|\vect{e}_{i}\|(1+ \alpha^2 + 2 \alpha \, cos(\varphi_i))^{1/2}},
	\quad
	\kappa_2 = \frac{4 \,tan(\varphi_i/2)}{\|\vect{e}_{i}\|(1+\alpha)}
\end{equation}

\begin{equation}
	\frac{\kappa_1}{\kappa_2}(\alpha) = \frac{\kappa_1}{\kappa_2}(1/\alpha)= \frac{1+\alpha}{(1+ \alpha^2 + 2 \alpha \, cos(\varphi_i))^{1/2}} \, cos^2(\varphi_i/2)
\end{equation}

\begin{figure}[H]
\begin{center}
\input{ch3_geometry/plot/plot_1.tex}
\end{center}
\caption{Discrete curvature comparison for $\alpha \in [0.5,2]$}
\end{figure}

\subsection{Convergence benchmark $\kappa_1$ vs. $\kappa_2$}

\subsubsection{Straight line}

\subsubsection{Circle}

Smooth curve settings:
\begin{equation}
	\mathcal{E} = \int_0^l \kappa^2 ds = \kappa \pi,
	\quad
	l = \pi r,
	\quad
	\kappa = \frac{1}{r}
\end{equation}

Discrete curve :
\begin{equation}
	\varphi_N = \tfrac{\pi}{N},
	\quad
	|\vect{e}| = 2r\sin \tfrac{\varphi}{2},
	\quad
	l_N = N|\vect{e}| =2Nr\sin \tfrac{\varphi}{2} = l \frac{\sin \tfrac{\varphi}{2}}{\tfrac{\varphi}{2}}
\end{equation}

Discrete bending energies :
\begin{equation}
	\mathcal{E}_1 = \mathcal{E} \frac{\sin \tfrac{\varphi}{2}}{\tfrac{\varphi}{2}},
	\quad
	\mathcal{E}_2 = \mathcal{E} \frac{\sin \tfrac{\varphi}{2}}{\tfrac{\varphi}{2} \cos^2 \tfrac{\varphi}{2}},
\end{equation}

Remarque that ratios are independent of scale change (independent of R)

\begin{figure}[H]
\begin{center}
\includegraphics[]{circle.pdf}
\caption{Another definition of the osculating circle for arc-length parametrized curves.}
\label{fig:1_1}
\end{center}
\end{figure}

qsmldkqsmldk s qsd qsd
 sqd
  qs dqs=dlk qs=ldk sq
\newpage
\begin{figure}[]
\begin{center}
\begin{tikzpicture}
	\begin{axis}[
	scale only axis,
	xmin=0, xmax= 21, restrict x to domain = 1 : 20,
	ymin=0.5, ymax= 1.7, restrict y to domain = 0 :1.6,
	grid=major,
	xlabel={$\mathcal{E}_i/\mathcal{E} = f_i(N=l/|e|)$},
	xtick={1,5,10,15,20},
	ylabel={},
	]
		\addplot[black, samples = 1000, domain = 1: 21, solid,  thin]
		{sin(deg(pi/(2*x)))/(pi/(2*x))};
		\node[style={font=\scriptsize}] at (axis cs:3.5,1.565) {$\mathcal{E}_1/\mathcal{E}$};

		\addplot[black, samples = 1000, domain = 1: 21, solid,  thin, dashed]
		{sin(deg(pi/(2*x)))/((pi/(2*x))*cos(deg(pi/(2*x)))^2)};
		\node[style={font=\scriptsize}] at (axis cs:2.5,0.65) {$\mathcal{E}_2/\mathcal{E}$};

		\addplot[Tblue, samples = 1000, domain = 0: 21, solid,  thick]
		{1};
	\end{axis}
\end{tikzpicture}
\end{center}
\caption{Discrete curvature comparison for $\alpha \in [0.5,2]$}
\end{figure}

\subsubsection{Elastica}

\begin{figure}[H]
\begin{center}
\includegraphics[]{elastica_withnum.pdf}
\caption{Another definition of the osculating circle for arc-length parametrized curves.}
\label{fig:1_1}
\end{center}
\end{figure}

\begin{figure}[H]
\begin{center}
\includegraphics[]{elastica_detail.pdf}
\caption{Another definition of the osculating circle for arc-length parametrized curves.}
\label{fig:1_1}
\end{center}
\end{figure}

\begin{figure}[H]
\centering
\subfloat[][$\mathcal{E}_1/\mathcal{E} = f(N = l/|e|)$]{%
  \begin{tikzpicture}
\begin{axis}[
	width = 7.5cm,
	xmin=8, xmax= 52, restrict x to domain = 10 : 50,
	ymin=0.915, ymax= 1.005, restrict y to domain = 0 :1.6,
	grid=major,
	xtick={0,3,5,10,15,20,25,30,35,40,45,50},
	]

       	\pgfplotstableread{ch3_geometry/plot/discrete_curvature_bench/elastica5.txt}\crvB;
	\addplot [black, smooth, very thin]
       	table [x expr=\thisrowno{0}, y expr=\thisrowno{3}] {\crvB};
%  	\node[Tgray, style={font=\scriptsize}] at (axis cs:9,1) {$5$};

       	\pgfplotstableread{ch3_geometry/plot/discrete_curvature_bench/elastica10.txt}\crvC;
	\addplot [black, smooth, very thin]
        	table [x expr=\thisrowno{0}, y expr=\thisrowno{3}] {\crvC};
    	\node[Tgray, style={font=\scriptsize}] at (axis cs:10,0.985) {$10$};

        \pgfplotstableread{ch3_geometry/plot/discrete_curvature_bench/elastica15.txt}\crvD;
	\addplot [black, smooth, very thin]
         table [x expr=\thisrowno{0}, y expr=\thisrowno{3}] {\crvD};
         \node[Tgray, style={font=\scriptsize}] at (axis cs:10,0.978) {$15$};

     	\pgfplotstableread{ch3_geometry/plot/discrete_curvature_bench/elastica20.txt}	\crvE;
	\addplot [black, smooth, very thin]
       	table [x expr=\thisrowno{0}, y expr=\thisrowno{3}] {\crvE};
       	\node[Tgray, style={font=\scriptsize}] at (axis cs:10,0.965) {$20$};

     	\pgfplotstableread{ch3_geometry/plot/discrete_curvature_bench/elastica25.txt}	\crvF;
	\addplot [black, smooth, very thin]
       	table [x expr=\thisrowno{0}, y expr=\thisrowno{3}] {\crvF};
       	\node[Tgray, style={font=\scriptsize}] at (axis cs:10,0.94) {$25$};

     	\pgfplotstableread{ch3_geometry/plot/discrete_curvature_bench/elastica30.txt}	\crvG;
	\addplot [black, smooth, very thin]
        	table [x expr=\thisrowno{0}, y expr=\thisrowno{3}] {\crvG};
       	\node[Tgray, style={font=\scriptsize}] at (axis cs:15,0.92) {$30$};

         \addplot[Tblue, samples = 100, domain = 0: 50, solid,  thick]
	{1};
\end{axis}
  \end{tikzpicture}}\quad
\subfloat[][$\mathcal{E}_2/\mathcal{E} = f(N = l/|e|)$]{%
  \begin{tikzpicture}
\begin{axis}[
	width = 7.5cm,
	xmin=8, xmax= 52, restrict x to domain = 10 : 50,
	ymin=0.995, ymax= 1.085, restrict y to domain = 1 :1.6,
	grid=major,
	xtick={0,3,5,10,15,20,25,30,35,40,45,50},
	]

         \pgfplotstableread{ch3_geometry/plot/discrete_curvature_bench/elastica5.txt}\crvB;
	\addplot [black, smooth, very thin]
         table [x expr=\thisrowno{0}, y expr=\thisrowno{4}] {\crvB};
         \node[Tgray, style={font=\scriptsize}] at (axis cs:10,1.024) {$5$};

        \pgfplotstableread{ch3_geometry/plot/discrete_curvature_bench/elastica10.txt}\crvC;
	\addplot [black, smooth, very thin]
         table [x expr=\thisrowno{0}, y expr=\thisrowno{4}] {\crvC};
         \node[Tgray, style={font=\scriptsize}] at (axis cs:10,1.044) {$10$};

         \pgfplotstableread{ch3_geometry/plot/discrete_curvature_bench/elastica15.txt}\crvD;
	\addplot [black, smooth, very thin]
         table [x expr=\thisrowno{0}, y expr=\thisrowno{4}] {\crvD};
         \node[Tgray, style={font=\scriptsize}] at (axis cs:10,1.07) {$15$};

     	\pgfplotstableread{ch3_geometry/plot/discrete_curvature_bench/elastica20.txt}	\crvE;
	\addplot [black, smooth, very thin]
         table [x expr=\thisrowno{0}, y expr=\thisrowno{4}] {\crvE};
         \node[Tgray, style={font=\scriptsize}] at (axis cs:13.5,1.08) {$20$};

     	\pgfplotstableread{ch3_geometry/plot/discrete_curvature_bench/elastica25.txt}	\crvF;
	\addplot [black, smooth, very thin]
         table [x expr=\thisrowno{0}, y expr=\thisrowno{4}] {\crvF};
         \node[Tgray, style={font=\scriptsize}] at (axis cs:18,1.08) {$25$};

     	\pgfplotstableread{ch3_geometry/plot/discrete_curvature_bench/elastica30.txt}	\crvG;
	\addplot [black, smooth, very thin]
         table [x expr=\thisrowno{0}, y expr=\thisrowno{4}] {\crvG};
         \node[Tgray, style={font=\scriptsize}] at (axis cs:27,1.08) {$30$};

         \addplot[Tblue, samples = 100, domain = 0: 50, solid,  thick]
	{1};
\end{axis}
  \end{tikzpicture}}
\caption[]{Bending energy representativity}
\label{g:nognot}
\end{figure}

\subsection{Edge versus vertex based tangent vector}

Problème de définition. Facile de définir une tangente sur un edge. Mais une infinité de tangentes possibles à chaque vertex.

So in case of an arc-length parameterized curve the vertex tangent vector points in the same direction as the averaged edge tangent vectors \cite[p. 12]{Hoffmann2008}.

Nous verrons que le cercle 3 points, en plus de mieux représenter l'énergie d'une courbe discrete dans les cas typiques, offre un choix de tangente non ambïgu.


\bibliographystyle{alpha}
\bibliography{../library}
