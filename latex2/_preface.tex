% !TEX encoding = UTF-8 Unicode
% Author :  Lionel du Peloux
% Contact : lionel.dupeloux@gmail
% Year : 2017

\chapter*{Préface}
\markboth{Préface}{Préface}
\addcontentsline{toc}{chapter}{Préface}
% put your text here

%\epigraph{Tu ne peux vivre que de cela que tu transformes, et dont un peu chaque jour, puisque tu t'échanges, tu meurs.}{Citadelle, Antoine de St Exupéry}
%
%\epigraph{Car tu n'as rien deviné de la joie si tu crois que l'arbre lui-même vit pour l'arbre qu'il est, enfermé dans sa graine. Il est source de graines ailées et se transforme et s'embélit de génération en génération. Il marche, non à ta façon, mais comme un incendie au gré des vents. Tu plantes un cèdre sur la montagne et voilà ta forêt qui lentement, au long des siècles, déambule.}{Citadelle, Antoine de St Exupéry}

%\begin{flushright}
%\small
%Tu ne peux vivre que de cela que tu transformes,\\
%et dont un peu chaque jour,\\
%puisque tu t'échanges contre,\\
%tu meurs.
%\end{flushright}


\epigraph{Tu ne peux vivre que de cela que tu transformes, et dont un peu chaque jour, puisque tu t'échanges contre, tu meurs.}%
{\textsc{Antoine de Saint-Exupéry}\\\textit{Citadelle}}
\vspace{1cm}

Si pour une raison quelconque il ne devait subsister qu'une unique page de ce manuscrit, j'aimerais autant que ce soit celle-là. Et qu'alors, seuls vivent les quelques mots de gratitude qui suivent pour les personnes qui m'ont accompagné sur ce chemin de fortune~; chemin initié en 2010 au sortir de l'Ecole Centrale et qui m'a conduit à présenter cette thèse.
\\
\\
Plus que la perspective d'une éventuelle contribution scientifique, c'est avant tout un certain goût pour la liberté \emph{d'aller et venir} qui m'a animé~: liberté des pieds qui vont~; liberté des mains qui font~; liberté de penser~; cette même liberté que je quête à travers mes sorties en montagne.
\\
\\
L'une de mes plus grandes chances aura été de faire participer mon corps tout entier à cette entreprise, de pouvoir d'un même mouvement concevoir et bâtir des gridshells, objets de mon étude, sans quoi ma compréhension du sujet serait restée beaucoup plus superficielle. Par ailleurs, les joies simples glanées sur les chantiers de ces projets atypiques -- je pense en particulier aux séminaires \emph{Construire le Courbe} avec des étudiants, à la construction du pavillon Solidays en 2011 avec des bénévoles et plus encore à la réalisation de la cathédrale éphémère de Créteil en 2013 avec des paroissiens -- furent pour moi sans égales avec celles, plus rares, reçues dans mon quotidien quelque peu taciturne de chercheur.
\\
\\
Chers Jean-François et Olivier, merci de m'avoir accueilli au sein de l'équipe MSA et d'avoir su me trouver une place sur mesure au fil de ces années. Merci pour la liberté que vous m'avez procurée et pour la confiance que vous m'avez accordée dans la conduite de mon travail de recherche, mais aussi dans certains projets annexes (e.g. solidays, thinkshell, booby). L'équipe s'est étoffée de nouveaux talents et la construction de cette dynamique vous dois beaucoup~: vous savez catalyser notre enthousiasme.
\\
\\
Cher Cyril, merci de ce compagnonnage de quelques années. Je me souviens de t'avoir (un peu) connu avant même de te rencontrer, par l'étude de ta thèse ! J'ai pris beaucoup de plaisir à travailler avec toi au cours de ces années et il en est sorti de beaux projets. Merci plus particulièrement pour les responsabilités que tu m'as confiées dans le séminaire Construire le Courbe et d'avoir accepté d'en chambouler le programme pédagogique d'année en année. Merci également pour ton écoute, tes conseils et ton précieux travail de relecture tout au long de l'élaboration de ce manuscrit. Sa qualité s'en est trouvée grandement améliorée.
\\
\\
Cher Bernard, nous avons partagé sans doute quelques angoisses sans nous le dire, mais cette cathédrale de Créteil restera pour moi un projet mémorable et intense. Merci de m'avoir fait confiance pour développer ce projet et d'avoir été présent dans les moments critiques de cette aventure. Plus qu'un bagage technique, j'ai appris durant ces trois années chez T/E/S/S une certaine façon de résoudre des problèmes, de chercher des solutions sans me décourager. Et cela m'a beaucoup profité dans mon travail de thèse et me restera acquis pour les années à venir. Merci donc à toi, Tom et Matt pour ce qui m'a été transmis au bureau.
\\
\\
Cher Frédéric, avec toi j'ai manié la clef dynamométrique comme jamais ! Ton travail a grandement contribué à la réussite des projets Solidays et Créteil. Tu es toujours disponible pour trouver une solution, bricoler un montage, faire fonctionner un four ou une fraise, imprimer une pièce en 3D, réparer des gouttières, partager ton analyse, donner un conseil, etc. J'ai beaucoup appris du travail que tu as initié au cours de ta thèse avec l'aide de Baptiste et dans la continuité duquel je m'inscris. Merci pour tout cela.
\\
\\
Merci chers Marine, Romain, Robert, Gilles, Marie, Tristan, Pierre M., Pierre C., Victor, \telp{} co-bureaux ou collègues de travail plus ponctuels, notamment lors des semaines \emph{Construire le Courbe}, pour les petits mots échangés ça et là lors d'un café ou d'un repas et pour votre enthousiasme quotidien. Merci Marie-Françoise, Christophe, Anne, Gilles, Géraldine, Alain, Hocine, pour l'aide constante apportée au cours de ces années passées au laboratoire.
\\
\\
Enfin, je ne serai pas allé au bout de ce travail sans le soutien des parents et des amis, nombreux, qui m'entourent quotidiennement. A vous tous, merci de votre soutien et de votre patience lors de ces derniers mois, avec une mention toute spéciale pour Blandine qui m'a gratifié de son affection indéfectible et qui a supporté mes horaires incongrues.








%qui dépassent largement les joies de la recherche.



%ce qui m'a animé fût un certain goût pour la liberté d'entreprendre et de créer
% à observer, à comprendre ou tout du moins à tenter de le faire.
%
%
%Mes motivations.
%Au dela de 
%Plus que la portée scientifique du travail que j'ai pu rassembler, c'est la liberté dont j'ai joui qui m'a 
%Loin des équations et autres résultats que j'ai 
%







%
%
%Si seule une page devait subsister de ce manuscrit, alors il faudrait que ce soit celle-là. 
%
%Des gens qui m'ont offert un espace de liberté et un lieu pour croître. 
%
%Si une seule page devrait rester de ce manuscript, je crois qu'il n'y en aurait pas d'autre que celle là.
%
%Cette
%
%
%Bien que cette thèse est été réalisée au terme d'un contrat de 3 ans, elle couvre une 
%Bien que 
%
%
%Les joies de la vie de tous les jours.
%
%JF, Olivier,
%
%BV, T/E/S/S
%
%
%Cyril, Fred, Baptiste, Pierre, Robert
%
%Romain, Marine, Pierre, Tristan
%
%Marie-Françoise, Alain, Gilles, Christophe, Géraldine Hocine et Anne.
% 
%Co-bureau, Marie et Pierre
%
%Tu ne peux vivre 


%Benvenuto 1991 : The battle between weight and rigidity constitutes, in itself, the single aesthetic theme of art in architecture: and to bring out this conflict in the most varied and clearest way is its office. Architecture accomplishes such a task, barring the direct route of free expansion to those indestructible forces, slowing them up by deflecting them; thus the battle continues and shows, in manifold forms, the unceasing efforts of the two opposing forces.

\bigskip
 \bigskip

\begin{flushright}
Lyon, le 4 novembre 2017\\
LdP.
\end{flushright}
