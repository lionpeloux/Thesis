% Author :  Lionel du Peloux
% Contact : lionel.dupeloux@gmail
% Year : 2017

\chapter{Introduction}\label{chp:intro}

La paternité des structures de type \emph{gridshell élastique} est couramment attribuée à l'architecte et ingénieur allemand Frei Otto, qui les a intensivement étudiés au XX\textsuperscript{ème} siècle. En 1975 il réalise, en collaboration avec l'ingénieur Edmund Happold du bureau Arup, un projet expérimental de grande ampleur~: la multi-halle de Mannheim. Cette réalisation emblématique ancrera durablement les gridshells dans le paysage des typologies structurelles candidates à l'avènement de géométries non-standard, caractérisées par l'absence d'orthogonalité. Cette capacité à \emph{former la forme} de façon efficiente prend tout son sens dans le contexte actuel où, d'une part la forme s'impose comme une composante prédominante de l'architecture moderne (Gehry, Hadid, \telp{}), et d'autre part l'enveloppe s'affirme comme le lieu névralgique de la performance des bâtiments, et plus spécifiquement de leur performance environnementale.

Littéralement, le terme \emph{gridshell} désigne une résille dont le comportement mécanique s'apparente à celui d'une coque ; c'est à dire que les efforts y transitent principalement de manière membranaire. Ces ouvrages peuvent franchir de grandes portées en utilisant un minimum de matière. Cependant, il semble plus rigoureux et plus fidèle à l'histoire de désigner par \emph{gridshell élastique} la combinaison indissociable d'un principe structurel – une résille qui fonctionne telle une coque – et d'une méthode constructive spécifique – qui tire partie d'une certaine souplesse de la résille pour sa mise en forme. Le projet de Mannheim – dans lequel une grille en bois, plane et régulière, sans rigidité de cisaillement, est déformée élastiquement jusqu'à la forme désirée via un dispositif d'étaiement, puis contreventée et finalement couverte – pose les bases de ce nouveau concept.

Il faudra attendre 25 ans et le développement des méthodes de calcul numérique, pour voir fleurir de nouvelles réalisations : en 2000 Shigeru Ban innovera en passant du bois au carton pour la construction du Pavillon de Hanovre  ; puis viendront les gridshells en bois de Downland en 2002 et de Savill en 2006 qui reprennent fidèlement les principes développés pour Mannheim mais emploient des méthodes constructives différentes.

Justification vs. Conception.

\blockcquote[Blbalb][p.~250]{IL10}{At present, the architectural significance of grid shells can hardly be fully envisioned. Too little experience has been gained as yet. However, the first signs of a grid shell architecture are already apparent. The projects planned and executed to date and the results of this research work indicate possibilities that open up a wide field of applications for grid shells. Two things stand out above all. The first is the nearly unlimited variety of forms that can be realized with grid shells, the second is the fact that there is no closed shell surface, but rather a simple, spatially curved grid composed of rods. Both features are of fundamental significance. They fully characterize the architectural essence of the grid shell support structure.}



%\section{Building free-forms}
%\subsection{Non-standard forms}
%\subsection{Importance of free-forms in modern architecture}
%\subsection{Canonical approaches to build free-forms}
%\subsection{Main challenges}

