% !TEX encoding = UTF-8 Unicode
% Author :  Lionel du Peloux
% Contact : lionel.dupeloux@gmail
% Year : 2017

\newrefsegment
\addtocontents{toc}{\protect\vspace*{\protect\fill}\protect\pagebreak}
\chapter*{Introduction}\label{chp=intro}
\addcontentsline{toc}{chapter}{Introduction}
\markboth{Introduction}{Introduction} % necessary to have the correct header with \chapter* and book.cls

\begin{otherlanguage}{french}

%The quick brown fox jumps over the lazy dog

La paternité des structures de type \emph{gridshell élastique} est couramment attribuée à l'architecte et ingénieur allemand Frei Otto, qui les a intensivement étudiées au XX\textsuperscript{ème} siècle. Fruit de son travail de recherche, il réalise en 1975, en collaboration avec l'ingénieur Edmund Happold du bureau Arup, un projet expérimental de grande ampleur~: la Multihalle de Mannheim \cite{IL13,Happold1975}. Cette réalisation emblématique ancrera durablement les gridshells dans le paysage des typologies structurelles candidates à l'avènement de géométries non-standard, caractérisées par l'absence d'orthogonalité. Cette capacité à \emph{former la forme} de façon efficiente prend tout son sens dans le contexte actuel où, d'une part la forme s'impose comme une composante prédominante de l'architecture moderne (F. Gehry, Z. Hadid,~\telp{}) et d'autre part l'enveloppe s'affirme comme le lieu névralgique de la performance des bâtiments, notamment environnementale.

Littéralement, le terme \emph{grid-shell} désigne une résille à double courbure dont le comportement mécanique s'apparente à celui d'une coque ; c'est à dire que les efforts y transitent principalement de manière membranaire. Ces ouvrages peuvent franchir de grandes portées en utilisant un minimum de matière. Cependant, il semble plus rigoureux et plus fidèle à l'histoire de désigner par \emph{gridshell élastique} la combinaison indissociable d'un principe structurel -- le gridshell, une résille qui fonctionne telle une coque -- et d'une méthode constructive astucieuse -- la déformation réversible d'une grille de poutre initialement plane pour former une surface tridimensionnelle à double courbure. Le projet de Mannheim -- dans lequel une grille en bois de trame régulière, initialement plane et sans rigidité de cisaillement est déformée élastiquement jusqu'à la forme désirée via un dispositif d'étaiement, puis contreventée pour mobiliser la raideur d'une coque et finalement couverte d'une toile -- pose les bases de ce nouveau concept et le rend populaire auprès d'un large public d'architectes et d'ingénieurs de par le monde.

Cependant, en dépit du potentiel de cette typologie, très peu de projets ont vu le jour suite à la construction de la Multihalle. Il faut en effet attendre 25 ans et le développement des méthodes de calcul numérique pour voir de nouveau éclore quelques réalisations iconiques~: Shigeru Ban innove en passant du bois au carton pour la construction du Pavillon de Hanovre en 2000 \cite{Ban2006}~; puis viennent les gridshells en bois de Downland en 2002 \cite{Harris2002} et de Savill en 2006 \cite{Harris2008} qui reprennent fidèlement les principes développés à Mannheim mais emploient des méthodes constructives différentes. Depuis une dizaine d'années le laboratoire Navier a investi ce champ de recherche sous le double aspect de la structure et du matériau, donnant lieu à la réalisation de quelques prototypes (en 2006 et 2007 \cite{Douthe2006,Douthe2010a}) et des deux premiers bâtiments de type gridshell élastique en matériau composite construits à ce jour (Solidays 2011 \cite{Baverel2012} et Créteil 2013 \cite{DuPeloux2016}).\footnote{Ici, le matériau employé, un composite à base de fibres de verre imprégnées dans une matrice polyester et obtenu par pultrusion, apporte un gain de performance très significatif par rapport au bois et permet de rester sur une conception à simple nape là où le bois aurait nécessité une grille à double nape beaucoup plus complexe à réaliser.} Plus récemment, on a pu observer un certain engouement pour la construction de pavillons en bois de petite taille, non couverts, réalisés selon des principes similaires à ceux de la Multihalle, essentiellement dans le cadre de workshops pédagogiques ou bien de projets de recherche \cite{DAmico2014,Naicu2014,DAmico2015a,Mork2016}.

Il est naturel de se demander pourquoi cette innovation prometteuse peine ainsi à essaimer~?~S'il est vrai que la construction de la Multihalle de Mannheim a permis de prouver la faisabilité économique et technique du concept de gridshell élastique à grande échelle, il faut bien reconnaître que cette prouesse n'a été rendue possible qu'au terme d'un long processus de maturation pour développer et acquérir l'ensemble des compétences scientifiques, techniques, méthodologiques et humaines nécessaires à sa conception et à sa construction.\footnote{\blockcquote[Georg Lewenton][p.~201]{IL13}{This is not a case of a building creatively designed, but based on a support system of additive known elements. This design is the result of a symposium of creative thought in the formation, the invention of building elements with the simultaneous integration of the theoretical, scientific contributions from mathematics, geodesy, model measuring, statics as well as control loading and calculation. We are dealing with more than pure \textquote{teamwork}, we are dealing with team creation.}}

En vérité, une telle dépense de moyens pour développer et rassembler ces compétences ne saurait assurer la reproductibilité de cette experience sauf en de très rares occasions et pour des projets d'exception. Par ailleurs, les techniques développées à l'époque sont pour partie tombées en désuétude (e.g. la recherche de forme par maquette physique) ou bien ont fortement évoluées voir même mutées (e.g. le calcul numérique). Des matériaux nouveaux, composites, ont vu le jour. Ils repoussent les limitations intrinsèques des matériaux usuels tel que le bois et offrent des performances techniques bien plus intéressantes pour ce type d'application (durabilité, allongement à la rupture, légèreté, résistance mécanique, fiabilité de niveau industrielle, \telp{}). Enfin, notons que le cadre réglementaire s'est considérablement étoffé apportant aussi son lot de rigidités vis-à-vis de la pénétration des innovations dans le secteur de la construction.

Ainsi la conception des gridshells se pose-t-elle en des termes nouveaux aux architectes et ingénieurs actuels. Elle se heurte aux deux difficultés majeures suivantes~:
\begin{itemize}
\item
La première difficulté est d'ordre technique et concerne la fonctionnalisation de la structure. En effet, bien que le principe du gridshell permette de réaliser des ossatures courbes de manière optimisée, il n'en reste pas moins complexe de constituer à partir de cette résille porteuse une véritable enveloppe de bâtiment capable de répondre à un large panel de critères performantiels (tels que l'étanchéité, l'isolation thermique, l'isolation acoustique,~\telp{}) sur un support qui ne présente aucune rationalité géométrique.\footnote{Pour contourner cette difficulté, une approche prometteuse consiste à identifier des classes de surfaces courbes (comme les maillages isoradiaux) dont certaines propriétés géométriques (e.g. facettes planes, noeuds sans torsion) s'avèrent avantageuses sur le plan constructif \cite{Mesnil2017}.}
\item
La seconde difficulté est d'ordre théorique et concerne la mise au point d'outils et de processus de conception adaptés à l'étude de ces structures d'un genre nouveau où Architecture et Ingénierie collaborent de manière indissociable à l'identité formelle de l'ouvrage. L'inadéquation des méthodes et des outils de design actuels, orientés davantage vers la justification des ouvrages que vers leur conception, constitue un des principaux freins à la diffusion de cette innovation.
\end{itemize}

Le présent manuscrit s'articule autour de deux grandes parties qui tentent chacune de construire des éléments de réponse aux défis identifiés précédemment. La première partie, composée des chapitres 1 et 2, est destinée à présenter en profondeur le concept de gridshell élastique, son potentiel et les difficultés techniques sous-jacentes (voir \cref{part=1}). La seconde partie, composée des chapitres 3 à 6, est consacrée au développement d'un élément de poutre discret original prenant en compte les sollicitations de flexion et de torsion et applicable à tout type de section dont le centre de torsion est confondu avec le centre de masse, ainsi que certains types de discontinuités liées à la présence de connexions dans les résilles de type gridshell (voir \cref{part=2}). Cette seconde partie constitue le coeur \emph{académique} de ce travail de thèse.

Dans le \cref{chp=gridshell} nous rappelons la genèse de cette invention et nous en donnons une définition précise et actualisée. Puis nous dressons un état des lieux critique des projets réalisés sur ce principe depuis le début des années 1960 à nos jours. Cette brève histoire des gridshells dessine à elle seule le potentiel de ces structures, notamment en terme d'expression formelle et de performance structurelle. Loin de les enfermer dans un style d'architecture particulier, elle en souligne au contraire la formidable variété. Cette revue de projet est complétée par une revue approfondie de la littérature existante sur l'ensemble des domaines connexes à cette thématique (géométrie, structure, matériaux, logiciel).

Dans le \cref{chp=creteil} nous présentons de manière détaillée la conception et la réalisation de la cathédrale éphémère de Créteil, un gridshell élastique en matériau composite construit en 2013 et toujours en service. Cette expérience peu commune a été une source inépuisable pour alimenter ce travail de thèse. Cette relecture expose les méthodes et les outils de conceptions développés pour faire aboutir le projet, les difficultés rencontrées, les pistes d'amélioration. Elle fournit également une analyse économique pour cerner les axes de progrès prioritaires dans l'optique d'une commercialisation future.

Dans le \cref{chp=curve} nous rappelons les notions fondamentales déjà connues, indispensables à notre étude, pour la caractérisation géométrique de courbes de l'espace et de repères mobiles attachés à des courbes. Ces notions sont présentées pour le cas continu puis pour le cas discret~; ce dernier étant essentiel pour la résolution numérique de notre modèle. Cependant, nous observons que la notion clef de courbure géométrique perd son univocité dans le cas discret. Nous identifions alors plusieurs définitions de la courbure discrète. Puis nous les comparons selon des critères propres à notre application (convergence géométrique, représentativité énergétique, forme d'interpolation). A l'issu de cette analyse, la définition la plus pertinente est retenue pour le développement du nouveau modèle numérique au cours des chapitres suivants.

Dans le \cref{chp=energy} nous élaborons un premier modèle de poutre à \dofs{4} par une approche variationnelle. Ici nous reprenons et enrichissons un travail initié lors d'une précédente thèse \cite{Tayeb2015} inspirée par des travaux récents sur la simulation des tiges élastiques dans le domaine des \emph{computer graphics} \cite{Bergou2008}, et à laquelle j'ai collaboré \cite{DuPeloux2015,Lefevre2017}. En particulier, notre développement permet d'aboutir à des expressions purement locales des efforts internes et prouve l'équivalence avec le membre statique des équations de Kirchhoff. Sur le plan mathématique, le modèle est développé en continu et son implémentation numérique n'est pas traité.

Dans le \cref{chp=kirchhoff} nous développons une nouvelle approche, plus directe et plus complète, pour construire à partir des équations de Kirchhoff un élément de poutre  enrichi par un noeud fantôme et possédant lui aussi un nombre de degré de liberté minimal. L'originalité de cet élément est de pouvoir localiser proprement dans l'espace certains types de discontinuités, notamment des discontinuités de courbures provoquées par des efforts ponctuels ou des sauts de propriétés matérielles. Cela permet une modélisation plus fine des phénomènes physiques au sein de la grille, aussi bien au niveau des connexions que des conditions aux appuis, ce qui était le principal objectif de ce travail de thèse.

Dans le \cref{chp=numerical_model} nous combinons les résultats des chapitres précédents pour construire un élément de poutre discret tout à fait adapté à la modélisation numérique des gridshells élastiques. Nous présentons la construction de cet élément et la méthode de résolution numérique employée pour trouver l'état d'équilibre statique du système, à savoir le relaxation dynamique. Enfin, nous donnons quelques éléments sur \emph{Marsupilami}, le programme informatique que nous avons mis au point et qui implémente l'élément de poutre discret élaboré au cours de cette thèse. Nous exposons aussi quelques résultats de comparaison avec des logiciels du commerce qui ont permi de valider notre travail. Plus généralement, l'élément développé convient bien pour modéliser des problèmes de couplage flexion-torsion dans des poutres élancées, comme par exemple les phénomènes de repositionnement des câbles et des gaînes accrochées aux bras robots, un matériel industriel qui se démocratise à grande vitesse.

\end{otherlanguage}
