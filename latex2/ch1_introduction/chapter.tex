% Author :  Lionel du Peloux
% Contact : lionel.dupeloux@gmail
% Year : 2017

% contexte historique et potentiel contemporain
% définition du terme gridshell élastique
% on constate peu de réalisation
% Mannheim, une preuve de concept mais dans un cadre non reproductible
% Dans quels termes ce pose le problème de la conception aujourd'hui ?

% pourquoi ca coince : technologie / méthodologie / théorie

% ce que propose cette thèse
% expérience personnelle (relecture) : thèse initiée en octobre 2014 mais qui couvre une periode plus large. Qui regroupe deisng to fabrication at large scale prototypes.
% partie technologique


\chapter{Introduction}\label{chp:intro}

La paternité des structures de type \emph{gridshell élastique} est couramment attribuée à l'architecte et ingénieur allemand Frei Otto, qui les a intensivement étudiées au XX\textsuperscript{ème} siècle. Fruit de son travail de recherche, il réalise en 1975, en collaboration avec l'ingénieur Edmund Happold du bureau Arup, un projet expérimental de grande ampleur~: la multi-halle de Mannheim. Cette réalisation emblématique ancrera durablement les gridshells dans le paysage des typologies structurelles candidates à l'avènement de géométries non-standard, caractérisées par l'absence d'orthogonalité. Cette capacité à \emph{former la forme} de façon efficiente prend tout son sens dans le contexte actuel où, d'une part la forme s'impose comme une composante prédominante de l'architecture moderne (F. Gehry, Z. Hadid, \telp{}), et d'autre part l'enveloppe s'affirme comme le lieu névralgique de la performance des bâtiments, notamment environnementale.

Littéralement, le terme \emph{grid-shell} désigne une résille à double courbure dont le comportement mécanique s'apparente à celui d'une coque ; c'est à dire que les efforts y transitent principalement de manière membranaire. Ces ouvrages peuvent franchir de grandes portées en utilisant un minimum de matière. Cependant, il semble plus rigoureux et plus fidèle à l'histoire de désigner par \emph{gridshell élastique} la combinaison indissociable d'un principe structurel -- le gridshell, une résille qui fonctionne telle une coque -- et d'une méthode constructive astucieuse -- la déformation réversible d'une grille de poutre initialement plane pour former une surface tridimensionnelle à double courbure. Le projet de Mannheim -- dans lequel une grille en bois de trame régulière, sans rigidité de cisaillement et initialement plane est déformée élastiquement jusqu'à la forme désirée via un dispositif d'étaiement, puis contreventée et finalement couverte d'une toile -- pose les bases de ce nouveau concept et le rend populaire auprès d'un large public d'architectes et d'ingénieurs de par le monde.

Cependant, en dépit du fort potentiel de cette typologie, très peu de projets ont vu le jour suite la construction de la Multihalle. Il faut en effet attendre 25 ans et le développement des méthodes de calcul numérique pour voir de nouveau éclore quelques réalisations iconiques~: Shigeru Ban innove en passant du bois au carton pour la construction du Pavillon de Hanovre en 2000~; puis viennent les gridshells en bois de Downland en 2002 et de Savill en 2006 qui reprennent fidèlement les principes développés à Mannheim mais emploient des méthodes constructives différentes. Depuis une dizaine d'années le laboratoire Navier a investi ce champ de recherche sous le double aspect de la structure et du matériau, donnant lieu à la réalisation de quelques prototypes (2006, 2007) et des deux premiers bâtiments (Solidays 2011, Créteil 2013) de type gridshell élastique en matériau composite construits à ce jour. Plus récemment, on a pu observer un certain engouement pour la construction de pavillons en bois de petite taille, non couverts, réalisés selon des principes similaires à ceux de la Multihalle, essentiellement dans le cadre de workshops pédagogiques ou bien de projets de recherche.

Il est naturel de se demander pourquoi cette innovation prometteuse peine ainsi à essaimer~? S'il est vrai que la construction de la Multihalle de Mannheim a permis de prouver la faisabilité économique et technique du concept de gridshell élastique à grande échelle, il faut bien reconnaître que cette prouesse n'a été rendue possible qu'au terme d'un long processus de maturation pour développer et acquérir l'ensemble des compétences scientifiques, techniques, méthodologiques et humaines nécessaires à sa conception et à sa construction~: \blockcquote[Georg Lewenton][p.~201]{IL13}{This is not a case of a building creatively designed, but based on a support system of additive known elements. This design is the result of a symposium of creative thought in the formation, the invention of building elements with the simultaneous integration of the theoretical, scientific contributions from mathematics, geodesy, model measuring, statics as well as control loading and calculation. We are dealing with more than pure \textquote{teamwork}, we are dealing with team creation.}

En vérité, une telle dépense de moyens pour développer et rassembler ces compétences ne saurait assurer la reproductibilité de cette experience sauf en de très rares occasions et pour des projets d'exception. Par ailleurs, les techniques développées à l'époque sont pour partie tombées en désuétude (e.g. la méthode du filet inverse) ou ont fortement évoluées voir même mutées (e.g. le calcul numérique). Les cadres réglementaires se sont considérablement étoffés apportant aussi leur lot de rigidités vis-à-vis de la pénétration des innovations. Ainsi la conception des gridshells se pose-t-elle en des termes nouveaux aux architectes et ingénieurs actuels.

Aujourd'hui, la conception d'un tel ouvrage ce heurte à deux difficultés majeures. 
La première difficulté est d'ordre technique et concerne la fonctionnalisation de la structure. En effet, bien que le principe du gridshell permette de réaliser des ossatures courbes de manière optimisée, il n'en reste pas moins délicat de constituer à partir de cette résille porteuse une véritable enveloppe de bâtiment (capable de répondre à un large panel de critères performantiels tels que l'étanchéité, l'isolation thermique, l'isolation acoustique, \telp{}) sur un support qui ne présente aucune rationalité géométrique. 
La seconde difficulté est d'ordre théorique et méthodologique et concerne la mise au point d'outils et de processus de conception adaptés à l'étude de ces structures d'un genre nouveau. Intrication Architecte/.Ingénieur Justification vs. Conception. Agilité du processus de conception. Les outils de calcul actuels.


\blockcquote[Frei Otto][p.~250]{IL10}{At present, the architectural significance of grid shells can hardly be fully envisioned. Too little experience has been gained as yet. However, the first signs of a grid shell architecture are already apparent. The projects planned and executed to date and the results of this research work indicate possibilities that open up a wide field of applications for grid shells. Two things stand out above all. The first is the nearly unlimited variety of forms that can be realized with grid shells, the second is the fact that there is no closed shell surface, but rather a simple, spatially curved grid composed of rods. Both features are of fundamental significance. They fully characterize the architectural essence of the grid shell support structure.}



%\section{Building free-forms}
%\subsection{Non-standard forms}
%\subsection{Importance of free-forms in modern architecture}
%\subsection{Canonical approaches to build free-forms}
%\subsection{Main challenges}

